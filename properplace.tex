\RequirePackage[l2tabu, orthodox]{nag}%less problems with LaTeX code
\RequirePackage{silence}\WarningFilter{newunicodechar}{Redefining Unicode character}
\pdfgentounicode=1 %permits (with package glyphtounicode) to copy eg x ⪰ y iff v(x) ≥ v(y) from pdf to unicode data. 
\input{glyphtounicode}%nice copy from PDF
\documentclass[preprint, french, english, 11pt, authoryear]{elsarticle}%english main language
\usepackage[T1]{fontenc}
\usepackage[utf8]{inputenc}
\usepackage{lmodern}
\usepackage{newunicodechar}%able to use e.g. → or ≤ in source
\usepackage{babel}
\frenchbsetup{AutoSpacePunctuation=false, SuppressWarning=true}
\usepackage{setspace}
\usepackage{enumitem}
\usepackage{amsthm}
\usepackage{mathrsfs}
\usepackage{color}
\usepackage{natbib}
\usepackage{doi}
\usepackage{hyperref}
\usepackage{bookmark}% hyperref doc says: Package bookmark replaces hyperref’s bookmark organization by a new algorithm (...) Therefore I recommend using this package.
\usepackage{cleveref}
\bibliographystyle{abbrvnat}
\usepackage{etoolbox}
\apptocmd{\thebibliography}{\hfuzz=20cm\raggedright}{}{}
\usepackage[nolist,smaller,printonlyused]{acronym}
\begin{acronym}
\acro{DM}{Decision Maker}
\end{acronym}

\onehalfspacing
\newtheorem{theorem}{Theorem}
\newtheorem{acknowledgement}[theorem]{Acknowledgement}
\newcommand{\commentYM}[1]{\textcolor{blue}{YM: #1}}
\newcommand{\commentOC}[1]{\textcolor{red}{OC: #1}}
\newcommand{\commentOCf}[1]{\textcolor{red}{\selectlanguage{french}{OC : #1}}}
\newcommand{\commentE}[1]{\textcolor{green}{RelecteurExterne: #1}}
\newunicodechar{ℝ}{\mathbb{R}}
\newunicodechar{≠}{\ensuremath{\neq}}
\newunicodechar{≤}{\ensuremath{\leq}}
\newunicodechar{≥}{\ensuremath{\geq}}
\newunicodechar{→}{\ifmmode\rightarrow\else\textrightarrow\fi}
\newunicodechar{⇒}{\ensuremath{\Rightarrow}}
\newunicodechar{∪}{\cup}
\newunicodechar{∩}{\cap}
\newunicodechar{¬}{\ifmmode\lnot\else\textlnot\fi}
\newunicodechar{…}{\ifmmode\ldots\else\textellipsis\fi}

\newcommand{\adv}{\mathscr{N}}

\begin{document}
\title{Justification and decision sciences}

\author[ld]{Y. Meinard\corref{cor1}}
\author[ld]{O. Cailloux}
\cortext[cor1]{Corresponding author}
\address[ld]{Universit\'e Paris-Dauphine, PSL Research University, CNRS, UMR [7243], LAMSADE, 75016 PARIS, FRANCE}

\begin{abstract}
Decisions are a core subject matter for many economic theories and sub-disciplines, which can be collectively called ``decision sciences''. This article aims at clarifying the normative status of practices that consist in using insights from decision sciences to support decision-making, by formulating recommendations. When applied to a concrete problem, decision sciences obtain conclusions of the form $N ⇒ R$. Such conclusions contain recommendations $R$ conditionned by the norms or normative conceptions $N$. This article explores strategies deployed in the economic and philosophical literature to jump from $N ⇒ R$ to $R$, while eschewing to take a stance on $N$. We argue that these strategies fail. As an alternative, we argue that, as decision scientists, we should openly endorse, as part of our scientific practice, the norm $\adv$, according to which we should embark in a ``quest for justification'': the attitude of people who endlessly keep on arguing about the justification of their stances about what ought to be done. We display reasons to accept $\adv$. We then argue that endorsing $\adv$ in our scientific practice implies deploying an attitude that consist in:
\begin{itemize}
\item[i.]	Systematically justifying our recommendations;
\item[ii.]	Being ready to defend them against criticisms, even when none are formulated;
\item[iii.]	Actively eliciting criticisms;
\item[iv.]	Actually enacting this defense, when we actually face criticisms;
\item[v.]	Understanding our own justifiability in a comparative sense.
\end{itemize}
\end{abstract}

\begin{keyword}
Decision Aiding, Normative Economics, Legitimacy, Justification, Ethics of Operational Research
\end{keyword}

\maketitle

\section{Introduction}
Decisions are a core subject matter for many economic theories and sub-disciplines. In this article, we will use the loose phrase ``decision sciences'' to refer to all these economic approaches, which take decision-making as their main topic, ranging from operational research to social choice theory, through public choice theory, the microeconomic theory of choice, rational choice theory, multi-criteria decision aiding methodology, and so on.

Some of the studies gathered under this umbrella present themselves as purely academic contributions, concerned only with establishing scientific results. However, most decision sciences have practical applications, both in the private sector and in policy-making \citet{tsoukias_policy_2013,marchi_evidence-based_2016}. Such application entail a call for knowledge to become advice, and purportedly scientific propositions to endorse the normative status of prescriptions. 
In this article, we want to clarify the normative status of such practices, which consist in using insights from decision sciences to support decision-making.

By talking about ``normative status'', we mean that we want to clarify the extent to which these practices are indeed normative, and to articulate a normative account of them. 
The term ``normative'', used to qualify a “practice”, will be taken here in a broad sense, to refer to practices that aim at recommending some actions. This understanding is admittedly broad, since it encompasses practices concerned with notions such as ethics, justice, the good and the just, and so on. 
Though broad, this understanding is not all-encompassing. In particular, it excludes purely positive or purely empirical attempts to capture the above mentioned notions. For example, empirical studies aimed at capturing what people in a given group mean when using the term ``justice'', as examplified by \citet{gaertner_empirical_2012}, does not fall within our definition of a ``normative'' inquiry. Besides, we will be concerned here only with recommendations, and will leave aside more binding or constraining normative notions such as obligation.

Based on this understanding of ``normative'', the issue articulated above in terms of ``normative status'' refers to the following difficulty. When applied to a concrete problem, decision sciences obtain conclusions of the form $N ⇒ R$. Such conclusions contain recommendations $R$ conditionned by norms or normative conceptions $N$: accepting  $N$ leads to accepting the recommendations, whereas rejecting $N$ does not lead to any recommendation. However, one would appreciate to obtain $R$, an unconditional recommendation, rather than $N ⇒ R$. This article discusses possible strategies to transition from $N ⇒ R$ to $R$.
\commentYM{j'ai enlevé le terme ``valid''. Voir plus bas également autour de ``minimal''. La notion de ``validity'' est très complexe, je pense qu'il vaudrait mieux éviter de mettre le doigt dans cet engrenage, c'est pourquoi je préfèrerais qu'on enlève les occurrences de ``valid'' et ``validity''}

The immense literature on the normative underpinnings of economic science (see for example \citep{buchanan_positive_1959,sen_nature_1967,dwyer_scientific_1985, heath_value_1994,mongin_value_2006} tackles some aspects of this issue. However, despite these numerous historical debates, the applied literature still expresses diametrically opposed stances on these matters (for an example \cite{spash_bulldozing_2015} vs. \cite{scharks_dont_2016}). This suggests that clarifications are still needed, especially when it comes to the practice of producing recommendations, as opposed to the theoretical aspects on which the above literature is focussed.

This article is an attempt in that direction. It is divided into seven parts. Following the present introduction, section 2 explores the two strategies to transition from $N ⇒ R$ to $R$ which dominate the economic literature. We argue that these strategies are defective. Section 3 explores the strategy deployed in the literature in political philosophy. In section 4 we argue for a completely different strategy, which consists in openly admitting that decisions sciences and their applications endorse a norm $\adv$, according to which we should embark in a ``quest for justification'': the attitude of people who endlessly keep on arguing about the justification of their stances about what ought to be done. Section 5 then explores the implications of our approach, and section 6 delves into more concrete considerations in order to sketch how our approach can overcome what we take to be the most difficult implementation challenge it faces, mamely the fear that it might collapse on the pitfalls of disagreements and clashing ``orders of justification''. Section 7 concludes.

\section{\texorpdfstring{Elusive strategies to leave $N$ outside the scientific part of the inquiry}{Elusive strategies to leave N outside the scientific part of the inquiry}}
A widespread vision among economists is that economics is a value-neutral, purely scientific endeavor. This approach admits that value-judgments are essentially non-scientific, and that economics as a science should therefore eliminate them.

This vision suggests a first strategy, $S1$, to jump from $N ⇒ R$ to $R$. $S1$ consists, for the economist, in sticking to conclusions of the form $N ⇒ R$, and letting \acp{DM}, or even “society” in general, decide of their attitude towards $N$ on their own. Deriving $R$ thus lies, in this strategy, outside the scientific part of the endeavor.

Being completely agnostic about $N$, $S1$ could be applied to any norm, however absurd or blatantly immoral, still leading to conclusions of the form $N ⇒ R$. Such a crude application of $S1$ would certainly produce hoards of results that would not be of interest to anyone, thereby basically amounting to a waste of resources. $S1$ accordingly does not captures what most researchers in economy, decision sciences, or related fields do.

In order to overcome this problem, a natural amendment of $S1$ is to confine the inquiry to norms which are clear and largely accepted enough for \acp{DM}, or society, to (somehow) decide, without the help of science, that they endorse them. Let us call such norms ``self-evident''.
This approach defines a second strategy, $S2$: science obtains conclusions of the form $N ⇒ R$, and $N$ is self-evident \commentOCf{Je trouve qu’il faudrait s’en tenir à self-evidently valid, ou alors expliquer ce qu’on veut dire par minimal, si c’est différent (moi je ne comprends pas, par exemple)}\commentYM{je propose alors ``self-evident'', defini plus haut, pour éviter d'utiliser le terme ``valid''}, therefore, $R$ holds.

This strategy failed so far to produce examples of such norms, however. Candidate for such norms in the literature are axioms that, under their technical, decontextualized, expression, may indeed appear convincing. But such axioms do not constitute self-evident norms, for two reasons. First, in order to know whether they apply to some context, one has to know and/or postulate much information on the context. Second, accepting some norms logically entails the accepting their implications, but it is not evident for an individual to know whether all the implications of some norms are acceptable for her/him.

The issue of inter-personal comparisons of utility (see \cite{baujard_leconomie_2011}) provides an apt illustration of our first reason for not considering usual economists' axioms norms appropiate to unleash $S2$. Inter-personal comparisons of utility involve value-judgments because one cannot claim that a given increase in the welfare of one person outweighs or even is equivalent to a given decrease in the welfare of a second person unless one judges the relative worth of the welfare of the two persons. A prominent conceptual trick to produce recommendations despite the ban on the value-judgments involved in interpersonal comparisons of utility is the strong Pareto principle, stating that state of affairs \emph{y} is better than \emph{x} if no one is worse-off in \emph{y} as compared to \emph{x}, and at least one person is better-off in \emph{y} as compared to \emph{x}. As famously emphasized by \cite{sen_rationality_2004}, among others, the strong Pareto principle is blatantly normative. This very idea is certainly largely accepted, but most authors present this normativity as ``minimal'', in the sense that it seems innocuous to admit that most people, if not everyone, accepts this normative principle. According to this argument, the strong Pareto principle is hence what we called above a ``self-evident'' $N$.

We argue that this claim is more debatable than most authors seem to admit. Take for example a slightly unequal situation \emph{x} where individual \emph{i} is quite well-off whereas individual \emph{j} is poor. Compare with situation \emph{y} where \emph{i} receives a bonanza and \emph{j}'s situation is unchanged. \emph{y} Pareto dominates \emph{x}, but is considerably more inequal. The idea that everyone would claim that \emph{y} is better than \emph{x} is far from self-evident. It relies on questionable assumptions about a total absence of aversion to inequality and envy. Of course, one can retort that our argument applies the Pareto principal to wealth, and that if it is applied to some aggregated welfare index integrating aversion to inequality and envy, then it might no longer be the case that \emph{y} Pareto dominates \emph{x}. But this very rejoinder shows how difficult it might be to set the situation so as to render it evident that everyone will agree on the Pareto principle. One can therefore not claim without further ado that the strong Pareto principle is self-evident.

Our first reason has also been illustrated by a famous example \citep{sen_maximization_1997} where an individual would appear to violate an apparently self-evident norm of rationality, whereas the norm in question is not violated if taking into account a broader context of choice.\commentOCf{Voir si pertinent d’en dire plus / moins.}
\commentYM{si on garde ce second exemple, ça me semble indispensable de le développer, ne serait-ce qu'en 2 phrases. Mais je ne suis pas absolument convaincu qu'on ait besoin de 2 exemples}
We should emphasize, at this stage, that what we just said about the strong Pareto principle is certainly true of all the axioms that the literature usually admits to be normative, but quickly qualifies as \emph{minimally} normative.\commentOCf{Reformuler si on laisse l’autre exemple (où l’axiôme n’est pas PD).}

To illustrate our second reason for rejecting $S2$, \commentOCf{TODO ce n’est pas exactement S2 mais S2 avec des axiômes de ce type.}\commentYM{je ne comprends pas ton commentaire} consider Arrow' s caracterization of the dictator rule \cite{arrow_social_2012} (we thank Ulle Endriss for this example). Arrow's argument is based on axioms that the author presents as liable to be endorsed by most of his readers. The argument shows that the Dictator rule is caracterized (in some formal context) by the axioms of Universal Domain, Pareto Dominance, and Independence of Irrelevant Alternatives. It is easily imaginable that non-expert individuals would willingly accept each of these axioms as capturing value-judgments they endorse about the demands of fairness for a voting rule, if the axioms were explained to them by focusing only on what each axiom demands separately. The point of Arrow's theorem, and the reason why this result is so powerful, is that a series of axioms which are all acceptable yield dictatorship, which our imaginary individuals would certainly reject. A natural way out of the conundrum is to question the spontaneous adherence to the axioms, which prove, on due reflection, to be less commendable. The example of the Dictator rule therefore shows that one cannot rest content with the bare fact that a norm $N$ seems self-evident in the abstract, since one can be led to reject a seemingly self-evident $N$ on due reflexion, once one has come to realize some of $N$'s implications.

$S1$ and $S2$ are two strategies whereby economists claim to use decisions sciences while leaving $N$ outside the purview of their scientific inquiry. We have argued that both strategy fail because they are predicated on a implausible premisse: that \acp{DM} do not need the help of decision sciences to make up their mind about $N$.

\section{\texorpdfstring{Elusive strategies to clear $N$ from its normative content}{Elusive strategies to clear N from its normative content}}
Another strategy designed to jump from $N ⇒ R$ to $R$ can be found in the philosophical literature, in debates between procedural and substantive approaches to the legitimacy of political decisions \cite{meinard_what_2017}. These debates are a cornerstone of the literature on democracy in political philosophy. However,there are actually two debates in the literature which are often articulated using the same terms, in spite of their profound differences. A brief clarification is therefore useful.

A first debate deals with the question whether policy decisions deserve to be called democratic depending on the so-called “output” of the decision, or depending of the process through which they have been taken (“input”) \cite{vatn_environmental_2016, backstrand_environmental_2010}. Proponent of an input theory of democracy claim that, if a decision has been taken through democratic procedures, then it is democratic, whatever its output. Proponents of output theories take the opposite stance. Recast in our simple formalism, this debate is about recommendations $R$ talking about democratic credentials. Input theorists claim that from $N ⇒ R$ we can derive $R$, for some appropriate norms $N$ constraining processes. Output theorists claim that there exists some norms $N$ constraining the possible recommendations which permit to go from $N ⇒ R$ to $R$.% As a third possibility, one could claim, and this is the stance most philosophers adopt, that the right norms are ones that have both constraints on the process and on the resulting recommendations

The second debate opposes purely procedural to substantive theories of democracy. Substantive theories claim that democracy is a matter of values, which can be materialized either in procedures, or in political outcomes, such as for example in rights that are entrenched in law. An example of such an approach is \cite{brettschneider_value_2006}, who claims that democracy is first and foremost a set of ``core values'', which can be materialized in the proceedings of constitutional courts just as well as in votes and institutional proceedings more usually called ``democratic''. As opposed to substantive theories, purely procedural theories claim to account for democracy by delineating formal properties of decision-making procedures that are supposed to be purely value-neutral, and from which democracy would emerge. \cite{habermas_faktizitat_1992} is often presented as the canonical example of such an approach. 

In our formalism, substantive theories claim that $N$, in order to be powerful enough to permit stating $N ⇒ R$ for interesting $R$’s, must contain value-judgments. Whereas purely procedural theories claim that it is possible to obtain $N ⇒ R$, for the kind of recommendations we are interested in (such as universal voting rights), with $N$ not containing value-judgments. 

These two debates have important similarities, mainly because input theories and purely procedural theories similarly focus on procedures.  But neither input nor output theories help obtaining $R$ from $N ⇒ R$. The second debate is more relevant to the subject of this article. Inded, the purely procedural approach can be seen as another strategy, $S3$, allowing to obtain $R$ from $N ⇒ R$, this time by claiming that $N$ does not contain value-judgments after all, and can therefore inoccuously be accepted.

%The purely procedural approach seems to offer a possibility for decision sciences to play such a role without compromising with ``substance'', that is, without having to make value-judgments. By contrast, in the substantive approach, or in any mixed theory involving procedural and substantive approaches, decision science takes it upon itself to address an openly normative task, and in that case we need an account of how it can endorse this task. (…) However, most input theories would be termed ``substantive'' by proponents of purely procedural theories, because they promote procedural \emph{values}. Substantive theories can materialize in both output and input theories. The second debate is a deeply philosophical one, opposing form to substance or values, the first one is more pedestrian, opposing concrete, worldly procedures, to worldly states of affairs which are no less concrete: patterns of endowments, distributions of income, and so on. The second, more philosophical debate is the more relevant one from our point of view here.

$S3$ is the strategy deployed in a couple of historical cornerstones of contemporary political philosophy: \cite{rawls_political_2005} and \cite{habermas_moralbewustsein_1983}. Let us examine whether $S3$ proved more powerful than $S1$ and $S2$.

\cite{rawls_political_2005}'s theory epitomizes what \cite{estlund_democratic_2009} called the ``flight from substance''. He did not want his theory to make any value-judgment about the kind of state of affair that should prevail in a democratic society. He therefore argued that a policy is democratically legitimate if it is based on a constitution whose justification is acceptable by all  ``reasonable'' citizens. But he did not want to make value-judgments about democratic processes either. He therefore further argued that the very definition of reasonableness should be something for reasonable citizens to pick-up. He thereby attempted to eschew making any value-judgments in his account of legitimacy and reasonableness. This was supposed to be a complete flight from substance, in the sense that this account was supposed to eschew any value-judgment, whatsoever. If this reasoning were successful, it would allow to derive $R$ from $N ⇒ R$ by emptying $N$ from its normative content, rendering it inoccuous to endorse.

This approach fails, however, for reasons articulated in different versions most prominently by \cite{habermas_reconciliation_1995} and \cite{estlund_democratic_2009}. Let us start with \cite{estlund_democratic_2009}'s argument because it is simpler to summarize. \cite{estlund_democratic_2009} noticed that, if one admits, following Rawls, that the notion of reasonableness should be selected by reasonable people themselves, there is an ``impervious'' plurality of groups that could select themselves as being ``reasonable''. He concluded that rawlsian political philosophers have no choice but to make bold claims about the content of the concept of reasonableness. \cite{habermas_reconciliation_1995}'s argument is, to a large extent, similar. He criticizes Rawls' presentation of his notions of the ``veil of ignorance'' and the ``overlapping consensus'' as \emph{devices} whose real-life functionning can give rise to principles of justice. According to \cite{habermas_reconciliation_1995}, these notions are rather rhetorical tools thanks to which Rawls exposes principles of justice that he deduces from various philosophical notions, such as the one of a moral person, which Rawls presupposes. Rawls' flight from substance hence abruptly collapses in a retreat back to the substantial inquiry into the nature and features of a moral subject.  Rawls' $N$ is therefore far from being cleared from normative content.

Rawls' attempt is therefore the paragon of the failure of the flight from substance (which is neither a logical flaw nor a failure to unveil illuminating insights, but a failure to strip the argument from its normative anchorage, or ``substance''). Whereas Habermas played a key-role in unveiling the problems crippling Rawls' approach, he himself embarked on a flight of his own, which has both important differences and interesting similarities with Rawls' flight from substance. \cite{habermas_moralbewustsein_1983}'s usage of the notion of ``performative contradiction'' is particularly interesting in this respect. Habermas deploys this argument in his presentation of his ``discourse ethics'', for which he was concerned to provide foundations. In broad outline, this argument states that refusing to accept the purportedly minimal normative discourse ethics would be committing a contradiction. This means that everyone implicitly already admits the tenets of discourse ethics.% By falling back upon a supposedly always already entrenched consensus, the performative contradiction argument would allow Habermas to go a step farther than Rawls, in the direction of the flight from substance. By deploying this argument, Habermas not only eschewed taking a stance on what is a good or a bad policy, or on what is a good or a bad procedure, he went as far as striving to show that there is no stance to be taken about the foundations of morals, because we all always already agree on them.

Unfortunately, the performative contradiction argument fails to evacuate normativity to such a radical extent. Indeed, it is empirically easy to verify by looking at the relevant literature that the tenets of discourse ethics are not accepted by all philosophers, some of whom explicitely denying that they are performing a contradiction by rejecting the proposed tenets \citep{heath_communicative_2001}. Taking seriously the performative argument would lead us to refuse discussion with such philosophers, on the account that they are incoherent. This is however not what we ought to do, and most probably not what Habermas would recommend. To put it otherwise, continuing the discussion requires acceptance of a higher-level norm that says that discussion ought to be pursued if the validity of the performative argument is denied. \commentYM{je pense que ce § ne va pas, mais je ne sais pas exactement comment changer pour l'instant}

It hence turns out that, far from clearing $N$ from its normative content, $S3$ (at least in its two main versions in the literature) mainly render invisible the normative content of theories and arguments. The three startegies $S1-3$ hence leave to the recipient of the recommendation the task of deciding which norms to apply to her/his context, without managing to show that this is an easy task. 
Instead of clinging to this stance, we suggest that, as decision scientists, we have to take upon ourselves to reflexively identify the normative stance that we take when we do our job. We have to inquire into the very normative foundations of our deeds and creeds as decision scientists. 

\section{The quest for justification}
The various strategies explored in the previous sections attempted to evacuate $N$. We argued that all these strategies fail. We claim neither that our inventory of strategies is exhaustive, nor to have demonstrated that it is impossible to carve out a successful strategy to the same effect. But we propose to embark in a completely different strategy, and we argue that it provides a sufficiently satisfactory solution to the problem that this article tackles. We propose a norm $\adv$, and an attitude, that consists in always admitting, whenever a recommendation is proposed, that its validity rests upon the acceptance of the norm $\adv$. $\adv$ refers to the norm that we should embark in a ``quest for justification'': the attitude of people who endlessly keep on arguing about the justification of their stances about what ought to be done.

In this section, we display four reasons to endorse $\adv$. In the next section, we will explain how endorsing $\adv$ solves our problem.

The first reason is that the main alternative to endorsing $\adv$ is to endorse a stance that can be called ``moral realism''. In our definition, this refers to the attitude of people who admit (sometimes explicitely, but most of the time implicitly, and certainly often without being fully aware of it) that they have a special access to a form of moral truth, and therefore can sometimes claim, without further ado, that this or that is ``right'' or ``good''. Notice that the phrase ``moral realism'', understood in various senses, plays important roles in the philosophical literature. We do not claim that our notion encompasses all these senses. Our argument makes sense when one uses our definition of ``moral realism'', and we do not make any broader claim. ``Moral realism'' understood in our sense, in predicated on a philosophy or ``revelation'' or ``epiphany''. We do not claim that this is an implausible moral view, but this a moral view that seems difficult to integrate in a scientific endeavour. \commentOC{I am not convinced. First, we are also moral realists, as we claim we have a special access to a moral claim that $\adv$ is right.}\commentYM{oui bien-sûr! contre-argument traité dans les § sur la stabilité que tu avais enlevés}\commentOC{ Second, the last claim “this a moral view that seems difficult to integrate in a scientific endeavour” is IMHO impossible to justify non-circularly: it relies on a concept of “science“ being based on endless arguing and questioning. So perhaps we’d better say this, more directly, as a simple argument in favor of $\adv$. Now someone could observe that this is an argument in favor of $\adv$ only if one considers science, understood as implying endless arguing and questioning, as good. An adversary could also propose a variant, which would use “science” as usual, with the only difference that it would take some things for granted, similarly to a physicist could work without questioning some basic laws, or a mathematician without questioning set theory (I believe this is done by some in theology for example). So our argument would ultimately rely on people accepting as a self-evident fact that science, understood as implying endless arguing and questioning, is good. I don’t mind, but we have to be honest about it.}\commentYM{je ne comprends pas cette seconde objection. J'ai probablement mal exprimé les choses, mais je ne vois aucune circularité. Pour moi, le premier argument en faveur de $\adv$ est qu'il est implicite dans la notion même de science que, quand on fait de la science, on ne croit pas à la révélation, or la principale alternative à $\adv$ croit en la révélation. Je ne vois pas de cercle}

The second reason is that $\adv$ is stable and ``attracts'' moral realism. A moral realist may claim he also has a solution to our problem of normativity. \commentYM{je ne comprends pas bien la phrase qui précède. A quel problème excatement le réaliste pense-t-il avoir une réponse ?}For him, there are norms $N$ which are such that, whenever they imply some recommendations, suffice to make those recommendations justified in themselves. \commentYM{je ne comprends pas la phrae qui précède: n'y a-t-il pas un pb de grammaire ?} We can thus pass from $N ⇒ R$ to $R$. This rests upon the claim of special access from the moral realist to what is right (captured in $N$). However, as soon as anybody will disagree with $N$, the moral realist is bound to abandon his pretention of validity, resort to violence, or try to justify himself. The last attitude would make his position undistinguishable from ours (moral realism is then attracted by $\adv$). Observe that we use here the resort to violence as a negative consequence of moral realism in our argumentation, which itself is a normative attitude. \commentYM{sur la stabilité, il a quelques trucs de ma version d'avant que j'aurais quand même bien sauvé... Sur l'histoire de la violence, ma version était ``Indeed, a worrying feature of moral realism is that either it converges towards a quest for justifications, or it collapses in an apology of violence. Indeed, imagine that, as a moral realist, you face a challenge to your vision of what ought to be done. You can reply by justifying your stance, and in so doing you give up realism and lean towards the quest for justifications. Or you can ankylose on your stance and strive to impose it by force. In that case, either violence is part of your vision of what ought to be done, in which case you are stuck in the apology of violence, or your vision of what ought to be done rejects violence and your moral realism is repudiated. The only stable moral realism is therefore the apology of violence -- where ``stable'' means here that this attitude can survive without converging towards the attitude with which we contrast is, that is: the quest for justification.'' Il me semble que ça dit la même chose que la tienne, mais en plus accessible je pense car + délayé.
Une autre différence est que tu mets en avant que le rejet de la violence est un $N$, ce qui est évident effectivement, et qu'on peut mettre en exergue dans ma version tout aussi bien.
Je comprends qu'on puisse être ennuyé par le fait qu'on critique le réalisme en s'appuyant sur un $N$. Ceci dit, ce point ne ruine pas l'argument du tout, à mon sens, puisque ce que l'agument dit c'est qu'un réaliste, quel que soit le contenu de son $N$, finit, soit par converger vers $\adv$, soit par se confiner à un $N$ bien particulier et un seul, qui est l'apologie de la violence. Cela ne revient pas à dire : si $N$ alors $N$!
Par ailleurs, dans ta version, je ne vois pas comment faire l'économie du contenu de ce que dit la suite de mon argument :
``What about the quest for justifications? Is it stable? The quest for justifications can be transient: one can be ready to argue up to a certain point, and then fall back upon realism. In such a case, the stability of this quest for justification is determined by the stability of the moral realism on which it falls back. What if it is not transient -- if it does not fall back upon moral realism? One can claim to take advantage of the structure of reasoning deployed in the substantive vs. procedural debate to reject this possibility. The argument would unfold as follows. If you endorse the quest for justification, it means that you endorse the values underlying the idea that moral stances should be backed by justifications. And these very values are the core of your moral realism. The quest for justification would hence unavoidably fall back on moral realism.''
Je ne vois pas comment on peut faire l'économe d'évoquer cette critique (qui est ta première critique à la première raison plus haut). Et la réponse à cette critique pour moi est :
``The quest for justification would hence unavoidably fall back on moral realism. However, the brand of moral realism on which the quest for justification would unavoidably fall back is of a very special kind. By definition, if one sticks to this moral realism by violence, one is no longer embarked in the quest for justification. This moral realism is therefore one that immediately falls back upon the quest for justifications. Hence, though the quest for justifications arguably is underlain by values, those values are immediately redirected towards a quest for justifications. The quest for justifications therefore is stable.''
Les choses me semblent claires : oui, $\adv$ peut être vu comme un réalisme moral, et oui rejeter la violence est un $N$, pour autant le réalisme moral de $\adv$ est bien différents de tous les autres, qui convergent tous vers la violence. Donc (me semble-t-il), il n'y a pas 36 possibilités : soit on s'engage dans $\adv$ soit on s'engae dans l'apologie de la violence, et on ne peut pas faire les deux en même temps. On pourrait rétorquer que, ce que cet argument montre, c'est que la substance de $\adv$, c'est précisément ce $N$ qui dit : ``la violence c'est mal''. Mais ce contrargument ne tient pas, car on peut défendre le $N$ qui dit : ``la violence c'est mal''de deux manières : par la violence, ou par $\adv$. La critique qui dit qu'adhérer à $\adv$ est un réalisme moral comme les autres ne tient donc pas, me semble-t-il. On peut rester dans $\adv$ sans jamais se retrouver dans un réalisme moral qui nous reconduit vers l'apologie de la violence. Je n'arrive pas à comprendre en quoi cela n'est pas un argument en faveur de $\adv$.}

The third reason is that \commentOCf{Aussi : les philosophes qui défendent d’autres normes semblent accepter la nôtre ; il est nécessaire d’accepter la nôtre si on veut la critiquer.}\commentYM{ça, je te laisse rédiger, car je ne vois pas la différence avec mon argument de la convergence vers $\adv$}

The fourth reason is that, we surmise that $\adv$ is truly minimal, in the sense that... although we still need to rely on accepting a norm in order to justify the recommendation, this norm is not as the other norms. We actually apply strategy $S2$ but claim this is truly an example of a norm that may be self-evident. 

We do not claim that $\adv$ is valid in every situation. \commentYM{je ne sais pas comment comprendre la phrase qui précède} In some situations, someone may prefer not to know some arguments, for example, because facing the truth might be too hard. \commentYM{le fait que i préfère quelquechose qui contredit $\adv$ ne me semble pas être une raison de rejeter $\adv$, à moins qu'on définisse une ``bonne'' norme comme étant une norme préférée par tous ou du moins par la personne concernée par son application, ce qui est une définition qui présuppose une norme, que peut être la personne de préfère pas...} Our weaker claim is that $\adv$ is self-evidently valid in many contexts where decision science is usually applied. \commentYM{ma vision des choses est que, dans la plupart des cas quand on applique des sciences de la décision, il est clair et évident que $\adv$, et la plupart des pratiques concernées présupposent, plus ou moins implicitement $\adv$. Je ne sais pas si tu veux dire la même chose} Proponents of $S2$ might observe that this is not really different from their strategy. They could claim that there are contexts where pareto-dominance, for example, can be applied in a self-evident manner (for example, because it is immediately evident in that context which dimensions should be taken into account). (Our rebuttal of $S2$ only showed that this strategy does not always work.) We agree that in some situations it might be that our strategy converges to $S2$ in practice. However, even in such a situation, we have to think about what would happen if an individual would question the application of an axiom. We would have to listen to her arguments. Here again, we see that $\adv$ is really the norm that justifies applying lower-level norms such as pareto-dominance to specific situations, whenever those lower-level norms indeed apply, and this justification being valid only under condition that counter-arguments to their application be accepted. \commentYM{on est d'accord je pense}

A similar remark holds about $S3$. It could be that, after much debate, the performative contradiction argument of Habermas (or some different flavor of it) ends up winning (in the sense that nobody would ever find anything to say against it any more). Even in such a case, $\adv$ would be the norm legitimating the lower-level norm, conditioned on the active search for counter-arguments (otherwise there is no way of knowing whether counter-arguments could be found).

\commentOCf{Les deux § ci-dessus sont mal rédigés, et sans-doute pas à leur place, mais ils servent à susciter le débat entre nous.}\commentYM{on est d'accord je pense, je proposerai des ajustements de rédaction par contre}

The reader will certainly have noticed that our notion of a quest for justification is, in sereval important respect, quite close to some notions developped by Habermas, and in a sense it is even quite close to some elements of his performative contradiction argument criticized above. We indeed full-heartedly acknowledge the influence of Habermas. As far as we know, Habermas however never explicitely spelled out the precise ideas captured by our notion of a quest for justification. And the purely exegetic question whether he would endorse our formulation falls beyond our scope in the present article. 

The quest for justifications is, we claim, what we should explicitely endorse as decision scientists. Not \emph{because} most people would certainly endorse it. But because of the reasons articulated above. We claim that we should not be afraid or shy, as decision scientists, to advocate it. There is no reason to flight this normative stance, no reason to (hopelessly) attempt to flight this substance, no reason to strive to reduce it to putative positive foundations. Accordingly, the strategy in which decision sciences endorse this normative stance has no reason to flight this substance and, quite the contrary, has the means to entrench its own normative stance: it is an embodiment of the quest for justification. 

At this stage, the reader might think that this provisional conclusion is trivial. We indeed hope that it is, in the sense that our aim was to capture a normative stance that all decision scientists will be liable to endorse. A question that remains open at this stage is: what are the implications of endorsing this normative stance? The next section explores this question.

\commentOCf{Envisager des exemples de la littérature pour montrer à quel point les chercheurs ne justifient pas leurs normes, et considèrent cet aspect comme hors de leur discipline, pour réduire le risque que le lecteur trouve ça évident (je suis moi-même maintenant persuadé que ça ne l’est pas, si notre thèse est bien comprise).}\commentYM{a minima on peut citer les débats sur la non-transparence de l'économie du bien-être, avec les refs données par Antoinette dans le bouquin philo éco}\commentOCf{J’ai rescucité un sous-ensemble de mes extraits de littérature, \cref{sec-related}. Il faudrait qu’on en discute pour voir ce dont on parle et à quel niveau de détail et où.}\commentYM{j'ai relu ces éléments, mais comme la première fois que j'ai fait, je les trouve pour l'heure trop elliptiques pour avoir des idées précises sur si et comment on pourrait les utiliser ici}

\section{The recommended model}
\commentOCf{Je te laisse reformuler cette section pour le moment, je pense qu’il faut comme on a dit présenter directement notre proposition, en cinq points (ou autant que nécessaire), plutôt que trois puis cinq, mais je pense que tu veux aussi faire le lien avec ton autre article. À toi de voir comment faire ça au mieux…}\commentYM{j'ai réarticulé une bonne partie de cette section, je ne sais pas si c'est à la hauteur de tes attentes...}

The proposition that we defend in this article is hence that, as decision scientists, we should advocate the quest for justification, which provides the right account of our normative stance and the appropriate model for our possible interventions.

This approach faces a difficult problem, however. This problem reflects a basic and very important ambiguity, which actually reproduces the structure of the procedural/substantive debate presented above. The quest for justification consists in displaying good arguments for the stances we take, judgments we make, etc. But what is a ``good'' argument: is it one that happens to be accepted, or one that should be accepted?

The Rawlsian and Habermassian literature bends towards the second reading, because it is concerned with a counterfactual concept of acceptability: Rawls is not interested in justifications that real people usually accept or will accept, but in justifications that citizens \emph{would} accept, if they were reasonable -- that is, if they had the features that moral persons have in his theory. \cite{habermas_faktizitat_1992} forcefully emphazises this counterfactual aspect.

We have to find a way out of this conundrum without falling back on a variant of moral realism. In other words, we have to find means to distinguish the kind of justifications that we can consider commendable, from those that we cannot, without falling back on a moral realist understanding of what is a ``good'' or ``acceptable'' justification. \cite{meinard_what_2017} ventured a partial solution to the problem in a study of the concept of legitimacy, based on two tenets.

The first tenet is ``incrementalism''. ``Incrementalism'' holds that the idea that one can be able to capture a definitive list of criteria defining what is a good argument, what is the reasonable, what is acceptable, and so on, is illusory. According to ``incrementalism'', one had better work incrementally, to improve step by step justifications, argumentations, etc. This tenet hence consists in accepting that any attempt to specify criteria to capture what makes a justification acceptable is bound to be provisional, and that one can only laboriously improve the blunt concept of acceptability step by step.

The second tenet is ``primacy of practice'': instead of searching for acceptability criteria through theoretical reflection, one should take the stance that consists in putting justifications to the test in real-life situations. According to ``primacy of practice'', justifiability is not a property of a recommendation formulated by a decision analyst, it is a property of the recommendation \emph{together with} the manner these recommendations are  implemented and argued against and in favor.

These two tenets allow to overcome a very problematic feature of philosophical approaches such as Rawls' and Habermas', which is that such approaches seem to admit that philosophical arguments can decide if an argument is a good one, or a justification is an acceptable one, even without a single real stakeholder or decision-maker having the opportunity to make up her mind about it.\cite{meinard_what_2017} used these two tenets to delineate an understanding of the concept of legitimacy. But this approach can easily be translated into an answer to our question in the present article. Here we will trasnlate some important elements of this approach to our context, and also address what we take to be weaknesses of \cite{meinard_what_2017}'s approach, so has to unfold a more satisfactory account.

Our proposed account will be articulated in five points.

A first point is designed to capture the idea that, as application of ``primacy of practice'', our endorsement of $\adv$ should first and foremost take the form of our actually articulating justifications:

As decision scientists, our quest for justification should consist in:
\begin{itemize}
\item[i.]	Systematically justifying our recommendations;
\end{itemize}

A second point shoud prevent our using justifications that happen to be accepted, as a matter of fact, at the moment when we articulate them, but whose weaknesses we sweep under the carpet. This point enbodies ``incrementalism'', by admitting that, once we have found arguments in favour of something, we should try to look for ways through which they could be discarded. Real-life examples of decision-aiding practices that flout this clause are given in \cite{meinard_what_2017}. This leads to the second requirement:

\begin{itemize}
\item[ii.]	Being ready to defend them against criticisms, even when none are formulated;
\end{itemize}

An account limited to these first two point would be impaired by a worrying weakness, which can readily be identified by referring to the literature on epistemic injustice \cite{fricker_epistemic_2007} (this problem is not addressed by \cite{meinard_what_2017}). Some people and group have access to knowledge, others have not. The former are in a position to articulate criticisms, the latter are not. By imposing that decision scientists should be ready to defend their recommendations, the above approach exposes decision science only to part of the spectrum from which criticisms can come. What if there are no criticisms addressed at us, whereas many could have been, but were not, because of epistemic injustices? 
Would we still feel confident in claiming that our approach materializes the idea of a quest for justifications in such a case? We do not think so. There is therefore something amiss in the above account.

Can we fix the problem by identifying a specific group of people that should be the source of criticisms, or even a procedure that should be used to encourage the formulation of such criticisms? Such an approach would not work, because it is hopeless to believe that we can once and for all identify a relevant group of people (or set of groups of people) and/or a magical procedure. We need a more astute approach.

Though incomplete, the above account contains the key to the conundrum. Indeed, this account successfully addresses a problem which is, in a sense, structurally similar to the one we are now addressing. We cannot expect to be able to articulate once and for all what a good justification is. That is why the above account does not spell out a purportedly definitive list of criteria, and rather delineates an attitude on the part of decision analysts. The same trick can do the job here again. We cannot identify once and for all a perfectly relevant group of people and/or a perfect procedure. What we can do is identify an attitude that will be conducive to the quest for justification, and this attitude is a requirement to actively elicit criticisms. This requirement is the missing element in our account to fix its first weakness.

\begin{itemize}
\item[iii.]	Actively eliciting criticisms;
\end{itemize}

We now need a fourth clause to allow clauses ii and iii to embody ``primacy of practice'', by putting justications to the test in real-life, instead of confining them to theoretical criteria (without clause iv, clauses ii and iii would be counterfacual, purely dispositional criteria).

\begin{itemize}
\item[iv.]	Actually enacting this defense, when we actually face criticisms;
\end{itemize}

But this account, although completed to fix the first weakness, still has another worrying weakness, which can be captured by raising the question: when can one admit that one has produced \emph{enough} justifications? (This weakness was alos ignored by  \cite{meinard_what_2017}) Imagine that you have embarked in a discussion with a stakeholder who always has new criticisms to raise. In such a situation, in our logic, should one admit that you cannot stop the discussion at some point or another? This would give to your contradictor a serious advantage, which certainly is undue: if he is ill-intentioned, he can condemn you to indefinitely argue in vain.

The key to solve this new problem can be found in Sen's contribution to political philosophy. \cite{sen_idea_2009} interestingly distinguished two visions of justice: the transcendental vision and the comparative vision. In broad outline, when applied to the notion of justice, a transcendental vision in his jargon is one that claims to answer questions such as ``what is just?'', ``what does justice consist in?'', ``which criterion can one use to decide if a given situation is just or unjust?'' and so on. By contrast, a comparative vision is one that claims that such questions are unanswerable, and that the point of theories of justice is rather to address comparative questions such as ``is situation \emph{x} more just or less just than situation \emph{y}?'' We do not claim in this article to adjudicate the credentials of an application of this account to theories of justice. We simply want to pinpoint that this structure of argument elegantly solves the remaining problem with the above account. If one admits that one cannot give a definitive answer to the question ``what is a good application of decision science in a democracy?'', but that this is not too bad, because the only truly relevant question is a comparative one such as ``is application \emph{A1} better than application \emph{A2}?'', then one can appreciate the relevance of the above account.

The comparatist approach is no panacea. It is, in a sense, a deflationary approach, and we should be prepared to live with the corresponding modesty. 
Besides, it can arouse expectations that it cannot deliver. It is therefore important to clarify what a comparatist reading of our approach cannot do. It does not provide a metric, no generally applicable mechanical means to compare any two applications without discussions. There is bound to be myriads of hard cases where one application of decision science will appear better than another on some respect, but worse on another respect. We do not claim to solve this problem, and doubt that it can be solved at a general level.

In concrete terms, the comparist leads to add a fifth clause to our account:
\begin{itemize}
\item[v.]	Understanding our own justifiability in a comparative sense.
\end{itemize}

To sum up, the model that we recommend is that, as decision scientists, we should advocate a quest for justification that consists in:
\begin{itemize}
\item[i.]	Systematically justifying our recommendations;
\item[ii.]	Being ready to defend them against criticisms, even when none are formulated;
\item[iii.]	Actively eliciting criticisms;
\item[iv.]	Actually enacting this defense, when we actually face criticisms;
\item[v.]	Understanding our own justifiability in a comparative sense.
\end{itemize}

Let us ponder on this formulation. Clearly, applying the formula [i-v] is unlikely to ensure that we will be able to identify \emph{the} ultimate justification for our recommendation. For that, we would need to have access to all the possible arguments, all the possibly relevant information, and we would need a perfect definitive definition of what is a \emph{good} justification. This \emph{first best} is unreachable. A commandable \emph{second best} would be to be able to empirically capture which justifications are acceptable by \acp{DM} and concerned stakeholders once they have taken all the relevant arguments and counterarguments into account. However, to our best knowledge, there is no operational framework to date to capture such ``deliberated judgments'' (see however Cailloux and Meinard), and even once one will be available, it is likely that operationality constraints such as time constraints will make it impossible to deploy such a technology in all applications of decision sciences. The account that we are looking for, and which is for the moment provisionally captured by the formula [i-iii], should hence be seen as an operational third-best--one that captures the essence of $\adv$ in real-life conditions where knowledge is drastically bounded.

\section{Practical pitfalls?}
At this stage, skeptical readers will certainly think that our reasoning is a purely theoretical construct which will in all likelihood face insuperable difficulties if one tries to implement it in practice. We indeed think that practical implementability is a crucial issue. Although we cannot have the ambition in this article to definitely settle the implementability issue in all its dimensions, the present section will delve into more concrete considerations in order to sketch how our approach can overcome what we take to be the most difficult implementation challenge it faces. This challenge is the unavoidable rejoinder that the quest for justification will collapse on the pitfalls of disagreements and clashing orders of justification.

In order to address this important criticism, let us take an example. The choice of this example among myriads of possible examples unavoidably involves some arbitrariness, but we will make a point to ensure that this arbitrariness will not undermine our conclusions. The example chosen is environmental economic valuation. One might think at first sight that this example is very specific. However, this example is less reductive than one might think, and it has interesting features for the purpose of exploring the above criticism. Indeed, in practice, a very vast series of issues are actually gathered under this umbrella, and accordingly focusing on this topic means encompassing applications of economic valuation methods to a very large range of issues \citep{kontoleon_biodiversity_2007}. Besides, in recent years, the environment as a valuation object provided the opportunity for researchers to introduce many methodological innovations, which can and possibly will be applied in the years to come to many other kinds of valuation objects \citep{bartkowski_economic_2017}. Therefore, environmental economic valuations actually encompass a very large array of methods, perhaps more than any other valuation object. This makes it an especially interesting example to discuss how different economic methods can be applied to similar objects and justified, which can provide relevant tests to assess the credentials of our approach.

The methods most prominently used in this domain are based on measurements of people's willingness to pay (WTP) (we will leave aside here the more anecdotal case of methods based on willingness to accept), as it can be elicited by surveys addressed at individual or revealed by these individuals' behavior on markets \citep{meinard_ethical_2016}. The first case encompasses contingent valuation and choice experiment, while the second one mainly encompasses travel cost and hedonic pricing methods. A prominent alternative which has seen numerous empirical applications to biodiversity in recent years is deliberative valuation \citep{bartkowski_economic_2017}. This refers to methods based on choice-experiments or WTP questionnaires embedded in protocols of exchanges of information and discussions. These methods were originally motivated by critical discussions of the ontological assumptions underlying WTP-based valuation methods, and more recently \citet{bartkowski_beyond_2018} explored their positive philosophical underpinnings, referring mainly to \citet{sen_idea_2009}. A third, much less developed method was introduced by \citet{meinard_measuring_2017}. Based on theories of impartialization \citep{kolm_macrojustice:_2004} protocols and on a reading of \citet{rawls_theory_1999}' philosophy, this method attempts to capture the impartial preference of citizens for the funding of biodiversity conservation policies.

As opposed to WTP-based valuation methods, an interesting feature of the literature on deliberative valuation and impartial preference measurement is that this literature explicitly explores the respective normative justifications of the methods. A similar identification of normative justifications for WTP-based methods was attempted by \citet{meinard_ethical_2016}, who proposed a typology of WTP-based valuation methods depending on their respective possible normative justifications. These various contributions thereby allow to draw a typology associating each method with a normative justificatory framework: impartial preference measurement is associated with Rawls' framework, deliberative valuation with Sen's philosophy, stated preference WTP-based methods with ``welfarism'' and revealed preference WTP-based methods with ``endowment conservatism''. This precise definition of these normative frameworks should not concern us here, and we will accept the validity of these associations for the purpose of the argument. The important point from our point of view here is that, thanks to these elements, one can identify a series of argumentative justifications that can be deployed to defend all those kinds of methods in their application to our example.

One might be tempted to conclude that, seen through the lenses of the quest for justification, all three methods would allow decision scientists to implement and justify them, following our recommended model. The latter would accordingly seem to be entirely irrelevant in practice. More precisely, one might surmise that various people will, in all likelihood, disagree on which justification is convincing: some people will accept the justification underlying WTP-based methods, other will harshly criticize them and champion deliberative methods because they will find the justification underlying them more convincing, and so on. Such a scenario, where various groups of people adhere to different and largely irreconciliable frameworks of justifications (\citet{boltanski_justification_2006}'s ``orders of justification''), is indeed considered to be a major phenomenon by some authors in environmental and social economics \citep{chateauraynaud_contrainte_2007}.

Though we take this criticism very seriously, 
\commentOC{Which criticism? I’m lost. The fact that people disagree? Is this a criticism of what we propose? Why?} \commentYM{``this criticism'' réfère à la posiiton décrite dans le § précédent. Je vais essayer d'exprimer + clairement} \commentOCf{Mon désarroi vient du fait que plusieurs critiques possibles semblent mélangées. Dans le § précédent, tu parles d’abord du fait que plusieurs justifications semblent possibles dans certains cas, et qu’il s’ensuivrait que notre proposition est entièrement non pertinente en pratique. Je répondrais que cette \og{}entière non-pertinence\fg{} ne découle pas des prémisses : il en découle seulement que notre proposition ne peut pas nécessairement trancher tout tout le temps (ce que personnellement je reconnais volontiers).}
\commentYM{je me suis mal exprimé, je pense, mais je ne sais pas où exactement. Je n'avais pas en tête cette idée que ``plusieurs justifications semblent possibles dans certains cas''. Où lis-tu ça ?}
\commentOC{ Par ailleurs, j’imagine qu’un autre argument (ou peut-être celui que tu as en tête) serait que si ces méthodes donnent des résultats différents, c’est un peu bizarre qu’elles soient toutes justifiables, si elles le sont auprès d’un individu donné. Je répondrais que c’est confondre le possible et le nécessaire.}
\commentYM{effectivement, c'est un peu ce que j'ai en tête. Je ne comprends pas ton truc de possible et nécessaire.}  
\commentOC{Juste après dans ce même §, tu parles de la difficulté du consensus, ce qui me semble un problème ≠ et dont tu parles plus bas. À propos de ce problème je dirais qu’il n’y a pas de raison a priori de penser que le consensus est toujours possible, et que c’est la tâche des philosophes et des scientifiques d’explorer les contours de ces possibilités, mais que c’est un autre problème, ce qui est aussi en gros ce que tu réponds, si je t’ai bien lu.}
\commentYM{tu as raison, il faut que je clarifie, je ne m'intéresse pas vraiment au consensus. Sur la question du consensus, j'ai l'impression qu'on est d'accord}
\commentOC{Sauf que je ne suis pas d’accord avec le lien moral realism et non-consensus (ou en tous cas tu vas trop vite sur ce point qui me parait difficile).}
we argue that it misses an important aspect of our approach. The fact that argumented justifications can be carved out for the different methods simply means that the methods can be put to the test of their acceptability by various people or groups. This raises the question: how can one know if a justification is acceptable? A natural but mistaken answer would consist in claiming that a justification qualifies as acceptable if and only if it turns out to be accepted in all the situations in which it can be applied. However, here again, how can one perform that kind of test? At best one can say whether applications of a given methods \emph{have so far been accepted}, but this leaves aside all possible but non actual applications, and replaces acceptability by acceptance (falling in the trap that Habermas had earmarked in his criticism of Rawls).

This insuperable problem suggests that, if our approach were applied to methods, it would indeed collapse due to the empirical fact highlighted by the literature on ``orders of justification'': this fact is that various groups typically refer to different and largely irreconciliable orders of justifications, which can (according to some authors at least) be formalized as sets of normative axioms accepted by some groups but rejected by others. Accordingly, the question of whether the justification underlying a given method is acceptable is too general and abstract to be answered. But such a general acceptability of the justification of a method is not what our recommended model is about. Far from being a concrete challenge to our supposedly too abstract account, the ``orders of justification'' criticism only appears challenging because it raises an all too abstract problem. 

Our recommended model, as articulated above, does not refer to methods or methodological frameworks, it talks about what happens in concrete decision aiding processes (understood in the sense spelled out in \citep{tsoukias_concept_2007}). Decision aiding processes are concrete sets of continued interactions between decision analysts, decision-makers and concerned stakeholders. Because our recommended model is about concrete decision aiding processes, the important element in our approach is not the putative general justification underlying the methods used, but rather the justification that can be articulated for the specific usages of the methods put to use at this or that stage during the decision aiding process. Coming back to our example of environmental economic valuations, the various methods mentioned about can be used for very different purposes at various stages in concrete decision aiding processes. Stated and revealed preferences studies are often used to feed cost-benefit analyses \citep{layard_cost-benefit_1994}. However, as emphasized by \citep{meinard_ethical_2016}, the very same monetary valuations can just as well be used as arguments to strengthen public awareness of the importance of the object they value \citep{salles_valuing_2011}, or to put a provisional figure on the impact that various kinds of actions can have on various groups of stakeholders, or in many other ways. In these various cases, the justification that can be developped can take advantage of the general justification underlying the method used, but it can also integrate many other elements pertaining the context, and the specific usage of the method within the particular decision aiding process.

Claiming that our approach is impractical because it is hopeless to find decision aiding methods that will prove acceptable with respect to all the ``orders of justification'' is therefore irrelevant. This irrelevant criticism however suggest another, more powerful rejoinder that deserves to be addressed as well. This possible criticism would claim that, even within a concrete decision process, when a decision scientist sets himself to articulate a justification and fulfill the requirements i-v of our framework, it is highly likely that in most cases he will face at one stage or another someone who will stick to a given ``order of justification'', and whatever the decision scientist's effort to discuss the justification of his recommendation, and the reasons why some of the axioms underlying his interlocutor's ``order of justification'' should be abandonned, still his interlocutor will reject his justification.

We perfectly agree that such difficult situations can happen, and are prepared to admit that they might often happen. However, taking such situations to be fatal pitfalls for our approach would be confusing two largely disconnected issues: on the one hand, the issue that we address in this article, which is the one of the normative status of decision science interventions, and on the other hand, the issue of the possibility to generate consensual decisions. If a decision scientist in a concrete decision process articulates a justification for his recommendation but, whatever his efforts, he always faces the stubborn resistance of some groups, one cannot take this failure of consensus to prove that the decision scientist failed. If he neglected to articulate a justification, then he failed; if he developed a justification but a more justified recommendation was developped by someone else, then he failed; by contrast, if he pursued as far as he could the quest for justification but faced the resistance of someone sticking to moral realism with respect to a given ``order of justification'', then it cannot be said to have failed.

We do not deny that issues such as how consensual group decisions can be generated, or in which conditions will there be this or that pattern of moral realism among sets of groups of decision-makers and stakeholders, are important. Quite the contrary, we think that decision science approaches which would be able to tackle such issues would be considerably more justifiable. What we claim is that such issues go well beyond the requirements encapsulated in our recommended model of the proper place of decision science.

\section{Conclusion}
In this article, we have introduced a normative account of the role of decision sciences in a democracy. For that purpose, we have emphasized the failure of various strategies designed to allow decision scientists to eschew value-judgements, flight ``substance'' and remain normatively neutral. Such attempts take the crude forms of a failed model of economics as pure science and a failed model of decision science as a transparent norm translater, and the subtler form of a model that endorses pure proceduralism to arrange a place for a purportedly value-neutral decision science upstream democracy.

We have strived to demonstrate that all these models fail, which has led us to develop an inquiry into a basic structure of moral reasoning, unveiling an important contrast between ``moral realism'' and the ``quest for justifications''. When then argued that, if one endorses the tenets of ``incrementalism'' and ``primacy of practice'', and if one conceives of the justifiability of applications of decision sciences in a comparative approach, then decision sciences can find their place upstream democracy without having to endorse a moral realist understanding of the concepts of a ``good'' or ``acceptable'' justification. We accordingly ended up recommending the corresponding model for the proper role of decision sciences in a democracy, which can be articulated as follows:

As decision scientists, we should advocate a quest for justification that consists in:
\begin{itemize}
\item[i.]	Systematically justifying our recommendations;
\item[ii.]	Being ready to defend them against criticisms, even when none are formulated;
\item[iii.]	Actively eliciting criticisms, especially from people or groups that we have good reasons to think are not spontaneously liable or willing or able to articulate such criticisms;
\item[iv.]	Actually enacting this defense, when we actually face criticisms;
\item[v.]	Understanding our own justifiability in a comparative sense.
\end{itemize}

Though this models aims to provide a general account of the normative stance that we should take as decision scientists when we work in a democratic setting, and therefore is undoubtedly very ambitious in its scope, we emphasize that it is also very modest in many respects. It does not provide a metric, no generally applicable mechanical means to compare any two applications of decision sciences without discussions. There is bound to be myriads of hard cases where one application of decision science will appear better than another on some respect, but worse on another respect.

It is clearly part of our hope that this account of the normative stance of decision sciences will not come as a surprise for most decision scientists, and is rather liable to gather large support among them. However, though we accordingly expect that most decision scientist will agree with us at this stage, we surmise that most of them do not have clearly articulated ideas about the concrete implications of this basic normative stance. This is why we devoted considerable space to develop some concrete implications of our account, whose actual implementation might prove more disrupting for decision sciences practices than the cheer endorsement of our abstract normative account taken in its most abstract form.
 


%So far, the argument has been entirely abstract. This section will use a series of example to try to flesh it out to some extent.
%Let us start by a burning issue in environmental sciences, the current biodiversity crisis. The current rate of species extinctions is unprecedented. Many conservationist succumb to authoritarian temptations. Democracy looks poorly adapted to take bold decisions to save the planet. It seems like there is a choice to be made. Either we believe in democracy, participation, etc., in which case planet Earth will collapse; or an authoritarian regime will save the Planet and Humanity against its own willing. I think that such a debate is ill-conceived, and that the approach developed here solves the problem, at least to some extent. The idea that urgency trumps the need for participation and democracy in an argument like any other. If a justification based on this argument wins, then the decision to trump participation is not undemocratic, in any non-trivial sense.

%\commentOC{ Bof. Ta dissolution du débat joue
%sur l’ambiguïté du concept de démocratie. Si on prétend que
%la démocratie, c’est prendre les décisions en fonction des
%jugements délibérés du peuple, alors faire la procédure que
%tu proposes, c’est ne pas céder à l’urgence.}

%The same logic shows that this approach can be used to unlock some of the perennial apparent dilemma plaguing the functioning of democratic policies, due to the fact that the above mentioned output/input debate appears undecidable. A typical example in this respect is the case of the 1991 cancellation of the legislative elections in Algeria between the two rounds of the election, following the government’s understanding of the fact that it would certainly be overwhelmingly won by the “Front Islamic du Salut”, championing a largely undemocratic policy agenda. An output theory would call this decision legitimate, an input one would deem it undemocratic. Our approach claims that none of the alternatives (canceling the election or letting it unfold) is intrinsically legitimate, and none is intrinsically more legitimate than the other. The more legitimate one would be the one buttressed more thoroughly by a quest for justification on the part of its champions. This, clearly, is a local answer. Small scale debates here and there probably have unfolded and generated different winners in this respect. There is no general answer to this question. There are illuminating, well-structures local answers.


\section*{Acknowledgements}
We thank Jerome Lang, Philippe Grill and Juliette Rouchier for powerful comments and suggestions on this manuscript.

\section*{References}

\bibliography{decision,philo-eco,beliefs,deliber}

%https://tex.stackexchange.com/questions/26333/elsarticle-appendix-and-a-table-of-contents
\renewcommand*{\appendixname}{}
\appendix

\section{Related perspectives}
\label{sec-related}
Here are perspectives found in the literature about validation, or about the dichotomy prescriptive – descriptive, and the comparison with our approach. (This is currently only a draft. And my quotes are not exact, they are mostly rephrased.)

\subsection{Von Winterfeldt and Fischer}
In Wendt, Vlek (eds.) - Utility, Probability, and Human Decision Making.pdf, p. 58.

\begin{quote}
Axioms distinguish models. They provide an analyst with a systematic testing procedure to choose an evaluation model. BUT: axioms can’t be validated (infinite domain), are not satisfied descriptively (inconsistencies in spontaneous behavior). But measurement theoretic justification of these models is possible. The basic axioms can be tested roughly on easy choices. Second, parameters can be assessed on easy comparisons. (More structured models provide easier comparisons.) If axioms are satisfied in a subset of choice alternatives, and the subset is sufficiently rich to assess the basic model parameters, then “one can have some faith in an extrapolation of the evaluation function constructed”. Idea is to follow the prescriptions as long as one believes that failures of assumptions in actual tests are not systematic, or are due to systematic applications of obviously unsatisfactory simplifying strategies. In summary: axioms permit to discover those systematic violations which are intended and rationally justified by the DM. 

P. 80: MAUT methods are typically validated against “wholistic (or intuitive)” (spontaneous) preferences, what’s called the convergent validity approach. This is ok as long as the nb of dimensions is small, say, not greater than five: then the simplifying heuristic strategies do not apply. Then, the convergent validity approach seems quite reasonable.

Axioms are called “behavioral assumptions”.
\end{quote}

The view proposed by these two authors requires the existence and knowledge of a set of “easy” choices, on which axioms can be tested. They propose no way of determining whether, assuming the tests succeed on those easy choices, the principles extend to the non-easy choices. By contrast, we give no special status to easy choices (or to any subset of the decision situations to which the aid is to be applied), and we propose a systematic, precise validation framework that gives a chance to detect any situation in which the principles on which the aid rests do not validly apply (as judged from $i$).
\commentYM{ce que tu dis me semble tenir la route, mais je pense que tu aurais du mal à convaincre de la supériorité de ton approche les auteurs que tu cites juste avant. Je ne pense pas qu'on puisse à si bon compte sabrer toute la littérature sur les axiomes et leur justitfication. Je te ferai passer un papier que j'ai écris là-dessus, et qui est actuellement en soumission}

About the possibility of intransitive preferences.
\begin{quote}
P. 114. intransitive preferences arise because of information overload or indifference thresholds. If information overload, then intransitivities do not necessarily lead to reject the weak order assumption. DM can be faced with his inconsistencies and asked to resolve them. (Sophisticated strategies involving sequential trade-offs can be used…) OR avoid inconsistencies by sticking to “those subsets of the evaluation space which he can handle easily”. Then extrapolate those judgments by applying the model. BUT indifference thresholds may produce intransitivities which cannot be resolved, since they constitute a genuine limitation in the decision maker’s judgmental process. In such a case (example of non discrimination of “interior noise”) of a genuine and non resolvable failure of the weak order assumption, the original model has to be replaced by a semiorder model. However allowing for intransitive indifferences lead to potential money pumping, hence “most decision analysts … would argue that a semiorder approach should only be a last resort in modeling”.

Authors take example where “it would be easy to convince the dm that he should care even about the smallest differences, thus making him conform with the weak order assumption”.
\end{quote}

Our approach offers a description of how such attempt of convincing the dm [that intransitivies should not occur] might take place, be faced with contradiction (to avoid brainwashing), and defines how one can determine whether it is successful. It is an empirical question, in our approach, whether methods that allow for intransitivities will be more successful than others, rather than a question of whether the practioner feels about it (ok, never, last resort only…)

\subsection{Raiffa}
\begin{quote}
Back from Prospect theory to UT, in Plural Rationality and Interactive Decision Processes (1984). Cites Allais and Ellsberg. Some people fully understand SEU but refuse to adopt these principles in their own important decision making. Then: 1) Raiffa agrees that important psychological concerns should be addressed in the theory; 2) sometimes the theory is right however and therapy is the appropriate remedy. He confesses that therapy has not always proven to be ingenious enough, however.
\end{quote}

Again, we provide a way to distinguish therapy from brain-washing, and a way of selecting those theories that are more successful than others in providing “therapy” or switching to weaker (or different) principles when appropriate.
\commentYM{là, c'est vraiment trop elliptique}

\subsection{Machina}
Machina - Choice Under Uncertainty: Problems Solved and Unsolved (2000).

\begin{quote}
	People have the right to be non SEU maximisers. The theory must take this into account.
\end{quote}

Maxima does not indicate how to systematically distinguish those situations where you are not an SEU maximiser simply because you haven’t considered the best arguments in favor of it. Describing usual behavior does not suffice to give good prescription. Our approach permits to distinguish those cases.

\subsection{Allais VS Morgenstern and Amihud}
Allais, M., \& Hagen, O. (Eds.). (1979). Expected utility hypotheses and the Allais paradox. Dordrecht: D. Reidel.

\begin{quote}
Allais (Foreword): “It would not be rational to admit a system of axioms and not accept its implications (principle of consistency), but it is still an open question whether rationality should be defined on the basis of criteria relating only to random choice, or following criteria which are independent of all consideration of random choice.” 

Morgenstern: utility theory is to be applied to normal circumstances, where the individual knows the situation we talk about. In game theory, we start with assumption that higher payoff is better, then derive the rational behavior. But we have to not deal in ranges that we know humans can’t cope with (e.g. comparing very small probabilities).

Morgenstern: Incidentally this is a further comment on the naivete of the so-called 'revealed preference' theory (Morgenstern, 1972). If one goes outside the range of normal experience of the individuals questioned, it becomes also clear that statements about their alleged consideration of variance, skewness, first, higher moments, etc., are subject to the same doubts as their gross 'decision' noted above. This matter is treated competently by Y. Amihud in this Volume (p.149) and I fully agree with his observations.

Morgenstern: This also takes care of the matter whether those questioned would 'correct' their behavior if it were pointed out to them that they ‘act’ in violation of the expected utility hypothesis. That theory, as formulated by the von Neumann-Morgenstern axioms, is normative in the sense that the theory is 'absolutely convincing' which implies that men will act accordingly. If they deviate from the theory, an explanation of the theory and of their deviation will cause them to readjust their behavior. This is similar to the man who tries to build a perpetuum mobile and then is shown that this will never be possible. Hence, on understanding the underlying physical theory, he will give up the vain effort. Or, an individual making a mistake in, say, long division, will clearly correct himself when shown the mistake. Naturally, it is assumed that the individuals are accessible intellectually whether it be physics or arithmetics or utility that is being explained to them. In that sense there is a limitation since there are certainly persons for whom this is impossible. Whether they then should be called 'irrational' is a matter of taste.

Amihud, A reply to Allais: displays a desire for correct predictions of the theory (towards risky choices), admits that UT predicts poorly but criticises Allais whose theory is not shown to predict better.
\end{quote}

About the application of the theory to “normal” circumstances, Morgenstern’s defense could be considered somewhat similar to the view of validation by Von Winterfeldt and Fischer (above), according to which you ought to test the principles on easy choices and just assume by faith that they extend beyond those easy choices. Morgenstern seems to consider that 1) indeed utility theory will pass the tests on easy choices (called “normal circumstances”, which might extend somewhat beyond Von Winterfeldt and Fischer’s easy choices, but is also left undefined), but 2) that UT should not be applied at all beyond those. (Other interpretations are probably possible however.)

About proper validation, let us consider a situation where UT, properly applied, suggests something that the DM is not a priori convinced is a good idea. Then, Morgenstern suggests the practicioner should explain that “they ‘act’ in violation of the expected utility hypothesis”. Assume the DM is still unconvinced. Now I see two possible interpretations of the position of Morgenstern. 1) the DM ought to be called “irrational”, or any other word that means that the DM is intellectually in the same position than someone who does not understand where their fault lies in a wrong computation (or, perhaps, the practitioner should be considered not a good educationist). Or 2) this would be a falsification of UT, that Morgenstern predicts will never happen. This second interpretation might seem too charitable, but it looks plausible to me, using two points: a) Amihud cites Morgenstern when saying that he has been “taught that ‘the aim of a good theory is prediction, and in prediction lies the ultimate test of its validity’ (Morgenstern (1972), p.704).” and b) Morgenstern declares he agrees with Amihud in his text above.

This article defines precisely tests similar to those Amihud might have had in mind, and proposes to select those theories that pass those tests.

This article proposes to see (a part of) the disagreement between Allais and the utilitarists as a controversy that is to be solved empirically.

\subsection{Slovic \& Tversky}
Slovic, Tversky - Who Accepts Savage's Axiom? (1974)

\begin{quote}
The authors compares the convincing power of two arguments, one in favor of Allais, the other one in favor of Savage. The article exhibits in its conclusion a hypothetical dialog between a utilitarist and a non-utilitarist. The argument used by the utilitarist is that a UT approach will be able to convince the DM, provided sufficient explanations are given to it, so it is prescriptively correct. The opposed argument is that this might be more brain-washing than explanation. The article does not propose a way to settle this debate.
\end{quote}

In this article we tried to do precisely this: give a way of settling this kind of debate. By giving all parties the right to counter-argue, we think we can escape the allegation of brain-washing while giving subtle theories the chance to argue in favor of their principles.

\subsection{The problem of lability}
Knowing what you want: Measuring labile values In Wallsten T.,(Ed.), Cognitive processes in choice and decision behavior (pp. 117–141), Fischhoff, Slovic, \& Lichtenstein, 1980
\begin{quote}
“subtle aspects of how problems are posed, questions are phrased and responses are elicited can have substantial impact on judgments that supposedly express people’s true values. Furthermore, such lability in expressed preferences in unavoidable: questions must be posed in some manner and that manner may have a large effect on the responses elicited.”

“The moral of these results is disturbing: Invariance is normatively essential, intuitively compelling, and psychologically unfeasible. Indeed, we conceive only two ways of guaranteeing invariance. The first is to adopt a procedure that will transform equivalent versions of any problem into the same canonical representation. This is the rationale for the standard admonition to students of business, that they should consider each decision problem in terms of total assets rather than in terms of gains or losses (Schlaifer, 1959). Such a representation would avoid the violations of invariance illustrated in the previous problems, but the advice is easier to give than to follow. Except in the context of possible ruin, it is more natural to consider financial outcomes as gains and losses rather than as states of wealth. Furthermore, a canonical representation of risky prospects requires a compounding of all outcomes of concurrent decisions (e.g., Problem 4) that exceeds the capabilities of intuitive computation even in simple problems. Achieving a canonical representation is even more difficult in other contexts such as safety, health, or quality of life. Should we advise people to evaluate the consequence of a public health policy (e.g., Problems 1 and 2) in terms of overall mortality, mortality due to diseases, or the number of deaths associated with the particular disease under study?

Another approach that could guarantee invariance is the evaluation of options in terms of their actuarial rather than their psychological consequences. The actuarial criterion has some appeal in the context of human lives, but it is clearly inadequate for financial choices, as has been generally recognized at least since Bernoulli, and it is entirely inapplicable to outcomes that lack an objective metric. We conclude that frame invariance cannot be expected to hold and that a sense of confidence in a particular choice does not ensure that the same choice would be made in another frame. It is therefore good practice to test the robustness of preferences by deliberate attempts to frame a decision problem in more than one way.”
\end{quote}
We could study how our approach might be used to overcome problems highlighted here above. Using different frames as counter-arguments, one can study, using our framework, which are the suggestions which better resist re-framing. (To be developed…)

\end{document}
