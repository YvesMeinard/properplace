\RequirePackage[l2tabu, orthodox]{nag}%less problems with LaTeX code
\RequirePackage{silence}\WarningFilter{newunicodechar}{Redefining Unicode character}
\pdfgentounicode=1 %permits (with package glyphtounicode) to copy eg x ⪰ y iff v(x) ≥ v(y) from pdf to unicode data. 
\input{glyphtounicode}%nice copy from PDF
\documentclass[preprint, french, english, 11pt]{elsarticle}%english main language
\usepackage[T1]{fontenc}
\usepackage[utf8]{inputenc}
\usepackage{newunicodechar}%able to use e.g. → or ≤ in source
\usepackage{babel}
\frenchbsetup{AutoSpacePunctuation=false, SuppressWarning=true}
\usepackage{setspace}
\usepackage{enumitem}
\usepackage{amsthm}
\usepackage{mathrsfs}
\usepackage{color}
\usepackage{hyperref}

\onehalfspacing
\newtheorem{theorem}{Theorem}
\newtheorem{acknowledgement}[theorem]{Acknowledgement}
\newcommand{\commentYM}[1]{\textcolor{blue}{YM: #1}}
\newcommand{\commentOC}[1]{\textcolor{red}{OC: #1}}
\newcommand{\commentOCf}[1]{\textcolor{red}{\selectlanguage{french}{OC : #1}}}
\newcommand{\commentE}[1]{\textcolor{green}{RelecteurExterne: #1}}
\newunicodechar{ℝ}{\mathbb{R}}
\newunicodechar{≠}{\ensuremath{\neq}}
\newunicodechar{≤}{\ensuremath{\leq}}
\newunicodechar{≥}{\ensuremath{\geq}}
\newunicodechar{→}{\ifmmode\rightarrow\else\textrightarrow\fi}
\newunicodechar{⇒}{\ensuremath{\Rightarrow}}
\newunicodechar{∪}{\cup}
\newunicodechar{∩}{\cap}
\newunicodechar{¬}{\ifmmode\lnot\else\textlnot\fi}
\newunicodechar{…}{\ifmmode\ldots\else\textellipsis\fi}

\begin{document}
\title{A deliberative role for decision sciences}
\author[ld]{Y. Meinard\corref{cor1}}
\author[ld]{O. Cailloux}
\cortext[cor1]{Corresponding author}
\address[ld]{Universit\'e Paris-Dauphine, PSL Research University, CNRS, UMR [7243], LAMSADE, 75016 PARIS, FRANCE}

\begin{abstract}
Decisions are a core subject matter for many economic theories and sub-disciplines. The idea that economic, and more generally scientific insights, should guide policy making has ancient roots, dating back at least to Hobbes’ seminal exposition of a science of policy, and in a sense even to Plato's figure of a ``philosopher king''. However, there is nowadays a growing tendency among policy makers to call for scientific knowledge to buttress policy-making and implementation. In this article, we elaborate a framework clarifying the normative status of practices that consist in using insights from decision sciences to support political decision-making in this way in Western, contemporary democracies. We begin by exploring accounts of the role of decision sciences that more or less underlie a large part of the economic literature. In their various guises, these accounts attempt to draw an impermeable divide between factual and normative claims, and to justify the role of decision science as confined to the factual realm. We recall the largely accepted reasons why such models should be considered defective. We then explore an alternative account, which locates the purported role of decision sciences upstream democracy. This account provides us the opportunity to investigate some fondamental issues in the foundations of normativity in a new perspective, which in the end allows us to flesh out a specific version of the third account and recommand it.
\end{abstract}

\begin{keyword}
Democracy, Decision Aiding, Ethics of Operational Research, Normative Economic
\end{keyword}

\maketitle

\section{Introduction}

\commentYM{elements of state-of-the-art to integrate: there are numerous papers within economics itself
which discuss the role and status of value judgments (Arrow 1951, Little 1952,
Buchanan 1959, Klappholz 1964, Blaug 1980[1995], Heath 1994, Some have
even considered economics not only to be a moral science, but a mere expression
of ideology (Myrdal 1954). And some have asserted that value judgments
are merely facts like any other (Ng 1972), or that, because the value judgments
typically made in welfare economics are uncontroversial, they may thus be considered
on a par with any positive assertion (Archibald 1959, Hennipman 1976).
Some have conducted contextual analyses of the importance of value judgment
in the contributions of various authors (e.g. Balisciano and Medema 1999, Davis
2005, Cherrier 2009, Andrews 2010), or textual analyses of the influence of value
judgments in allegedly neutral textbooks (e.g. Hill and Myatt 2010). Conversely,
and regrettably, fewer papers conduct minute methodological analyses aimed at
tracing the role of value judgments in economics (Mongin 2006); and, to my
knowledge, no such paper has been concerned specifically with the case of expertise.
We also have to find a way to talk about ``thin'' and ``thick'' predicates sensu Williams 1985, don't know where
Sen (1967: 50) defines a value judgment
as “basic” to a person if no conceivable revision of factual assumptions can
make him revise the judgment. If such revisions can take place, the judgment is
“non-basic” in his value system.
there are six phases to the decision
process, which we may present in chronological order as follows: First comes
(1) the scientific phase. The middle stage, expertise, contains four phases: (2) request,
(3) selection of the model, (4) application to the data, and (5) formulation of
a report, with descriptions, evaluations and prescriptions. The decision phase (6),
which comes last, may include a political process in the case of a public decision.
Mongin 2006: strong neutrality thesis = our section 2, weak neutrality = our third section}

\commentE{Ph. Grill: peut-être commencer directement par un ex, pour intéresser le lecteur économiste?}
\commentYM{je suis circonspect. Si on commence par un exemple reel, il y aura toujours des lecteurs mal intentionnes qui critiqueront le detail de l exemple au lieu du contenu du texte. Si on prend un exemple fictif, ca risque d'etre pris pour une figure de rhetorique et ne pas avoir beaucoup de poids. Par ailleurs, difficile de trouver un exemple qui capture toutes les applications possible du raisonnement, meme dans les grandes lignes}
\commentOC{D’accord avec l‘élevé YM (qui, sans accents, serait devenu l’eleve YM, quand même moins classe).}

\noindent Decisions are a core subject matter for many economic theories and sub-disciplines. In this article, we will use the loose phrase ``decision sciences'' to refer to all these economic approaches, which take decision-making as their main topic, ranging from operational research to social choice theory, through public choice theory, the microeconomic theory of choice, rational choice theory, multi-criteria decision aiding methodology, and so on.

Some of the studies gathered under this umbrella present themselves as purely academic contributions, concerned only with establishing scientific results. However, as pointed out by \cite{tsoukias_policy_2013} and \cite{marchi_evidence-based_2016} 
\commentOC{Beurk c’est pas beau ces citations.}, among others, there is a growing tendency among policy makers to call for scientific knowledge to buttress policy-making, often in a so-called ``evidence-based'' approach. This is a call for knowledge to become advice, and for purportedly scientific propositions to endorse the normative status of prescriptions. Though this tendency might have indeed gained prominence, or rather visibility, in recent years, the idea that economic, and more generally scientific insights, should guide policy making has ancient roots, dating back at least to Hobbes’ seminal exposition of a science of policy \citep{skinner_reason_1996}, and in a sense even to Plato's figure of a ``philosopher king''. Issues such as whether the wisest should rule the masses, or how scientific insights can contribute to collective decision-making in different political regimes, are accordingly addressed by a vast literature. In this article, we claim neither to exhaustively review this immense literature, nor to do justice to all the aspects of the issues it tackles. What we want to do is to elaborate a framework clarifying the normative status of practices that consist in using insights from decision sciences to support political decision-making in western, modern democracies. \commentOC{Je propose de remplacer cette dernière phrase par : We rather want to clarify the normative status of practices that use insights from decision sciences to support decision-making. (Je trouvais ton titre prétentieux mais je me rends compte que je le suis encore plus.)}

By talking about ``normative status'', we mean that we want to clarify the extent to which these practices are indeed normative, and to articulate a normative account of them. 
\commentOC{Je trouve qu’il faut clarifier ce qu’est une pratique normative. Pour moi (sans-doute trop naïvement), une pratique est normative si elle s’appuie sur des normes considérées comme valides et à ne pas remettre en question, ou, une pratique est normative si ne se demande pas activement si on pourrait les remettre en question, et elle omet de conditionner ses conclusion à l’acceptation de ces normes (puisqu’elles sont considérées comme valides de toute façon). Ainsi, la pratique normative de MAVT omettra l’antécédent dans la conclusion suivante : “si on considère que les préférences sont transitives, et, etc., alors il convient de…”. Ainsi, le (mauvais) praticien de MAVT considère comme une norme la transitivité des préférences, la complétude, et s’appuie sur elles pour justifier ses recommandations sans questionner explicitement le décideur sur ces aspects, et d’ailleurs sans se demander comment on pourrait faire pour savoir si ces normes sont valides. Il faudrait voir si on peut trouver une meilleure définition dans la littérature ; car je sors celle-ci de mon chapeau à l’instant et elle n’est sans-doute pas très satisfaisante. Mais si on poursuit sur cette définition, on pourrait comparer l’attitude positive du physicien : \og{}\emph{Si} vous voulez maximiser la vitesse, alors il faut concevoir l’engin de telle façon.\fg{} et l’attitude normative : \og{}Il faut concevoir l’engin de telle façon.\fg{} (Sous-entendu : c’est ce qu’il faut faire pour maximiser la vitesse, et c’est sans aucun doute la seule chose qu’on peut vouloir faire dans l’application dont je m’occupe.)} The term ``normative'' will be taken here in a broad sense, to refer to the large set of issues concerned with elucidating the content of concepts such as ethics, justice, the good and the just, and so on. We emphasize at the outset that, though this understanding is broad, it is not all-encompassing. In particular, in our understanding of ``normative'' in this article, we exclude purely positive or purely empirical attempts to capture the above mentionned notions. For example, empirical studies aimed at capturing what people in a given group mean when using the term ``justice'' does not fall within our definition of a ``normative'' inquiry.
\commentOC{Ça me semble un peu trop flou à ce stade si on veut appliquer cette définition à l’usage de \og{}normatif\fg quand il qualifie \og{}pratique\fg. Veux-tu dire qu’une pratique est normative si elle se réfère à un concept tel que l’éthique, le juste, ou le bien, and so on, à l’exclusion des pratiques purement positives ou empiriques ? Dans ce cas, il s’ensuit immédiatement que toute pratique d’aide à la décision est normative. Je pense que tu veux plutôt obtenir cette conclusion au terme d’un raisonnement, et expliquer comment on peut s’abstraire des problèmes qu’elle pose.}

Based on this understanding of ``normative'', the question articulated above in terms of ``normative status'' can accordingly be reformulated as: why and to what extent can one consider that using insights from decision sciences to aid political decision-making is something that ought to be done?
\commentOC{Si tu aimes bien mon approche par une définition plus précise du normatif, on pourrait ici poursuivre en expliquant qu’on va s’intéresser à la question suivante. Si on est normatif, on omet des hypothèses dans nos conclusions et on obtient des conclusions de la forme $R$, où $R$ désigne une recommandation, et ce n’est justifié que si les normes sur lesquelles on s’appuie sont valides, et comment va-t-on faire pour le savoir ? Par ailleurs, si on n’est pas normatif, on pense bien à mentionner les hypothèses explicitement dans les conclusions et on obtient des conclusions de la forme $N ⇒ R$, mais alors nos recommandations risquent de ne jamais être utiles, puisqu’elles sont subordonnées à l’acceptation des normes, dont il faudrait prouver la validité. Et on revient à la même question : comment justifier la validité des normes pour la prise de décision ?}
 A similarly permissive approach will characterize our usage of the terms ``values'' and ``value-judgments''.

We will tackle the above question here only in the context of what we will call ``western, modern democracies''. This phrase will also be understood in a very broad sense. We do not want to confine this inquiry to a specific theory or vision of democracy. An immense literature in a large series of disciplines has produced a vest array of competing definitions of democracy. As our rationale will unfold in this article, we will be led from time to time to refer explicitly to several of these understandings of the term ``democracy''. But the question that we tackle is not anchored in any of these theories. This question owes part of its significance and importance to the fact that it uses the idea of “western, modern democracies” in the largely undetermined sense in which one uses it in ordinary conversation, broadly to refer to the kinds of institutions and political processes that historically emerged and matured in Europe and North-America, and some of the values and principles that are more or less naturally associated with them.
\commentOC{Conceptuellement, je ne vois pas à ce stade de ma réflexion pourquoi le propos est restreint aux démocraties. Stratégiquement aussi, c’est à mon avis une mauvaise idée. Je trouve le risque de confusion important, avec un terme aussi chargé. (Cf. notre discussion orale.)}

Because the literature on the topics concerned by this article is so vast, one can doubt that anything really new can be written on it. Decision sciences are, however, in perpetual transformation, and so are the political practices in which they are called to play a role. This is why we think that it is not unreasonable to hope that thinking anew these issues can shed new light on them.
\commentOC{À réfléchir. Au stade actuel de ma compréhension, je ne vois pas que notre proposition repose sur un changement récent dans les DS ou les pratiques politiques.}

This article is divided into nine parts. \commentOC{Mon co-auteur n’a peur de rien !}
Following the present introduction, sections 2 and 3 explore accounts of the role of decision sciences that more or less underlie a large part of the economic literature. We argue that these accounts are defective. Though they use different strategy, they both attempt to draw an impermeable divide between factual and normative claims, and to justify the role of decision sciences as confined to the factual realm. We argue that these accounts fail, for various reasons that have already been articulated in various strands of the literature. These relatively short parts should be seen as a lapidary review of existing knowledge, rather than as an original contribution. Section 4 explore an alternative account, which locates the purported role of decision sciences upstream democracy. In our view, this account is implicit in many contemporary approaches. However, to our best knowledge, it has so far never been spelled out explicitly. In sections 6 to 7, we explore some foundamental questions of normative theory so as to argue that a specific version of this third model deserves to be recommended. Section 8 delves into more concrete considerations in order to sketch how our approach can overcome what we take to be the most difficult implementation challenge it faces, mamely the fear that it might collapse on the pitfalls of disagreements and clashing ``orders of justification''. Section 9 concludes.

\section{A first defective model: Decision science as pure science}
\commentOC{Dans ma vision, voici l’essence du propos de cette section. (Je pense que je ne fais que reformuler ton propos.) La stratégie $S1$ pour justifier les recommandations est de dire : certes, nos conclusions sont de la forme $N ⇒ R$, mais on obtient $R$ facilement car $N$ est acceptée par tous. À cela, on répond (ou on devrait répondre) deux choses. Premièrement, $N$ est acceptable ou pas en fonction non de sa forme symbolique, mais du contexte et des éléments sur lesquels l’axiôme (si N est exprimé comme un axiôme) s’appuie. Par exemple, PD n’est pas acceptable si on a oublié des éléments importants dans le calcul de l’utilité des individus composant la société. Donc en somme, la stratégie $S1$ ne résoud pas le problème mais laisse encore une fois au décideur final le soin de décider, sans aide, si $N$ est acceptable dans son cas. Or, c’est souvent un aspect difficile, sur lequel on a justement besoin d’aide. Ou pire, $S1$ commet l’erreur de la normativité en faisant semblant que $N$ est toujours acceptable en omettant de vérifier le contexte. Un deuxième problème de $S1$ est que $R$ est peu riche, car doit s’appuyer sur des normes $N$ qui doivent paraitre immédiatement évidentes aux yeux de tous (tel que PD), et ceci est une conséquence du fait que le décideur n’est pas aidé dans $S1$ pour déterminer la validité de $N$, et ne peut donc que s’appuyer sur son bon sens. Or, on pourrait dans certains cas peut-être obtenir des $R$ plus fortes en s’appuyant sur des $N$ plus ambitieux. Mais il faut pour cela justifier $N$. On revient au problème de base.}
\noindent The idea that economics is a value-neutral, purely scientific endeavor, is traditionally associated with \cite{robbins_essay_2007}'s classical contribution. This approach admits that value-judgments are essentially non-scientific, and that economics as a science should therefore eliminate them. One might think that the proponents of this model would naturally claim that economic science simply cannot produce policy advices without denaturing itself. Numerous authors have however strived to demonstrate that, even in this approach, economic science manages to produce so-called “policy relevant” results, liable to feed policy advices.
\commentOC{Ajouter des \textbackslash noindent systématiquement n’est pas très élégant. Ça rend le déplacement de § plus difficile, et c’est du code redondant qui vise à faire une seule chose (ne pas indenter les premiers §). Mieux vaut dire à \LaTeX{} une fois pour toute ce qu’on veut. (Ça se change aussi plus facilement si on ne le dit qu’à un endroit.) Par ailleurs, c’est une décision typographique qui appartient à mon avis à l’éditeur, et qui est probablement programmée dans la classe elsarticle. Mais on peut le changer quand-même si tu insistes : c’est toi le premier auteur.}

As \cite{baujard_leconomie_2011} recalls, the issue of inter-personal comparisons of utility provided a prominent historical exercise for economists engaged in this endeavor. Inter-personal comparisons of utility involve value-judgments because one cannot claim that a given increase in the welfare of one person outweighs or even is equivalent to a given decrease in the welfare of a second person unless one judges the relative worth of the welfare of the two persons. A prominent conceptual trick to produce policy-relevant results despite the ban on the value-judgments involved in interpersonal comparisons of utility is the strong Pareto principle, stating that state of affairs \emph{y} is better than state of affairs \emph{x} if no one is worse off in \emph{y} as compared to \emph{x}, and at least one person is better off in \emph{y} as compared to \emph{x}. But this principle has two important defects. First, it leaves economists incapable to formulate any policy-relevant result in situations where there is no state of affairs that strongly Pareto dominates the others, which is bound to happen quite often. 

\commentOC{ On s'egare : c'est sans-doute vrai,
mais sur un plan tres différent de ce dont je pensais qu'on
discutait ici, à savoir la validité en principe du raisonnement
``sans valeur'', plutot que son efficacite pratique.}
\commentYM{je suis d'accord, on est sur un plan different du reste de la discussion. Pour autant, il ne me semble pas que cette petite phrase coute grand chose, et si l'on veut que le texte ait une portee sur la question de ce que les sciences de la decision peuvent apporter dans les politiques REELLES, ce point me semble important}

Second, as famously emphasized by \cite{sen_rationality_2004}, among others, the strong Pareto principle is itself normative. This very idea is certainly largely accepted, but many authors often present this normativity as ``minimal'', in the sense that it seems innocuous to admit that most people, if not everyone, accepts this normative principle. We argue that this claim is more debatable than most authors seem to admit. Take for example a slightly unequal situation \emph{x} where individual \emph{i} is quite well-off whereas individual \emph{j} is poor. Compare with situation \emph{y} where \emph{i} receives a bonanza and \emph{j}'s situation is unchanged. \emph{y} Pareto dominates \emph{x}, but is considerably more inequal. The idea that everyone would claim that \emph{y} is better than \emph{x} is far from self-evident. It relies on questionable assumptions about a total absence of aversion to inequality and envy. Of course, one can retort that our argument applies the Pareto principal to wealth, and that if it is applied to some aggregated welfare index integrating aversion to inequality and envy, then it might no longer be the case that \emph{y} Pareto dominates \emph{x}. But this very rejoinder shows how difficult it might be to set the situation so as to render it evident that evryone will agree on the Pareto principal. We therefore claim that one has to face the fact that, if one wants to claim that the Pareto principle is minimal, one cannot simply claim that it is self-evident, one has to prove it.

We should emphasize, at this stage, that what we just said about the strong Pareto principle is certainly true of all the axioms that the literature usually admits to be normative, but quickly qualifies as \emph{minimally} normative.

Though one should certainly celebrate that there is now a large consensus that the model of economics as a pure science is a mirage, not only because sciences are all anchored in values, be they only epistemic values \cite{longino_science_1990}, but also because many economic studies explicitly tackle normative issue, which implies that role of values is all the more prominent in this discipline, the view that decision sciences are anchored in \emph{minimally} normative still plagues decision sciences. We argued that such claims cannot be heralded without justification, which decision sciences seeing themselves as pure sciences do not even attempt to do. 

\commentOC{je trouve la section 2 peu utile. À la réflexion, elle est utile, mais je n’ai pas vu toute suite la force du lien avec le propos général. Il faudrait réfléchir à une éventuelle reformulation, que ce soit dans le sens de ce que je viens de proposer ci-dessus ou autre.}
\commentYM{je suis d'accord, elle est peu utile. Cependant, je pense qu'un article qui ne contient que des trucs utiles est indigeste, voire complètement illisible. Il faut parfois dire des choses simples avec lesquels tout lecteur se sentira a l'aise, d'une part pour qu'on ne puisse pas nous dire qu'on aurait du y penser, d'autre part pour etre un tant soit peu exhaustif, enfin pour rendre les choses digestes. Je dis ca, mais je ne ne serais pas traumatise si on faisait sauter cette section}

\section{A second defective model: Decision science as a norm translater}
\noindent The failure of the pure science model suggests a way out, which consists in integrating value-judgments as axioms with respect to which economists should take an agnostic stance, and whose implications they should take upon themselves to delineate. This is the approach developed by many authors in the social choice literature.

A typical example of this approach is the literature on ``formal welfarism'' \cite{fleurbaey_informational_2003}. The point of this literature is to formalize social welfare functions. The latter are functions allowing to compare several social states of affairs by aggregating individual information in these different states of affairs. The various possible functions have different properties, which in turn capture different value-judgments. By formalizing these functions and properties, the economist does not himself make any value-judgement. But he produces a tool that a decision-maker can use: if the decision-maker endorses a given value-judgement, then the economist can identify the corresponding property and then recommend the function that the decision-maker should use. In so doing, the economist is purely value-neutral. This line of argument fails, however, for at least two reasons.

First, the exercise that consists in equating a value-judgment with an axiom is more difficult than most social choice theorists seem to admit, and is perhaps doomed to be elusive. To explain why, let us take \cite{arrow_social_2012}'s “Non Dictatorship” axiom. This axiom states that there is no individual \emph{i} in society such that for all profiles and all pairs of alternatives \emph{x} and \emph{y}, if \emph{i} prefers \emph{x} to \emph{y}, then \emph{x} is preferred to \emph{y} at social level. This axiom is expected to be largely accepted as a minimal requirement for any collective decision rule.

But there are two, very different reasons why one can be expected to endorse this axiom. The first reason is that this axiom and its name echo our shared endorsement of democracy and our shared refusal of dictatorship, where ``dictatorship'' refers to a complex picture of political arrangements, associated in intricate ways with notions such as arbitrariness, illegitimacy, and rule by force, and so on. Our shared refusal of dictatorship, understood in this sense, is also nourished by our culture and historical experiences. But plainly enough, none of these is captured by Arrow's axiom. This leaves us with a second reason to reject the axiom, one that tighly sticks to its formal articulation, which is that we would reject an arrangement, whatever it might be, that happens to satisfy perfectly one and only one individual, always the same. The second reason is the only one that the axiom strictly speaking talks about. But it is upon the first one that hinges the political relevance of the axiom, because the first does not even talk about political settings or ideas. 
\commentOC{Tu veux dire … on the second one that. Je ne comprends pas. Il faut définir plus précisément alors ce que tu entends par des idées ou settings politiques, et pourquoi l’interprétation restreinte de ND ne rentre pas dans cette catégorie, et pourquoi c’est important qu’il y rentre.}In our view, this contrast shows that, although Arrow's axiom does capture shared normative intuitions, it certainly does not capture our shared refusal of dictatorship. Axioms undoubtedly are useful to formalize some aspects of value-judgments. But one cannot expect it to be possible to correctly and completely capture value-judgments or values in axioms. 
\commentOC{C’est tout autre chose, ça, non ? Et ça requiert une précision (qu’entends-tu par correctement et complètement ici ?) et une preuve. Et pourquoi serait-ce important ? On ne peut jamais rien faire parfaitement, probablement, mais ce n’est une objection à rien.}
Articulating values is an endless task, using axioms for that purpose is an endless part of this endless task.

One might retort here that most social choice theorist would agree, and most of them eschew making the hopelessly ambitious claim to identify axioms capturing values or value-judgements perfectly. Even if this were true, a second argument should lead us to reject the vision of decision science as a norm translater. One might admit the soundness of an axiom when looking at the axiom itself, while in fact rejecting the implications of this axiom.

\commentYM{Olivier, je te laisse développer ?}\commentOC{D’accord avec ce que tu as écrit ci-dessous, j’ai juste légèrement modifié pour insister sur les deux étapes, engagement puis conséquences.}

Arrow' s  caracterization of the dictator rule \cite{arrow_social_2012} again provides an apt example. Arrow's argument is based on axioms that the author presents as liable to be endorsed by most of his readers. The argument shows that the Dictator rule is caracterized (in some formal context) by the axioms of Universal Domain, Pareto Dominance, and Independence of Irrelevant Alternatives. It is easily imaginable, and could most probably be confirmed by experimental studies, that non expert individuals would happily accept each of these three axioms as capturing a part of their value judgment about the demands of fairness for a voting rule, if the axioms would be explained to them by focusing only on what each axiom demand separately. The point of the theorem and the reason why this result is so powerful, is that a series of axioms which are all acceptable yield dictatorship, which our imaginary individuals would certainly reject. A natural way out of the conundrum is to question the spontaneous adherence to the axioms, which prove, on due reflection, to be less commendable. This is the way out that Arrow himself suggested. \commentOC{Je ne mettrais pas cette dernière phrase, elle me semble trompeuse en suggérant qu’Arrow a pris une position dans ce débat en tranchant entre plusieurs options envisageables. Mais en un certain sens, c’est la seule échappatoire possible, non ?}


The above scheme through which social choice theorists pick-up axioms that correspond to the decision-maker's value-judgments therefore fails in both its stronger and its weaker version. In its weaker version, it fails because it is anchored in the unwarranted premise that decision-makers are able to make clear and definitive value-judgments once and for all, in the abstract. In its stronger version, on the top of that it also fails because it is anchored in yet another unwarranted premise: that it is undebatable that axioms can aptly capture value-judgments.

\commentOC{Dans ma perspective, cette section fait remarquer à quel point il est important d’aider le décideur à vérifier si $N$ est acceptable, on ne peut pas le laisser seul face à ce problème. Cette remarque s’applique également à la section précédente. Je trouve qu’il ne faudrait pas présenter nos remarques comme des réponses à deux approches distinctes.}

\section{A third model: Decision science upstream democracy}
\noindent The two models explored above are explicitly defended by some authors in the literature. 

\commentOC{ Are they?}
\commentYM{en economie, je dirais que oui. Je cherche des references.
“Le domaine des théories économiques de la justice, dont traite le présent ouvrage, appartient
à l’économie positive, et ne contient pas d’assertion normative ou prescriptive ! [...] Le positif,
dans l’acception proposée ici, ne concerne pas seulement la description des phénomènes, mais englobe
également l’étude des arguments et des raisonnements." Fleurbaey (1996:2),} 

A third model, more powerful than the above two, can be carved out to attempt to overcome their respective shortcomings. Though it is arguably implicit in many decision science contributions, this model has, to our best knowledge, never been explicitly articulated in the literature. We will accordingly devote a bit more space to present and explain it and its philosophical motivation.

The source of this third model can be found in the debates between procedural and substantive approaches to the legitimacy of political decisions \cite{meinard_what_2017}. These debates are a cornerstone of the literature on democracy in political philosophy. However, the key-terms of these debates are sometimes used in different senses. A brief clarification is therefore useful. There are actually two debates in the literature which are often articulated using the same terms, in spite of their profound differences. It is therefore useful to recall the basic structure of these two debates.

A first debate deals with the question whether policy decisions deserve to be called democratic depending on the so-called “output” of the decision, or depending of the process through which they have been taken (“input”) \cite{vatn_environmental_2016, backstrand_environmental_2010}. Proponent of an input theory of democracy claim that, if a decision has been taken through democratic procedures, then it is democratic, whatever its output. Proponents of output theories take the opposite stance. These two extreme approaches have problematic implications. \commentOC{Pas convaincu par “problematic implications”.} A radical input theorist would be led to claim that a decision to disenfranchise half the citizenry would be democratic if taken through a democratic procedure. 

\commentOC{Not an absurd implication if the
theorist also claims that this result is impossible, when
proper democratic procedure is applied. Quelqu’un peut, sans absurdité, être un partisan optimiste de la input theory, et affirmer que si les gens suivent le bon processus pour arriver à leur décision, ils ne prennent jamais de décision du genre dont tu parles. Note que cette dernière affirmation est factuelle, et non une condition sur les outputs acceptables. Plus généralement, je n’ai jamais bien compris ce genre de théorie (mais c’est sans-doute parce que j’ai très peu lu à leur sujet) : qu’est-ce qu’on a le droit de mettre comme contraintes pour juger qu’une procédure est bonne ? Parce que si par exemple on dit que une procédure est démocratique seulement à condition que chacun ait un droit de veto (ce que qqn pourrait j’imagine défendre au nom du libéralisme et de la valeur de l’individu), alors la conséquence dont tu parles est immédiatement exclue, et on peut être un partisan très tranquille de l’input theory. De même, un partisan de l’output theory pourrait (dans mon esprit) dire que toute procédure est bonne, du moment qu’elle aboutit (mettons) au jugement délibéré (à notre sens) des gens (sauf que nous c’est dans le cas d’un seul individu, mais on pourrait imaginer une extension quelconque à un groupe pour les besoins de mon raisonnement ici). Dans ce cas, le output theorist définit le bon output par une procédure ! Est-ce encore de la output theory, ou devient-ce de la input theory ? De manière liée, y a-t-il vraiment dichotomie entre input et output theory ? Je soupçonne qu’on peut être les deux à la fois. Mais, encore une fois, j’ai peut-être mal compris la distinction entre les deux, auquel cas il faudra que tu m’expliques ça. (Notre lectorat pourrait être plus connaisseur que moi sur cette question, il n’est peut-être pas nécessaire d’être suffisament explicite pour que je comprenne tout si je ne suis pas représentatif.)}

Symmetrically, a radical output theorist would be led to claim that a benevolent dictator could achieve a democracy. 

\commentOC{ I see no absurdity here. If the
dictator takes all preferences into account, what is the
problem?} 
\commentYM{je concois qu'on puisse ne pas etre choque par une telle idee. Je ne dis pas qu'il faut l'etre. Je note que la litterature considere que c'est un argument contre ces theories}
\commentOC{Je me demande s’il est pertinent de notre part de simplement répéter sans critique des arguments trouvés dans la littérature (et éventuellement s’appuyer dessus, puisque j’imagine que c’est ce que tu veux faire), s’ils ne nous convainquent pas ou si on pense qu’ils reposent sur une ambiguïté conceptuelle. Et si on ne veut pas vraiment s’appuyer sur ces arguments parce qu’on n’en aurait pas besoin, alors ils sont peut-être une digression à éviter.}

Most political philosophers admit that such implications are untenable (it is no part of our ambition in this article to take a stance on this issue), and therefore strive to elaborate theories of democracy that reach a commendable equilibrium between input and output. The output element in the notion of democracy then captures requirements that should be imposed to the proceedings of democratic processes from the outside. If these requirements are not fulfilled, the processes no longer deserve to be called democratic. %One easily understands how decision sciences can take advantage from this move. If there is such a thing as an element that should be imposed from outside to the proceedings of democratic processes, then this thing stands out as a perfect candidate to be a subject matter for decision sciences.

The second debate opposes purely procedural to substantive theories of democracy. Substantive theories claim that democracy is a matter of values, which can be materialized either in procedures, or in political outcomes, such as for example in rights that are entrenched in law. An example of such an approach is \cite{brettschneider_value_2006}, who claims that democracy is first and foremost a set of ``core values'', which can be materialized in the proceedings of constitutional courts just like it can be in votes and institutional proceedings more usually called ``democratic''. As opposed to substantive theories, purely procedural theories claim to account for democracy by delineating formal properties of decision-making procedures that are supposed to be purely value-neutral, and from which democracy would emerge. \cite{habermas_faktizitat_1992} is often presented as the canonical example of such an approach. Purely procedural approaches in that sense are criticized in the literature for being untenable, because it appears impossible to recommend procedural properties while remaining value-neutral. For example, a prominent procedural property is universal voting right. Critics of purely procedural approaches argue that there can be no reason to recommend universal voting right except if this recommendation expresses an endorsement of a value-judgment such as the judgment that human being are equal and should be treated as such. Symmetrically, purely substantive approaches are criticized because, if there is a substance that defines what counts as democratic (as they claim), then it means that there is not even a need for citizens to vote or express their view in order to achieve a democracy, which many authors take to be contradictory.

\commentOC{Voici comment je vois à ce stade l’articulation entre ces réflexions et le sujet qui nous intéresse. Les quatre approches s’intéressent à la relation $N ⇒ D$, avec $D$ la démocratie, et $N$ les normes suffisantes pour entrainer $D$. Input theory dit que $N$ contient des contraintes sur les procédures de décision (contraintes substantielles, c-à-d basées sur des valeurs, ou pas, c-à-d neutres). Output theory dit que $N$ contient des contraintes (substantielles ou pas) sur les résultats des décisions. Pure procedure theory dit que $N$ contient des contraintes neutres sur les procédures de décision. Substantive theory dit que $N$ contient des contraintes substantielles sur les procédures ou sur les valeurs (je ne sais pas lequel). On pourrait sans-doute transposer tout ça en parlant des normes nécessaires pour obtenir une décision légitime, car dans le fond les mêmes questions se posent à ce niveau il me semble, pour ce qui nous concerne. Quoi qu’il en soit, dans toutes ces approches, on retrouve une forme de notre questionnement précédent : si $N$ est substantiel, qu’est-ce qui le justifie à son tour ? Et si on ne veut pas mettre des valeurs dans $N$, comment obtenir réellement la démocratie (ou quelque décision légitime que ce soit) ?}

These two debates have important similarities, mainly because input theories and purely procedural theories similarly focus on procedures. However, most input theories would be termed ``substantive'' by proponents of purely procedural theories, because they promote procedural \emph{values}. Substantive theories can materialize in both output and input theories. The second debate is a deeply philosophical one, opposing form to substance or values, the first one is more pedestrian, opposing concrete, worldly procedures, to worldly states of affairs which are no less concrete: patterns of endowments, distributions of income, and so on.

The second, more philosophical debate is the more relevant one from our point of view here. This debate indeed drives a wedge thanks to which decision science expertise can enter the scene by the backdoor and find its place upstream any political process. Indeed, whatever the conclusion of the debate, it interestingly frames the issues surrounding democracy in such a way that a role for decision science becomes visible upstream democracy. In the purely procedural approach, decision sciences can play a role in identifying and characterizing the supposedly pure procedure -- without thereby making any value-judgment. In the substantive approach, decision science can play a role in identifying the substance of democracy -- be it a matter of input or output. If a middle way has to be found between these two extremes, in this middle way the proper role of decision science is a bit of the two.
\commentOC{Hmm, avec ce positionnement, DS sert à trouver $N$, mais ça n’aide pas à répondre aux questions posées dans mon commentaire précédent. (Je me rends compte que, du coup, ce n’était peut-être pas les questions que tu avais en tête.)}

This third model hence appears to be liable to materialize in two, very different guises, depending on whether it is anchored in a substantive or a purely procedural stance. In this article, we will eventually recommand a specific version of this model. But identifying which one is to be recommended will first require some more theorizing, to which the two sections to come are devoted. 

\commentOC{Si je te suis bien, tu proposes une
autre place à la DS que celle envisagée par le FW (formal
welfarism). Ce n’est donc pas contradictoire avec le FW, mais
bien sur un autre plan. Je ne vois pas alors le lien avec la
critique du FW.
Pour le dire autrement, je vois deux questions distinctes : 1)
est-ce que DS est utile dans la perspective FW ; 2) est-ce que
DS est utile en amont ; et je ne vois pas de lien entre ces
deux questions.}
\commentYM{je ne comprends pas ce que tu dis. Je propose un troisieme modele. Une chose ambigue dans la premiere version du texte est que le titre pouvait laisser penser que je critiquais ce modele dans ce paragraphe comme je le fais pour les deux autres modeles. A cette etape en fait, je ne fais qu'exposer le modele. Je ne pense pas que les critiques du modele de la section 2 soient operantes pour critiquer ce modele}
\commentOC{Dans ma vision, il y a deux contextes, et deux questions : justifier d’obtenir $R$ à partir d’affirmations de type $N ⇒ R$, et justifier d’obtenir $D$ à partir d’affirmations de type $N ⇒ D$. Si on n’établit pas de liens entre ces deux questions, à quoi servent les premières sections de cet article ? Je pensais que tu allais lier toutes ces questions. Mais ici on dirait plutôt que tu prétends que la place de DS est dans l’aide à la justification de la démocratie (dont tu parles Section 4), et non l’aide, plus généralement, à la recommandation ou à la décision.}

\section{Varieties of flight from normativity}
\noindent The third model just introduced identifies a place where decision sciences can play a role. Let us investigate the kind of role they can play there.

The purely procedural approach seems to offer a possibility for decision sciences to play such a role without compromising with ``substance'', that is, without having to make value-judgments. By contrast, in the substantive approach, or in any mixed theory involving procedural and substantive approaches, decision science takes it upon itself to address an openly normative task, and in that case we need an account of how it can endorse this task.
But such formulations are ambiguous. So far, we have taken as synonymous the phrases ``making value-judgments'', being ``substantial'' and being ``normative''. What do these phrases really mean? To clarify this matter, it is useful to come back to an analysis of a couple of historical cornerstones of contemporary political philosophy: the stance articulated by \cite{rawls_political_2005} and \cite{habermas_moralbewustsein_1983}.

\cite{rawls_political_2005}'s theory epitomizes what \cite{estlund_democratic_2009} called the ``flight from substance''. He did not want his theory to make any value-judgment about the kind of state of affair that should prevail in a democratic society. He therefore argued that a policy is democratically legitimate if it is based on a constitution whose justification is acceptable to all ``reasonable'' citizens. But he did not want to make value-judgments about democratic processes either. He therefore further argued that the very definition of reasonableness should be something for reasonable citizens to pick-up. He thereby attempted to eschew making any value-judgments in his account of legitimacy and reasonableness. This was supposed to be a complete flight from substance, in the sense that this account was supposed to eschew any value-judgment, whatsoever. If such an approach were tenable, it would provide a positive subject matter for positive decision scientists to study without making any value-judgment.

This approach fails, however, for reasons articulated in different versions most prominently by \cite{habermas_reconciliation_1995} and \cite{estlund_democratic_2009}. Let us start with \cite{estlund_democratic_2009}'s argument because it is simpler to summarize. \cite{estlund_democratic_2009} noticed that, if one admits, following Rawls, that the notion of reasonableness should be selected by reasonable people themselves, there is an ``impervious'' plurality of groups that could select themselves as being ``reasonable''. He concluded rawlsian political philosophers have no choice but to make bold claims about the content of the concept of reasonableness. \cite{habermas_reconciliation_1995}'s argument is, to a large extent similar. He criticizes Rawls' presentation of his notions of the ``veil of ignorance'' and the ``overlapping consensus'' as \emph{devices} whose real-life functionning can give rise to principles of justice. According to \cite{habermas_reconciliation_1995}, these notions are rather rhetorical tools thanks to which Rawls exposes principles of justice that he deduces from various philosophical notions, such as the one of a moral person, which Rawls presupposes. Rawls' flight from substance hence abruptly collapses in a retreat back to the substantial inquiry into the nature and features of a moral subject.
\commentOC{Je me demande si on pourrait mettre cette remarque dans le même cadre que précédemment, pour plus de clarté. Ainsi, Rawls, pardon, [26], propose de résoudre le problème $N ⇒ D$ (mais la réflexion s’étend je trouve immédiatement à $N ⇒ R$) en faisant appel à : $N$ est acceptable aux yeux de tous les citoyens raisonnables qui réfléchissent. Ce qui est une extension d’une des stratégies dont on a parlé plus haut (mais qui n’a pas les mêmes défauts, puisqu’elle ne fait pas appel à la notion d’immédiateté, de bon sens, mais est plus réfléchie). À ça, on répond : 1) oui mais raisonnable n’est pas défini, c’est un peu facile jeune homme ; et 2) si le processus par lequel les citoyens réfléchissent et délibèrent est lui-même soumis à des normes, comme Rawls propose (telles que le voile d’ignorance), ça re-pose le problème de justification des normes à un ordre supérieur, et donc ne résoud pas le problème, c-à-d qu’on peut l’accuser, comme le font Mrs Onze ou Quatorze, d’avoir choisi ces normes comme ça l’arrange.}

Rawls' attempt is therefore the paragon of the failure of the flight from substance (which is neither a logical flaw nor a failure to unveil illuminating insights, but a failure to strip the argument from its normative anchorage, or ``substance''). Whereas Habermas played a key-role in unveiling the problems crippling Rawls' approach, he himself embarked on a flight of his own, which has both important differences and interesting similarities with Rawls' flight from substance. \cite{habermas_moralbewustsein_1983}'s usage of the notion of ``performative contradiction'' is particularly interesting in this respect. Habermas deploys this argument in his presentation of his ``discourse ethics'', for which he was concerned to provide foundations. In broad outline, this argument states that refusing to accept the purportedly minimal normative discourse ethics would be committing a contradiction. This means that everyone implicitly already admits the tenets of discourse ethics. By falling back upon a supposedly always already entrenched consensus, the performative contradiction argument would allow Habermas to go a step farther than Rawls, in the direction of the flight from substance. By deploying this argument, Habermas not only eschewed taking a stance on what is a good or a bad policy, or on what is a good or a bad procedure, he went as far as striving to show that there is no stance to be taken about the foundations of morals, because we all always already agree on them. This would not only be a flight from substance, but a flight from any element of ``advocacy'' in normative attitude -- where ``advocacy'' is used here to refer to the attitude that consists in an active espousal of and support for normative tenets \emph{as normative}. 
\commentOC{Je suppose que tu veux dire ici que l’advocacy est l’attitude consistant à se choisir des normes $N$, puis à dire de ces normes qu’elles devraient être acceptées par tous, parce qu’elles sont (en un sens) correctes, acceptées par tous au sens où chacun devrait suivre les recommandations découlant de ces normes, et consistant à soutenir activement ces normes-là. On pourrait aussi comprendre que l’advocacy est l’attitude plus générale consistant à trouver parfaitement valide de passer de $N ⇒ R$ à $R$, quel que soit $N$, ou à condition de choisir $N$ “convenablement” (avec des contraintes donc sur la façon dont $N$ est choisie), et consistant à soutenir ce raisonnement général.}

If the performative contradiction argument manages to achieve this purpose, it means a there is material that decision scientists (among others) can study without stepping outside the realm of positive science, but with normative bearings. Unfortunately, the performative contradiction argument fails to evacuate normativity to such a radical extent. Indeed, what if someone happens to reject the tenets of discourse ethics? He commits a performative contradiction, and then? We can try to show him that he committed a contradiction, but what if he does not surrender? We have no choice but to take a normative stance with respect to his attitude, and claim that he should not have that kind of attitude. Rather than evacuating normativity, the performative contradiction argument hence hides its normative content in its implicit assumption that contradiction ought to be excluded. Can it justify this assumption? It cannot do it simply by claiming that we all already accept that contradictions should be avoided, because this would lead to an infinite regress.
\commentOC{The argument works only if indeed, some people reject the tenets of discourse ethics. It is an empirical question, maybe.}

Though, according to the reading just spelled out, the performative contradiction argument can be interpreted as an attempt to completely evacuate normativity, Habermas himself never articulated his approach as a complete rejection of normativity. In some of his writings he argues that it is impossible to develop a perfectly satisfactory positive account of normative behavior
\commentOC{Est-ce réellement la même question que celle qui nous intéresse ? Je trouve “develop a perfectly satisfactory positive account of normative behavior” très obscur.}%
, which suggests that he probably did not intend to evacuate normativity altogether. This impossibility to reduce normative discourse to a positive discourse on normativity stems, in our view, from the impossibility to eliminate ``advocacy'' from normative attitude without flighting from normativity altogether. 
\commentOC{Je ne comprends pas cette phrase, elle est difficile à comprendre car contient beaucoup de négations et peut-être par conséquence du commentaire précédent.} 

\commentOC{Par ailleurs, il me semble que si on trouve une norme $N$ qui entraine des recommandations ($N ⇒ R$), et qu’on pouvait, par un moyen magique, être absolument certain que $N$ est effectivement acceptée par tous, alors on pourrait raisonnablement affirmer que les recommandations sont valides pour tous ($R$). Dans ma définition de normatif (qui est que on passe de $N ⇒ R$ à $R$ sans validation de $N$), la procédure cesserait donc d’être normative.}

The failed models explored above (sections 2-3) strive to salvage decision scientists from a task they don’t want to take upon themselves: make value-judgments, advocate a vision. In this sense, these models exemplify a flight from advocacy. In some of its versions (the ones endorsing pure proceduralism), the third model (section 4) can also exemplify this flight. 
Our discussion of Rawls' and Habermas' attempted flights from normativity however suggested that this flight is a flawed strategy. It mainly renders invisible the normative content of theories and arguments. 
\commentOC{Je trouve qu’on peut affirmer ça simplement en restant sur le concept pur de normatif. Le détour par l’avocat obscurcit le message, et je ne vois pas ce qu’il apporte. En plus, ce n’est pas la saison.}
Instead of clinging to this stance, we suggest that, as decision scientists, we have to take upon ourselves to reflexively identify the normative stance that we take when we do our job. We have to inquire into the very normative foundations of our deeds and creeds as decision scientists. If the model that sees decision sciences as standing upstream democracy is to hold water, it should take a form endorsing such a normative view -- or so, we will argue.
\commentOC{Il y a ambiguïté sur ce qui est entendu par norme, ici, en tous cas si on parle de la recommandation individuelle. $\mathscr{N}$ : la norme supérieure selon laquelle \og{}le débat est bon\fg{} (à savoir, en gros, ce qui sous-tend notre proposition DJ). Ou $N$ : une “norme” plus spécifique, plus locale, dépendante du contexte et du problème, par exemple : \og{}you ought not to push the fat guy over the bridge\fg{}. $N$ est, à mon sens, justifiée par sa résistance aux contre-arguments, et donc découle de l’acceptation (éventuelle) de $\mathscr{N}$. Laquelle as-tu en tête dans ce § ?}

\section{Two visions of what ought to be done}
\commentE{P. Grill: tester sur le réalisme moral de Putnam}
\commentYM{j'ai peur que cela n'embrouille les choses, je pense preferable de juste dire que le terme ``realism'' est pris en un sens tres particulier dans ce texte}
\noindent Advocacy, in its crudest form, has its drawbacks, however. Though the failures of the flight from normativity suggest that advocacy should not be completely abandoned
\commentOC{Ça me semble complètement à l’envers, cette argumentation. La raison pour laquelle il faut adopter l’advocacy, s’il le faut, n’est pas simplement que certaines autres tentatives d’échapper à la normativité ont voulu également échapper à l’advocacy et ont échoué : c’est que tu proposes une façon de faire qui résoud le problème de la normativité, identifié au début de l’article ; qu’on est bien d’accord (ou on imagine que le lecteur l’est) qu’il faut le résoudre ; et qu’on ne voit pas comment faire d’autre, parce que les autres tentatives de réoudre le problème, \emph{qu’elles tentent d’échapper à l’advocacy ou pas}, ont échoué. Dit autrement, je trouve que les échecs précédents ne suggèrent pas grand-chose concernant l’advocacy, ni en faveur ni en défaveur. Mais l’argument en sa faveur, tu vas (on imagine) le donner ci-dessous, c’est tout simplement que passer par-là permet de contribuer à résoudre le problème de la normativité.}, it is therefore important to catch a more clearly articulated picture of normativity, liable to help us to understand how and why advocacy can become a problem, and how and why we can retain it. For that purpose, we want to elaborate on a very basic structure of normative reasoning -- one that, despite its simplicity, has to our best knowledge never received the attention it deserves in the literature.

There are, we argue, two very different approaches to what ought to be done. The first one we call ``moral realism'', and the second one ``the quest for justifications''.
``Moral realism'', in our definition, is the attitude of people who admit (sometimes explicitely, but most of the time implicitly, and certainly often without being fully aware of it) that they have a special access to a form of moral truth, and therefore can sometimes claim, without further ado, that this or that is ``right'' or ``good''. Notice that the phrase ``moral realism'', understood in various senses, plays important roles in the philosophical literature. We do not claim that our notion encompasses all these senses. Our argument makes sense when one uses our definition of ``moral realism'', and we do not make any broader claim. The ``quest for justification'', in our definition, is the attitude of people who endlessly keep on arguing about the justification of their stances about what ought to be done: there never is a point when they stop arguing and looking for further arguments and simply say ``that's the way it is''. Advocacy is a problem only when it is in the hands of a moral realist.

Indeed, a worrying feature of moral realism is that either it converges towards a quest for justifications, or it collapses in an apology of violence. Indeed, imagine that, as a moral realist, you face a challenge to your vision of what ought to be done. You can reply by justifying your stance, and in so doing you give up realism and lean towards the quest for justifications. Or you can ankylose on your stance and strive to impose it by force. In that case, either violence is part of your vision of what ought to be done, in which case you are stuck in the apology of violence, or your vision of what ought to be done rejects violence and your moral realism is repudiated. The only stable moral realism is therefore the apology of violence -- where ``stable'' means here that this attitude can survive without converging towards the attitude with which we contrast is, that is: the quest for justification.

\commentOC{ Ça me semble clair que la
recherche de justification est bonne, je ne crois pas que ça
requière une preuve. Un philosophe remettrait-il ça en
question ? Par ailleurs, la violence ne peut pas toujours être
condamnée, c’est une autre question à mon avis. Il faudrait
peut-être parler de ce qu’il faut faire dans le cas a priori plus
raisonnable où 1) on adopte la voie de la justification et 2)
on constate un cul-de-sac. Mais ça me semble être un autre
débat (et compliqué).}

\commentYM{je ne sais pas, peut-etre que ce que je dis est trivial. Ce ne serait pas la premiere fois!... ni la derniere}

What about the quest for justifications? Is it stable? The quest for justifications can be transient: one can be ready to argue up to a certain point, and then fall back upon realism. In such a case, the stability of this quest for justification is determined by the stability of the moral realism on which it falls back. What if it is not transient -- if it does not fall back upon moral realism? One can claim to take advantage of the structure of reasoning deployed in the substantive vs. procedural debate to reject this possibility. The argument would unfold as follows. If you endorse the quest for justification, it means that you endorse the values underlying the idea that moral stances should be backed by justifications. And these very values are the core of your moral realism.

\commentOC{Il faudrait ici aussi distinguer
dans moral realism le cas où on s’appuie sur des choses que
personne ne remet en question (tq le besoin de
justification).}
\commentYM{a mon sens, le moral realism consensuel est le pire ennemi, il a toutes les chances de cacher des incomprehension ou des refoulements de remises en question. C'est probablement un point sur lequel insister quelque part...}
\commentOC{Ok, si ce que tu veux dire, c’est qu’il est bon de chercher activement des contre-arguments, d’accord. (Mais cette thèse n’est pas très précise : faut-il chercher en dépensant beaucoup d’énergie ? Ou juste garder en tête que les contre-arguments pourraient exister ? Faut-il financer systématiquement en priorité les gens qui sont sceptiques des lois physiques connues actuellement ?) Je pense, encore une fois, qu’il n’est pas nécessaire d’invoquer un réaliste moral (qui à mon humble avis est un homme de paille, si c’est qqn qui pense que chercher activement des contre-arguments, c’est mal) pour faire passer ce message, il suffit d’affirmer cette position et passer à la suite.}

The quest for justification would hence unavoidably fall back on moral realism. However, the brand of moral realism on which the quest for justification would unavoidably fall back is of a very special kind. By definition, if one sticks to this moral realism by violence, one is no longer embarked in the quest for justification. This moral realism is therefore one that immediately falls back upon the quest for justifications. Hence, though the quest for justifications arguably is underlain by values, those values are immediately redirected towards a quest for justifications. The quest for justifications therefore is stable.

The reader will certainly have noticed that our notion of a quest for justification is, in sereval important respect, quite close to some notions developped by Habermas, and in a sense it is even quite close to some elements of his performative contradiction argument criticized above. We indeed full-heartedly acknowledge the influence of Habermas. As far as we know, Habermas however never explicitely spelled out the precise ideas captured by our notion of a quest for justification. And the purely exegetic question whether he would endorse our formulation falls beyond our scope in the present article. 

The quest for justifications is, we claim, what we should advocate as decision scientists. Not \emph{because} most people would certainly endorse it. But because it is stable and it is a very basic structure of moral reasoning whose sole stable counterpart is the apology of violence. We claim that we should not be afraid or shy, as decision scientists, to advocate it. There is no reason to flight this advocatory stance, no reason to (hopelessly) attempt to flight this substance, no reason to strive to reduce it to putative positive foundations. Accordingly, a version of the model placing decision sciences upstream democracy that endorses this normative stance has no reason to flight this substance and, quite the contrary, has the means to entrench its own normative stance: it is an embodiment of the quest for justification. 

\commentOCf{Je trouve difficile de te suivre, car ta thèse n’est pas encore claire à ce stade dans mon esprit. Que veut dire exactement être prêt à débattre ? Dans un sens très faible, tout le monde est d’accord, mais je pense que tu veux aller au-delà.}

\commentOCf{La raison pour laquelle les pratiquants de la \href{https://en.wikipedia.org/wiki/Nintendo_DS}{DS} n’aiment d’habitude pas beaucoup la norme \og{}j’aime le débat\fg{} est à mon avis qu’ils ne savent pas comment faire pour vérifier, en pratique, qu’une recommandation est bien légitime, appuyée sur une norme aussi vague. Ça ressemble plutôt à une invitation à transformer la DS en propagande ou en écharpage de tricot, si on ne précise pas ce point important. Je pense que c’est une force de notre cadre DJ de rendre ce point (un peu plus) clair. Mais si le lecteur de ce papier-ci n’en a pas connaissance, il risque fort d’adhérer à cette objection également. Et notre proposition (DJ) n’est qu’une ébauche, elle ne donne pas une manière concrète de faire, donc en attendant, le problème n’est pas résolu.}

At this stage, the reader might think that this provisional conclusion is self-evident. We indeed hope that it is, in the sense that our aim was to capture a normative stance that all decision scientists will be liable to endorse -- not as a matter of fact, but for normative reasons. What is more questionable is whether decision scientists all see this normative stance as advocacy, in our definition of this term. And a question that remains open at this stage is: what are the implications of endorsing this normative stance? The next section explores this question.

\commentOC{Qqn remet-il ça en question,
vraiment ? C’est l’essence même de la philosophie que de
prétendre que le débat réfléchi est bon. Et il est consensuel
que DS ne s’adresse pas à ceux qui ne veulent pas débattre
réflexivement, il me semble. N’y a-t-il pas un risque
d’enfoncer des portes ouvertes, ici ?}
\commentYM{possible. En meme temps, comme ca on se fait moins mal qu'en enfoncant des portes fermees... Plus serieusement, je pense qu'en pratique l'aide a la decision se passe tres souvent de faire l'effort de se justifier. J'ai developpé des exemples dans mon papier sur la legitimite, et en ce moment avec Juliette on est sur un papier qui va dans ce sens egalement}
\commentOC{Peut-être que leur lecture m’aiderait à comprendre ta thèse.}

\section{The recommended model}
\noindent The proposition that we defend in this article is hence that, as decision scientists, we should advocate the quest for justification, which provides the right account of our normative stance and the appropriate model for our possible interventions, until upstream democray if justified. This is unquestionably, openly a normative stance, one that we should take upon ourselves to advocate, by arguing in favor of it, and by enacting it -- which, by definition, is the same thing.
\commentOC{Si tout le monde est d’accord, pourquoi argumenter ? Et comment ?}

This approach faces a difficult problem, however. This problem reflects a basic and very important ambiguity, which actually again reproduces the structure of the procedural/substantive debate. The quest for justification consists in displaying good arguments for the stances we take, judgments we make, etc. But what is a ``good'' argument: is it one that happens to be accepted, or one that should be accepted?
The formal literature on argumentation \cite{dung_acceptability_1995,besnard_elements_2008} seems to favor the second option, because it defines arguments as good arguments—as arguments that should be accepted if properly understood. 

\commentOC{Pas nécessairement, chez Dung.}
\commentYM{faudra que tu m'expliques. Les 3-4 paragraphes ici sont a completer}

And so does, as explained above, the Rawlsian and Habermassian literature, because it is concerned with a counterfactual concept of acceptability: Rawls is not interested in justifications that real people usually accept or will accept, but in justifications that citizens \emph{would} accept, if they were reasonable -- that is, if they had the features that moral persons have in his theory. %More empirically-minded approaches strive to capture the arguments that are acceptable to people, in the sense that people will accept them, if they are given the opportunity to make up their mind about.

%\commentOC{Tu en parles au pluriel, mais je ne
%connais pas de telle approche à part la nôtre.}

 %Though less theoretical than the above-mentioned philosophical literature, these approaches in their various guises are anchored in counterfactual conditional clauses that they bind themselves to be able to normatively account for. 
 
 %\commentOC{ Pas compris.}
 
% So, just like the notion of acceptability appeared to conceal a normative element, so does the notion of opportunity to make up one’s mind about an argument. In the end, empirically-minded approaches prove to be anchored in normative assumptions, which have to do with the goodness of being anchored in justifications, and the goodness of respecting the plurality of points of view that one can have on a given justification.
 
 %\commentOC{Mélange décision individuelle et
%collective. L’approche proposée ne requiert par d’accepter
%les points de vues des autres, si c’est pour se forger sa
%propre opinion.}
 
We have to find a way out of this conundrum without falling back on a variant of moral realism. In other words, we have to find means to distinguish the kind of justifications that we can consider commendable, from those that we cannot, without falling back on a moral realist understanding of what is a ``good'' or ``acceptable'' justification. \cite{meinard_what_2017} ventured a partial solution to the problem in a study of the concept of legitimacy, based on two tenets.

The first tenet is ``incrementalism''. ``Incrementalism'' holds that the idea that one can be able to capture a definitive list of criteria defining what is a good argument, what is the reasonable, what is acceptable, and so on, is illusory. According to ``incrementalism'', one had better work incrementally, to improve step by step justifications, argumentations, etc. This tenet hence consist in accepting that any attempt to specify criteria to capture what makes a justification acceptable is bound to be provisional, and that one can only laboriously improve the blunt concept of acceptability step by step.

The second tenet is ``primacy of practice'': instead of searching for acceptability criteria through theoretical reflection, one should take the stance that consists in putting justifications to the test in real-life situations. According to ``primacy of practice'', justifiability is not a property of a recommendation formulated by a decision analyst, it is a property of the attitude of the analyst producing it.
\commentOC{Ne devrait-on pas écrire : it is a property of the recommendation, joined to the arguments used to support it? Ça me semble moins ambigu.}

These two tenets allow to overcome a very problematic feature of philosophical approaches such as Rawls' and Habermas', which is that such approaches seem to admit that philosophical arguments can decide if an argument is a good one, or a justification is an acceptable one, even without a single real stakeholder or decision-maker having the opportunity to make up his/her mind about it, which is bound to look unacceptably disconnected from reality.

\cite{meinard_what_2017} used these two tenets to delineate an understanding of the concept of legitimacy. But this approach can easily be translated into an answer to our question in the present article. This translation would state that, as decision scientists, our quest for justification should consist in:
\begin{itemize}
\item[i.]	Systematically justifying our recommendations;
\item[ii.]	Being ready to defend them against criticisms, even when none are formulated;
\item[iii.]	Actually enacting this defense, when we actually face criticisms.
\end{itemize}
\commentOCf{Ici je me rends compte que j’avais mal compris ton propos sur la justification. Je laisse quand-même mes commentaires précédents pour que tu voies (peut-être) ce que j’avais mal compris.}
\commentOCf{Je ne comprends pas le deuxième point. Comment se défendre d’une critique absente ?}

However, as it stands, this approach has two worrying weakness.
The first weakness can readily be identified by referring to the literature on epistemic injustice \cite{fricker_epistemic_2007}. Some people and group have access to knowledge, others have not. The former are in a position to articulate criticisms, the latter are not. By imposing that decision scientists should be ready to defend their recommendations and enact this readiness, the above approach exposes decision science only to part of the spectrum from which criticisms can come. What if there are no criticisms addressed at us, whereas many could have been, but were not, because of epistemic injustices? 
\commentOCf{N’as-tu pas déjà répondu en disant qu’on doit se défendre contre les critiques mêmes non formulées ? Par ailleurs, le problème me semble bien plus général : comment être sûr qu’on a trouvé collectivement toutes les critiques pertinentes ? Même si tout le monde avait les mêmes \og{}accès épistémiques\fg{}, on ne pourrait pas s’en assurer !}
Would we still feel confident in claiming that our approach materializes the idea of a quest for justifications in such a case? We do not think so. There is therefore something amiss in the above account.

Can we fix the problem by identifying a specific group of people that should be the source of criticisms, or even a procedure that should be used to encourage the formulation of such criticisms? Such an approach would not work, because it is hopeless to believe that we can once and for all identify a relevant group of people (or set of groups of people) and/or a magical procedure. We need a more astute approach.

Though incomplete, the above account contains the key to the conundrum. Indeed, this account successfully addresses a problem which is, in a sense, structurally similar to the one we are now addressing. We cannot expect to be able to articulate once and for all what a good justification is. That is why the above account does not spell out a purportedly definitive list of criteria, and rather delineates an attitude on the part of decision analysts. The same trick can do the job here again. We cannot identify once and for all a perfectly relevant group of people and/or a perfect procedure. What we can do is identify an attitude that will be conducive to the quest for justification, and this attitude is a requirement to actively elicit criticisms. This requirement is the missing element in our account to fix its first weakness.

But this account, although completed to fix the first weakness, still has another worrying weakness, which can be captured by raising the question: when can one admit that one has produced \emph{enough} justifications? Imagine that you have embarked in a discussion with a stakeholder who always has new criticisms to raise. In such a situation, in our logic, should one admit that you cannot stop the discussion at some point or another? This would give to your contradictor a serious advantage, which certainly is undue: if he is ill-intentioned, he can condemn you to indefinitely argue in vain.

The key to solve this new problem can be found in Sen's contribution to political philosophy. \cite{sen_idea_2009} interestingly distinguished two visions of justice: the transcendental vision and the comparative vision. In broad outline, when applied to the notion of justice, a transcendental vision in his jargon is one that claims to answer questions such as ``what is just?'', ``what does justice consist in?'', ``which criterion can one use to decide if a given situation is just or unjust?'' and so on. By contrast, a comparative vision is one that claims that such questions are unanswerable, and that the point of theories of justice is rather to address comparative questions such as ``is situation \emph{x} more just or less just than situation \emph{y}?'' We do not claim in this article to adjudicate the credentials of an application of this account to theories of justice. We simply want to pinpoint that this structure of argument elegantly solves the remaining problem with the above account. If one admits that one cannot give a definitive answer to the question ``what is a good application of decision science in a democracy?'', but that this is not too bad, because the only truly relevant question is a comparative one such as ``is application \emph{A1} better than application \emph{A2}?'', then one can appreciate the relevance of the above account.

The comparatist approach is no panacea. It is, in a sense, a deflationary approach, and we should be prepared to live with the corresponding modesty. 
\commentOC{Comprends pas. En quoi cela résoud-il le problème ? Si le décideur argumente tout le temps en demandant de justifier tout, concernant l’affirmation A1 > A2 (et l’affirmation inverse), que fait-on ?}
Besides, it can arouse expectations that it cannot deliver. It is therefore important to clarify what a comparatist reading of our approach cannot do. It does not provide a metric, no generally applicable mechanical means to compare any two applications without discussions. There is bound to be myriads of hard cases where one application of decision science will appear better than another on some respect, but worse on another respect. We do not claim to solve this problem, and doubt that it can be solved at a general level.

To sum up, the model that we recommend is that, as decision scientists, we should advocate a quest for justification that consists in:
\begin{itemize}
\item[i.]	Systematically justifying our recommendations;
\item[ii.]	Being ready to defend them against criticisms, even when none are formulated;
\item[iii.]	Actively eliciting criticisms, especially from people or groups that we have good reasons to think are not spontaneously liable or willing or able to articulate such criticisms;
\item[iv.]	Actually enacting this defense, when we actually face criticisms;
\item[v.]	Understanding our own justifiability in a comparative sense.
\end{itemize}
\commentOCf{Mon impression est que tout ça va un peu trop vite, dans cette section. Par exemple, je ne trouve pas que iii. (partie \og{}especially\fg{}) soit justifié. Pour des raisons d’efficacité, on pourrait vouloir chercher des arguments plutôt chez ceux qui peuvent en donner facilement.}

\section{Practical pitfalls?}
\noindent At this stage, skeptical readers will certainly think that our reasoning is a purely theoretical construct which will in all likelihood face insuperable difficulties if one tries to implement it in practice. We indeed think that practical implementability is a crucial issue. Although we cannot have the ambition in this article to definitely settle the implementability issue in all its dimensions, the present section will delve into more concrete considerations in order to sketch how our approach can overcome what we take to be the most difficult implementation challenge it faces. This challenge is the unavoidable rejoinder that the quest for justification will collapse on the pitfalls of disagreements and clashing orders of justification.

In order to address this important criticism, let us take an example. The choice of this example among myriads of possible examples unavoidably involves some arbitrariness, but we will make a point to ensure that this arbitrariness will not undermine our conclusions. The example chosen is environmental economic valuation. One might think at first sight that this example is very specific. However, this example is less reductive than one might think, and it has interesting features for the purpose of exploring the above criticism. Indeed, in practice, a very vast series of issues are actually gathered under this umbrella, and accordingly focusing on this topic means encompassing applications of economic valuation methods to a very large range of issues \citep{kontoleon_biodiversity_2007}. Besides, in recent years, the environment as a valuation object provided the opportunity for researchers to introduce many methodological innovations, which can and possibly will be applied in the years to come to many other kinds of valuation objects \citep{bartkowski_economic_2017}. Therefore, environmental economic valuations actually encompass a very large array of methods, perhaps more than any other valuation object. This makes it an especially interesting example to discuss how different economic methods can be applied to similar objects and justified, which can provide relevant tests to assess the credentials of our approach.

The methods most prominently used in this domain are based on measurements of people's willingness to pay (WTP) (we will leave aside here the more anecdotal case of methods based on willingness to accept), as it can be elicited by surveys addressed at individual or revealed by these individuals' behavior on markets \citep{meinard_ethical_2016}. The first case encompasses contingent valuation and choice experiment, while the second one mainly encompasses travel cost and hedonic pricing methods. A prominent alternative which has seen numerous empirical applications to biodiversity in recent years is deliberative valuation \citep{bartkowski_economic_2017}. This refers to methods based on choice-experiments or WTP questionnaires embedded in protocols of exchanges of information and discussions. These methods were originally motivated by critical discussions of the ontological assumptions underlying WTP-based valuation methods, and more recently \citep{bartkowski_beyond_2018} explored their positive philosophical underpinnings, referring mainly to \citep{sen_idea_2009}. A third, much less developed method was introduced by \citep{meinard_measuring_2017}. Based on theories of impartialization \citep{kolm_macrojustice:_2004} protocols and on a reading of \citep{rawls_theory_1999}' philosophy, this method attempts to capture the impartial preference of citizens for the funding of biodiversity conservation policies.

As opposed to WTP-based valuation methods, an interesting feature of the literature on deliberative valuation and impartial preference measurement is that this literature explicitly explores the respective normative justifications of the methods. A similar identification of normative justifications for WTP-based methods was attempted by \citep{meinard_ethical_2016}, who proposed a typology of WTP-based valuation methods depending on their respective possible normative justifications. These various contributions thereby allow to draw a typology associating each method with a normative justificatory framework: impartial preference measurement is associated with Rawls' framework, deliberative valuation with Sen's philosophy, stated preference WTP-based methods with ``welfarism'' and revealed preference WTP-based methods with ``endowment conservatism''. This precise definition of these normative frameworks should not concern us here, and we will accept the validity of these associations for the purpose of the argument. The important point from our point of view here is that, thanks to these elements, one can identify a series of argumentative justifications that can be deployed to defend all those kinds of methods in their application to our example.

One might be tempted to conclude that, seen through the lenses of the quest for justification, all three methods would allow decision scientists to implement and justify them, following our recommended model. The latter would accordingly seem to be entirely irrelevant in practice. More precisely, one might surmise that various people will, in all likelyhood, disagree on which justification is convincing: some people will accept the justification underlying WTP-based methods, other will harshly criticize them and champion deliberative methods because they will find the justification underlying them more convincing, and so on. Such a scenario, where various groups of people adhere to different and largely irreconcialiable frameworks of justifications (\citep{boltanski_justification_2006}'s ``orders of justification''), is indeed considered to be a major phenomenon by some authors in environmental and social economics \citep{chateauraynaud_contrainte_2007}.

Though we take this criticism very seriously, 
\commentOC{Which criticism? I’m lost. The fact that people disagree? Is this a criticism of what we propose? Why?}
we argue that it misses an important aspect of our approach. The fact that argumented justifications can be carved out for the different methods simply means that the methods can be put to the test of their acceptability by various people or groups. This raises the question: how can one know if a justification is acceptable? A natural but mistaken answer would consist in claiming that a justification qualifies as acceptable if and only if it turns out to be accepted in all the situations in which it can be applied. However, here again, how can one perform that kind of test? At best one can say whether applications of a given methods \emph{have so far been accepted}, but this leaves aside all possible but non actual applications, and replaces acceptability by acceptance (falling in the trap that Habermas had earmarked in his criticism of Rawls).

This insuperable problem suggests that, if our approach were applied to methods, it would indeed collapse due to the empirical fact highlighted by the literature on ``orders of justification'': this fact is that various groups typically refer to different and largely irreconciliable orders of justifications, which can (according to some authors at least) be formalized as sets of normative axioms accepted by some groups but rejected by others. Accordingly, the question of whether the justification underlying a given method is acceptable is too general and abstract to be answered. But such a general acceptability of the justification of a method is not what our recommended model is about. Far from being a concrete challenge to our supposedly too abstract account, the ``orders of justification'' criticism only appears challenging because it raises an all too abstract problem. 

Our recommended model, as articulated above, does not refer to methods or methodological frameworks, it talks about what happens in concrete decision aiding processes (understood in the sense spelled out in \citep{tsoukias_concept_2007}). Decision aiding processes are concrete sets of continued interactions between decision analysts, decision-makers and concerned stakeholders. Because our recommended model is about concrete decision aiding processes, the important element in our approach is not the putative general justification underlying the methods used, but rather the justification that can be articulated for the specific usages of the methods put to use at this or that stage during the decision aiding process. Coming back to our example of environmental economic valuations, the various methods mentioned about can be used for very different purposes at various stages in concrete decision aiding processes. Stated and revealed preferences studies are often used to feed cost-benefit analyses \citep{layard_cost-benefit_1994}. However, as emphasized by \citep{meinard_ethical_2016}, the very same monetary valuations can just as well be used as arguments to strengthen public awareness of the importance of the object they value \citep{salles_valuing_2011}, or to put a provisional figure on the impact that various kinds of actions can have on various groups of stakeholders, or in many other ways. In these various cases, the justification that can be developped can take advantage of the general justification underlying the method used, but it can also integrate many other elements pertaining the context, and the specific usage of the method within the particular decision aiding process.

Claiming that our approach is impractical because it is hopeless to find decision aiding methods that will prove acceptable with respect to all the ``orders of justification'' is therefore irrelevant. This irrelevant criticism however suggest another, more powerful rejoinder that deserves to be addressed as well. This possible criticism would claim that, even within a concrete decision process, when a decision scientist sets himself to articulate a justification and fulfill the requirements i-v of our framework, it is highly likely that in most cases he will face at one stage or another someone who will stick to a given ``order of justification'', and whatever the decision scientist's effort to discuss the justification of his recommendation, and the reasons why some of the axioms underlying his interlocutor's ``order of justification'' should be abandonned, still his interlocutor will reject his justification.

We perfectly agree that such difficult situations can happen, and are prepared to admit that they might often happen. However, taking such situations to be fatal pitfalls for our approach would be confusing two largely disconnected issues: on the one hand, the issue that we address in this article, which is the one of the normative status of decision science interventions, and on the other hand, the issue of the possibility to generate consensual decisions. If a decision scientist in a concrete decision process articulates a justification for his recommendation but, whatever his efforts, he always faces the stubborn resistance of some groups, one cannot take this failure of consensus to prove that the decision scientist failed. If he neglected to articulate a justification, then he failed; if he developed a justification but a more justified recommendation was developped by someone else, then he failed; by contrast, if he pursued as far as he could the quest for justification but faced the resistance of someone sticking to moral realism with respect to a given ``order of justification'', then it cannot be said to have failed.

We do not deny that issues such as how consensual group decisions can be generated, or in which conditions will there be this or that pattern of moral realism among sets of groups of decision-makers and stakeholders, are important. Quite the contrary, we think that decision science approaches which would be able to tackle such issues would be considerably more justifiable. What we claim is that such issues go well beyond the requirements encapsulated in our recommended model of the proper place of decision science.
\commentYM{faire reference ici aux deliberated judgments}

\section{Conclusion}
\noindent In this article, we have introduced a normative account of the role of decision sciences in a democracy. For that purpose, we have emphasized the failure of various strategies designed to allow decision scientists to eschew value-judgements, flight ``substance'' and remain normatively neutral. Such attempts take the crude forms of a failed model of economics as pure science and a failed model of decision science as a transparent norm translater, and the subtler form of a model that endorses pure proceduralism to arrange a place for a purportedly value-neutral decision science upstream democracy.

We have strived to demonstrate that all these models fail, which has led us to develop an inquiry into a basic structure of moral reasoning, unveiling an important contrast between ``moral realism'' and the ``quest for justifications''. When then argued that, if one endorses the tenets of ``incrementalism'' and ``primacy of practice'', and if one conceives of the justifiability of applications of decision sciences in a comparative approach, then decision sciences can find their place upstream democracy without having to endorse a moral realist understanding of the concepts of a ``good'' or ``acceptable'' justification. We accordingly ended up recommending the corresponding model for the proper role of decision sciences in a democracy, which can be articulated as follows:

As decision scientists, we should advocate a quest for justification that consists in:
\begin{itemize}
\item[i.]	Systematically justifying our recommendations;
\item[ii.]	Being ready to defend them against criticisms, even when none are formulated;
\item[iii.]	Actively eliciting criticisms, especially from people or groups that we have good reasons to think are not spontaneously liable or willing or able to articulate such criticisms;
\item[iv.]	Actually enacting this defense, when we actually face criticisms;
\item[v.]	Understanding our own justifiability in a comparative sense.
\end{itemize}

Though this models aims to provide a general account of the normative stance that we should take as decision scientists when we work in a democratic setting, and therefore is undoubtedly very ambitious in its scope, we emphasize that it is also very modest in many respects. It does not provide a metric, no generally applicable mechanical means to compare any two applications of decision sciences without discussions. There is bound to be myriads of hard cases where one application of decision science will appear better than another on some respect, but worse on another respect.

It is clearly part of our hope that this account of the normative stance of decision sciences will not come as a surprise for most decision scientists, and is rather liable to gather large support among them. However, though we accordingly expect that most decision scientist will agree with us at this stage, we surmise that most of them do not have clearly articulated ideas about the concrete implications of this basic normative stance. This is why we devoted considerable space to develop some concrete implications of our account, whose actual implementation might prove more disrupting for decision sciences practices than the cheer endorsement of our abstract normative account taken in its most abstract form.
 


%So far, the argument has been entirely abstract. This section will use a series of example to try to flesh it out to some extent.
%Let us start by a burning issue in environmental sciences, the current biodiversity crisis. The current rate of species extinctions is unprecedented. Many conservationist succumb to authoritarian temptations. Democracy looks poorly adapted to take bold decisions to save the planet. It seems like there is a choice to be made. Either we believe in democracy, participation, etc., in which case planet Earth will collapse; or an authoritarian regime will save the Planet and Humanity against its own willing. I think that such a debate is ill-conceived, and that the approach developed here solves the problem, at least to some extent. The idea that urgency trumps the need for participation and democracy in an argument like any other. If a justification based on this argument wins, then the decision to trump participation is not undemocratic, in any non-trivial sense.

%\commentOC{ Bof. Ta dissolution du débat joue
%sur l’ambiguïté du concept de démocratie. Si on prétend que
%la démocratie, c’est prendre les décisions en fonction des
%jugements délibérés du peuple, alors faire la procédure que
%tu proposes, c’est ne pas céder à l’urgence.}

%The same logic shows that this approach can be used to unlock some of the perennial apparent dilemma plaguing the functioning of democratic policies, due to the fact that the above mentioned output/input debate appears undecidable. A typical example in this respect is the case of the 1991 cancellation of the legislative elections in Algeria between the two rounds of the election, following the government’s understanding of the fact that it would certainly be overwhelmingly won by the “Front Islamic du Salut”, championing a largely undemocratic policy agenda. An output theory would call this decision legitimate, an input one would deem it undemocratic. Our approach claims that none of the alternatives (canceling the election or letting it unfold) is intrinsically legitimate, and none is intrinsically more legitimate than the other. The more legitimate one would be the one buttressed more thoroughly by a quest for justification on the part of its champions. This, clearly, is a local answer. Small scale debates here and there probably have unfolded and generated different winners in this respect. There is no general answer to this question. There are illuminating, well-structures local answers.


\section*{Acknowledgements}
\noindent We thank Jerome Lang, Philippe Grill and Juliette Rouchier for powerful comments and suggestions on this manuscript.

\section*{References}

\bibliographystyle{plain}

\bibliography{decision,philo-eco,beliefs,deliber}

\end{document}
