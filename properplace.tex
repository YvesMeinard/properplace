\RequirePackage[l2tabu, orthodox]{nag}%less problems with LaTeX code
\RequirePackage{silence}\WarningFilter{newunicodechar}{Redefining Unicode character}
\pdfgentounicode=1 %permits (with package glyphtounicode) to copy eg x ⪰ y iff v(x) ≥ v(y) from pdf to unicode data. 
\input{glyphtounicode}%nice copy from PDF
\documentclass[preprint, french, english, 11pt, authoryear]{elsarticle}%english main language
\usepackage[T1]{fontenc}
\usepackage[utf8]{inputenc}
\usepackage{lmodern}
\usepackage{newunicodechar}%able to use e.g. → or ≤ in source
\usepackage{babel}
\frenchbsetup{AutoSpacePunctuation=false, SuppressWarning=true}
\usepackage{setspace}
\usepackage{enumitem}
\usepackage{amsthm}
\usepackage{mathrsfs}
\usepackage{color}
\usepackage{natbib}
\usepackage{doi}
\usepackage{hyperref}
\usepackage{bookmark}% hyperref doc says: Package bookmark replaces hyperref’s bookmark organization by a new algorithm (...) Therefore I recommend using this package.
\bibliographystyle{abbrvnat}
\usepackage{etoolbox}
\apptocmd{\thebibliography}{\hfuzz=20cm\raggedright}{}{}
\usepackage[nolist,smaller,printonlyused]{acronym}
\begin{acronym}
\acro{DM}{Decision Maker}
\end{acronym}

\onehalfspacing
\newtheorem{theorem}{Theorem}
\newtheorem{acknowledgement}[theorem]{Acknowledgement}
\newcommand{\commentYM}[1]{\textcolor{blue}{YM: #1}}
\newcommand{\commentOC}[1]{\textcolor{red}{OC: #1}}
\newcommand{\commentOCf}[1]{\textcolor{red}{\selectlanguage{french}{OC : #1}}}
\newcommand{\commentE}[1]{\textcolor{green}{RelecteurExterne: #1}}
\newunicodechar{ℝ}{\mathbb{R}}
\newunicodechar{≠}{\ensuremath{\neq}}
\newunicodechar{≤}{\ensuremath{\leq}}
\newunicodechar{≥}{\ensuremath{\geq}}
\newunicodechar{→}{\ifmmode\rightarrow\else\textrightarrow\fi}
\newunicodechar{⇒}{\ensuremath{\Rightarrow}}
\newunicodechar{∪}{\cup}
\newunicodechar{∩}{\cap}
\newunicodechar{¬}{\ifmmode\lnot\else\textlnot\fi}
\newunicodechar{…}{\ifmmode\ldots\else\textellipsis\fi}

\newcommand{\adv}{\mathscr{N}}

\begin{document}
\title{Justification and decision sciences}

\author[ld]{Y. Meinard\corref{cor1}}
\author[ld]{O. Cailloux}
\cortext[cor1]{Corresponding author}
\address[ld]{Universit\'e Paris-Dauphine, PSL Research University, CNRS, UMR [7243], LAMSADE, 75016 PARIS, FRANCE}

\begin{abstract}
Decisions are a core subject matter for many economic theories and sub-disciplines, which can be collectively called ``decision sciences''. This article aims at clarifying the normative status of practices that consist in using insights from decision sciences to support decision-making, by formulating recommendations. When applied to a concrete problem, decision sciences obtain conclusions of the form $N ⇒ R$. Such conclusions contain recommendations conditionned by the norms or normative conceptions $N$ on which they rest. This article explores strategies deployed in the economic and philosophical literature to jump from $N ⇒ R$ to $R$, while eschewing to take a stance on $N$. We argue that these strategies fail. As an alternative, we argue that, as decision scientists, we should openly endorse, as part of our scientific practice, the norm $\adv$, according to which we should embark in a ``quest for justification'': the attitude of people who endlessly keep on arguing about the justification of their stances about what ought to be done. We display reasons to accept $\adv$. We then argue that endorsing $\adv$ in our scientific practice implies deploying an attitude that consist in:
\begin{itemize}
\item[i.]	Systematically justifying our recommendations;
\item[ii.]	Being ready to defend them against criticisms, even when none are formulated;
\item[iii.]	Actively eliciting criticisms;
\item[iv.]	Actually enacting this defense, when we actually face criticisms;
\item[v.]	Understanding our own justifiability in a comparative sense.
\end{itemize}
\end{abstract}

\begin{keyword}
Decision Aiding, Normative Economics, Legitimacy, Justification, Ethics of Operational Research
\end{keyword}

\maketitle
\section{Introduction}

Decisions are a core subject matter for many economic theories and sub-disciplines. In this article, we will use the loose phrase ``decision sciences'' to refer to all these economic approaches, which take decision-making as their main topic, ranging from operational research to social choice theory, through public choice theory, the microeconomic theory of choice, rational choice theory, multi-criteria decision aiding methodology, and so on.

Some of the studies gathered under this umbrella present themselves as purely academic contributions, concerned only with establishing scientific results. However, most decision sciences have practical application, both in the private sector and in policy-making \citet{tsoukias_policy_2013,marchi_evidence-based_2016}. In this process, knowledge to become advice, and purportedly scientific propositions to endorse the normative status of prescriptions. Issues such as whether the wisest should rule the masses, or how scientific insights can contribute to collective decision-making in different political regimes, are accordingly addressed by a vast literature. In this article, we claim neither to exhaustively review this immense literature, nor to do justice to all the aspects of the issues it tackles. We rather want to clarify the normative status of practices that consist in using insights from decision sciences to support decision-making.

By talking about ``normative status'', we mean that we want to clarify the extent to which these practices are indeed normative, and to articulate a normative account of them. 
The term ``normative'', used to qualify a “practice”, will be taken here in a broad sense, to refer to practices that aim at recommending some actions. This understanding is admittedly broad, as it encompasses practices concerned with notions such as ethics, justice, the good and the just, and so on. 
Though broad, this understanding is not all-encompassing. In particular, it excludes purely positive or purely empirical attempts to capture the above mentionned notions. For example, empirical studies aimed at capturing what people in a given group mean when using the term ``justice'', as examplified by \citep{gaertner_empirical_2012}, does not fall within our definition of a ``normative'' inquiry. Besides, we will be concerned here only with recommendations, and will leave aside more obligatory normative notions such as obligation.

Based on this understanding of ``normative'', the question articulated above in terms of ``normative status'' refers to the following difficulty. When applied to a concrete problem, decision sciences obtain conclusions of the form $N ⇒ R$. Such conclusions contain recommendations conditionned by the norms or normative conceptions $N$ on which they rest: accepting  $N$ leads to accepting the recommendation, and rejecting $N$ does not lead to any recommendation. However, it is desirable to obtain $R$, meaning obtain unconditionally valid recommendation, rather than $N ⇒ R$. This article discusses possible strategies to transition from $N ⇒ R$ to $R$.

Though this is, in a sense, the very old “is-ought problem” revisited again, this article will review in particular the economists’ positions toward this problem… \commentOCf{J’ai quand-même l’impression qu’il faudrait dire deux mots de pourquoi on pense qu’il est encore possible de dire des choses pertinentes concernant ce problème débattu mille fois, mais je ne trouve pas une bonne formulation… À éclaircir quand on aura avancé sur le reste du texte, peut-être.}

This article is divided into nine parts. Following the present introduction, sections 2 and 3 explore accounts of the role of decision sciences that more or less underlie a large part of the economic literature. We argue that these accounts are defective. Though they use different strategies, they both attempt to draw an impermeable divide between factual and normative claims, and to account for the role of decision sciences as confined to the factual realm. We argue that these accounts fail, for various reasons that have already been articulated in various strands of the literature. These relatively short parts should be seen as a lapidary review of existing knowledge, rather than as an original contribution. Section 4 explores an alternative account, which locates the purported role of decision sciences upstream democracy. In our view, this account is implicit in many contemporary approaches. However, to our best knowledge, it has so far never been spelled out explicitly. In sections 6 to 7, we explore some foundamental questions of normative theory so as to argue that a specific version of this third model deserves to be recommended. Section 8 delves into more concrete considerations in order to sketch how our approach can overcome what we take to be the most difficult implementation challenge it faces, mamely the fear that it might collapse on the pitfalls of disagreements and clashing ``orders of justification''. Section 9 concludes.

\section{Elusive strategies to leave $N$ outside the scientific part of the inquiry}
\commentYM{en fait, ta reformulation fait complètement sauter tout ce que racontait ma section ``pure science'' et les trucs sur Robbins. C'était une section tarte-à-la-crème pour mettre le lecteur en confiance. Je poserai quand même la question à quelques économistes pour voir si ça serait utile de la réhabiliter, en tout petit, pour rassurer le lecteur} 
A widespread vision among economists is that economics is a value-neutral, purely scientific endeavor. This approach admits that value-judgments are essentially non-scientific, and that economics as a science should therefore eliminate them.

This vision suggests a first strategy, $S1$, to jump from $N ⇒ R$ to $R$. $S1$ consists, for the economist, in sticking to conclusions of the form $N ⇒ R$, and letting \acp{DM}, or even “society” in general, decide of their attitude towards $N$ on their own. Deriving $R$ thus lies, in this strategy, outside the scientific part of the endeavor.

Being completely agnostic about $N$, $S1$ could be applied to any norm, however absurd or blatantly immoral, still leading to conclusions of the form $N ⇒ R$. Such a crude application of $S1$ would certainly produce hoards of results that would not be of interest to anyone, thereby basically amounting to a waste of resources. $S1$ accordingly does not captures what most researchers in economy, decision sciences, or related fields do.

In order to overcome this problem, a natural amendment of $S1$ is to confine the inquiry to norms self-evident enough for \acp{DM}, or society, to (somehow) decide, without the help of science, whether they endorse them.
This defines a second strategy, $S2$: science obtains conclusions of the form $N ⇒ R$, and $N$ is ``minimal'', therefore, $R$ holds.

This strategy failed so far to produce examples of such norms, however. Candidate for such norms in the literature are axioms that, under their technical, decontextualized, expression, may indeed appear convincing. But as has been remarked \commentYM{by whom?}, in order to know whether they apply to some context, much has to be known and postulated about the context. This is enough to remove their self-evident nature. A second reason for rejecting such axioms as self-evidently valid norms is that accepting some norms logically entails the acceptance of their implications, and it is not evident for an individual to know whether all the implications of some norms are acceptable.

%A second problem of this strategy is that norms that must be self-evident must be minimal, so much so that possibly it prevents from giving recommendations that would appear relevant for non obvious reasons. 
%\commentOC{I really don’t like this point.}

As \cite{baujard_leconomie_2011} recalls, the issue of inter-personal comparisons of utility provided a prominent historical exercise for economists engaged in this endeavor, which illustrates our two arguments against $S2$. Inter-personal comparisons of utility involve value-judgments because one cannot claim that a given increase in the welfare of one person outweighs or even is equivalent to a given decrease in the welfare of a second person unless one judges the relative worth of the welfare of the two persons. A prominent conceptual trick to produce policy-relevant results despite the ban on the value-judgments involved in interpersonal comparisons of utility is the strong Pareto principle, stating that state of affairs \emph{y} is better than state of affairs \emph{x} if no one is worse off in \emph{y} as compared to \emph{x}, and at least one person is better off in \emph{y} as compared to \emph{x}. But this principle has two important defects. As famously emphasized by \cite{sen_rationality_2004}, among others, the strong Pareto principle is blatantly normative. This very idea is certainly largely accepted, but many authors often present this normativity as ``minimal'', in the sense that it seems innocuous to admit that most people, if not everyone, accepts this normative principle. We argue that this claim is more debatable than most authors seem to admit. Take for example a slightly unequal situation \emph{x} where individual \emph{i} is quite well-off whereas individual \emph{j} is poor. Compare with situation \emph{y} where \emph{i} receives a bonanza and \emph{j}'s situation is unchanged. \emph{y} Pareto dominates \emph{x}, but is considerably more inequal. The idea that everyone would claim that \emph{y} is better than \emph{x} is far from self-evident. It relies on questionable assumptions about a total absence of aversion to inequality and envy. Of course, one can retort that our argument applies the Pareto principal to wealth, and that if it is applied to some aggregated welfare index integrating aversion to inequality and envy, then it might no longer be the case that \emph{y} Pareto dominates \emph{x}. But this very rejoinder shows how difficult it might be to set the situation so as to render it evident that evryone will agree on the Pareto principal. We therefore claim that one has to face the fact that, if one wants to claim that the Pareto principle is minimal, one cannot simply claim that it is self-evident, one has to prove it.

We should emphasize, at this stage, that what we just said about the strong Pareto principle is certainly true of all the axioms that the literature usually admits to be normative, but quickly qualifies as \emph{minimally} normative.

To illustrate our second reason for rejecting $S2$, consider Arrow' s caracterization of the dictator rule \cite{arrow_social_2012} (we thank Ulle Endriss for this example)\commentYM{on ne peut pas citer votre papier?}. Arrow's argument is based on axioms that the author presents as liable to be endorsed by most of his readers. The argument shows that the Dictator rule is caracterized (in some formal context) by the axioms of Universal Domain, Pareto Dominance, and Independence of Irrelevant Alternatives. It is easily imaginable, and could most probably be confirmed by experimental studies \commentYM{j'enlèverais : s'il existe des données exp, on les cite, mais sinon je ne suis pas pour qu'on verse dans la spéculation méta-empirique}, that non-expert individuals would willingly accept each of these three axioms as capturing a part of their value-judgment about the demands of fairness for a voting rule, if the axioms were explained to them by focusing only on what each axiom demands separately. The point of the theorem and the reason why this result is so powerful, is that a series of axioms which are all acceptable yield dictatorship, which our imaginary individuals would certainly reject. A natural way out of the conundrum is to question the spontaneous adherence to the axioms, which prove, on due reflection, to be less commendable.

$S1$ and $S2$ are two strategies whereby economists claim to use decisions sciences while leaving $N$ outside the purview of their scientific inquiry. We have argued that both strategy fail because they are predicated on a implausible premisse: that \acp{DM} do not need the help of decision sciences to make up their mind about $N$.

\section{Elusive strategies to retreat from $N$}
Another strategy designed to jump from $N ⇒ R$ to $R$ can be found in the philosophical literature, in debates between procedural and substantive approaches to the legitimacy of political decisions \cite{meinard_what_2017}. These debates are a cornerstone of the literature on democracy in political philosophy. However,there are actually two debates in the literature which are often articulated using the same terms, in spite of their profound differences. A brief clarification is therefore useful.

A first debate deals with the question whether policy decisions deserve to be called democratic depending on the so-called “output” of the decision, or depending of the process through which they have been taken (“input”) \cite{vatn_environmental_2016, backstrand_environmental_2010}. Proponent of an input theory of democracy claim that, if a decision has been taken through democratic procedures, then it is democratic, whatever its output. Proponents of output theories take the opposite stance. Recast in our simple formalism, this debate is about recommendations $R$ talking about democratic credentials. Input theorists claim that from $N ⇒ R$ we can derive $R$, for some appropriate norms $N$ constraining processes. Output theorists claim that there exists some norms $N$ constraining the possible recommendations which permit to go from $N ⇒ R$ to $R$. As a third possibility, one could claim, and this is the stance most philosophers adopt, that the right norms are ones that have both constraints on the process and on the resulting recommendations

%Let us relate these two attitudes to the subject of this article through our simple notation. This debate wonders how to justify a recommendation. (Justification is understood here as meaning: how to consider a recommendation as democratic.) The question as we view it is: what are the appropriate norms that would justify a recommendation? Thus, for which $N$ can we, whenever $N ⇒ R$, get rid of $N$ so as to obtain simply $R$, an unconditionally valid recommendation? 

%Recall that $N ⇒ R$ means that if norms $N$ were accepted as indeed putting sufficient constraints (to the process or content of $R$), $R$ would be legitimate. Hence, $N ⇒ R$ holds whenever $R$ indeed were reached by following an appropriate process, or satisfies the right prerequisites, as judged by $N$.
%But in order to justify $R$, merely obtaining $N ⇒ R$ is insufficient. . In all cases, what is searched for are norms permitting to go from $N ⇒ R$ to $R$, an issue to which philosophical literature has devoted much energy and to which we will come back.

The second debate opposes purely procedural to substantive theories of democracy. Substantive theories claim that democracy is a matter of values, which can be materialized either in procedures, or in political outcomes, such as for example in rights that are entrenched in law. An example of such an approach is \cite{brettschneider_value_2006}, who claims that democracy is first and foremost a set of ``core values'', which can be materialized in the proceedings of constitutional courts just like it can be in votes and institutional proceedings more usually called ``democratic''. As opposed to substantive theories, purely procedural theories claim to account for democracy by delineating formal properties of decision-making procedures that are supposed to be purely value-neutral, and from which democracy would emerge. \cite{habermas_faktizitat_1992} is often presented as the canonical example of such an approach. 

In our formalism, substantive theories claim that $N$, in order to be powerful enough to permit stating $N ⇒ R$ for interesting $R$’s, must contain value-judgments. Whereas purely procedural theories claim that it is possible to obtain $N ⇒ R$, for the kind of recommendations we are interested in (such as universal voting rights), with $N$ not containing value-judgments. Observe that this strategy seems promising for us, as such an $N$, non-substantive, would indeed pave a way towards obtaining $R$ from $N ⇒ R$.

%Purely procedural approaches in that sense are criticized in the literature for being untenable, because it appears impossible to recommend procedural properties while remaining value-neutral. For example, a prominent procedural property is universal voting right. Critics of purely procedural approaches argue that there can be no reason to recommend universal voting right except if this recommendation expresses an endorsement of a value-judgment such as the judgment that human being are equal and should be treated as such. Symmetrically, purely substantive approaches are criticized because, if there is a substance that defines what counts as democratic (as they claim), then it means that there is not even a need for citizens to vote or express their view in order to achieve a democracy, which many authors take to be contradictory.
%\commentOCf{J’enlèverais ce §.}

These two debates have important similarities, mainly because input theories and purely procedural theories similarly focus on procedures. However, most input theories would be termed ``substantive'' by proponents of purely procedural theories, because they promote procedural \emph{values}. Substantive theories can materialize in both output and input theories. The second debate is a deeply philosophical one, opposing form to substance or values, the first one is more pedestrian, opposing concrete, worldly procedures, to worldly states of affairs which are no less concrete: patterns of endowments, distributions of income, and so on. The second, more philosophical debate is the more relevant one from our point of view here.

The purely procedural approach seems to offer a possibility for decision sciences to play such a role without compromising with ``substance'', that is, without having to make value-judgments. By contrast, in the substantive approach, or in any mixed theory involving procedural and substantive approaches, decision science takes it upon itself to address an openly normative task, and in that case we need an account of how it can endorse this task.
But such formulations are ambiguous. So far, we have taken as synonymous the phrases ``making value-judgments'', being ``substantial'' and being ``normative''. What do these phrases really mean? To clarify this matter, it is useful to come back to an analysis of a couple of historical cornerstones of contemporary political philosophy: the stance articulated by \cite{rawls_political_2005} and \cite{habermas_moralbewustsein_1983}.

\cite{rawls_political_2005}'s theory epitomizes what \cite{estlund_democratic_2009} called the ``flight from substance''. He did not want his theory to make any value-judgment about the kind of state of affair that should prevail in a democratic society. He therefore argued that a policy is democratically legitimate if it is based on a constitution whose justification is acceptable by all
 ``reasonable'' citizens. But he did not want to make value-judgments about democratic processes either. He therefore further argued that the very definition of reasonableness should be something for reasonable citizens to pick-up. He thereby attempted to eschew making any value-judgments in his account of legitimacy and reasonableness. This was supposed to be a complete flight from substance, in the sense that this account was supposed to eschew any value-judgment, whatsoever. If such an approach were tenable, it would provide a positive subject matter for positive decision scientists to study without making any value-judgment.

This approach fails, however, for reasons articulated in different versions most prominently by \cite{habermas_reconciliation_1995} and \cite{estlund_democratic_2009}. Let us start with \cite{estlund_democratic_2009}'s argument because it is simpler to summarize. \cite{estlund_democratic_2009} noticed that, if one admits, following Rawls, that the notion of reasonableness should be selected by reasonable people themselves, there is an ``impervious'' plurality of groups that could select themselves as being ``reasonable''. He concluded that rawlsian political philosophers have no choice but to make bold claims about the content of the concept of reasonableness. \cite{habermas_reconciliation_1995}'s argument is, to a large extent, similar. He criticizes Rawls' presentation of his notions of the ``veil of ignorance'' and the ``overlapping consensus'' as \emph{devices} whose real-life functionning can give rise to principles of justice. According to \cite{habermas_reconciliation_1995}, these notions are rather rhetorical tools thanks to which Rawls exposes principles of justice that he deduces from various philosophical notions, such as the one of a moral person, which Rawls presupposes. Rawls' flight from substance hence abruptly collapses in a retreat back to the substantial inquiry into the nature and features of a moral subject.
\commentOCf{J’attends de voir si on est d’accord sur la stratégie générale avant de proposer une autre présentation pour cet aspect. Je pense qu’il n’y a que l’aspect présentation à discuter ici, ce n’est pas essentiel.}\commentYM{go ahead}

Rawls' attempt is therefore the paragon of the failure of the flight from substance (which is neither a logical flaw nor a failure to unveil illuminating insights, but a failure to strip the argument from its normative anchorage, or ``substance''). Whereas Habermas played a key-role in unveiling the problems crippling Rawls' approach, he himself embarked on a flight of his own, which has both important differences and interesting similarities with Rawls' flight from substance. \cite{habermas_moralbewustsein_1983}'s usage of the notion of ``performative contradiction'' is particularly interesting in this respect. Habermas deploys this argument in his presentation of his ``discourse ethics'', for which he was concerned to provide foundations. In broad outline, this argument states that refusing to accept the purportedly minimal normative discourse ethics would be committing a contradiction. This means that everyone implicitly already admits the tenets of discourse ethics. By falling back upon a supposedly always already entrenched consensus, the performative contradiction argument would allow Habermas to go a step farther than Rawls, in the direction of the flight from substance. By deploying this argument, Habermas not only eschewed taking a stance on what is a good or a bad policy, or on what is a good or a bad procedure, he went as far as striving to show that there is no stance to be taken about the foundations of morals, because we all always already agree on them.

If the performative contradiction argument manages to achieve this purpose, it means a there is material that decision scientists (among others) can study without stepping outside the realm of positive science, but with normative bearings. Unfortunately, the performative contradiction argument fails to evacuate normativity to such a radical extent. Indeed, it is empirically easy to verify by looking at the relevant literature that the tenets of discourse ethics are not accepted by all philosophers, some of whom explicitely \commentOC{I suppose} denying that they are performing a contradiction by rejecting the proposed tenets \commentYM{I'd simply cite \cite{heath_communicative_2001}}. Taking seriously the performative argument would lead us to refuse discussion with such philosophers, on the account that they are incoherent. This is however not what we ought to do, and most probably not what Habermas would recommend. To put it otherwise, continuing the discussion requires acceptance of a higher-level norm that says that discussion ought to be pursued if the validity of the performative argument is denied.
\commentOC{I currently think we should present the norm we will propose before this discussion, but I am not sure.} \commentYM{je ne comprends pas ce que tu dis là}

\commentOCf{Modifier cette conclusion en fonction des décisions sur le reste de l’article.}
The failed strategies explored above ($S1$, $S2$) strive to salvage decision scientists from a task they don’t want to take upon themselves: make value-judgments, advocate a vision. In this sense, these models exemplify a flight from advocacy. In some of its versions (the ones endorsing pure proceduralism), the third model (section 4) can also exemplify this flight. 
Our discussion of Rawls' and Habermas' attempted flights from normativity however suggested that this flight is a flawed strategy. It mainly renders invisible the normative content of theories and arguments. 
Instead of clinging to this stance, we suggest that, as decision scientists, we have to take upon ourselves to reflexively identify the normative stance that we take when we do our job. We have to inquire into the very normative foundations of our deeds and creeds as decision scientists. If the model that sees decision sciences as standing upstream democracy is to hold water, it should take a form endorsing such a normative view -- or so, we will argue.


\commentYM{oublions mes histoires d'advocacy. C'est pas mûr dans ma tête. Ce sera pour un prochain article!}
\section{The quest for justification}
The various strategies explored in the previous sections attempted to evacuate $N$. We argued that all these strategies fail. We do not claim that our inventory of strategies is exhaustive. We propose to embark in a completely different strategy, and we argue that it provides a sufficiently satisfactory solution to the problem we highlighted in this article. We propose a norm $\adv$, and an attitude, that consists in always admitting, whenever a recommendation is proposed, that its validity rests upon the acceptance of the norm $\adv$. $\adv$ refers to the norm that we should embark in a ``quest for justification'': the attitude of people who endlessly keep on arguing about the justification of their stances about what ought to be done.

In this section, we display four reasons to endorse $\adv$. In the next section, we will explain how endorsing $\adv$ solves our problem.

The first reason is that the main alternative to endorsing $\adv$ is to endorse a stance that can be called ``moral realism''. In our definition, this refers to the attitude of people who admit (sometimes explicitely, but most of the time implicitly, and certainly often without being fully aware of it) that they have a special access to a form of moral truth, and therefore can sometimes claim, without further ado, that this or that is ``right'' or ``good''. Notice that the phrase ``moral realism'', understood in various senses, plays important roles in the philosophical literature. We do not claim that our notion encompasses all these senses. Our argument makes sense when one uses our definition of ``moral realism'', and we do not make any broader claim. ``Moral realism'' understood in our sense, in predicated on a philosophy or ``revelation'' or ``epiphany''. We do not claim that this is an implausible moral view, but this a moral view that seems difficult to integrate in a scientific endeavour.

The second reason is that $\adv$ is stable and ``attracts'' moral realism. A moral realist may claim he also has a solution to our problem of normativity. For him, there are norms $N$ which are such that, whenever they imply some recommendations, suffice to make those recommendations justified in themselves. We can thus pass from $N ⇒ R$ to $R$. This rests upon the claim of special access from the moral realist to what is right (captured in $N$). However, as soon as anybody will disagree with $N$, the moral realist is bound to abandon his pretention of validity, resort to violence, or try to justify himself. The last attitude would make his position undistinguishable from ours (moral realism is then attracted by $\adv$. Observe that we use here the resort to violence as a negative consequence of moral realism in our argumentation, which itself is a normative attitude. \commentYM{sur la stabilité, il a quelques trucs de ma version d'avant que j'aurais quand même bien sauvé...}

The third reason is that \commentOCf{Aussi : refus de la révélation ; qu’est-ce qu’on demande quand on demande une preuve ; les philosophes qui défendent d’autres normes semblent accepter la nôtre ; il est nécessaire d’accepter la nôtre si on veut la critiquer.}\commentYM{ça, je te laisse rédiger, car je ne vois pas la différence avec mon argument de la convergence vers $\adv$}

The fourth reason is that, we surmise that $\adv$ is truly minimal, in the sense that... although we still need to rely on accepting a norm in order to justify the recommendation, this norm is not as the other norms. We actually apply strategy $S2$ but claim this is truly an example of a norm that may be self-evidently valid. Also, it is not necessary to be able to foresee all the consequences of this norm in order to decide whether we accept it. \commentOC{Is this correct?} \commentYM{not sure about that...}

The reader will certainly have noticed that our notion of a quest for justification is, in sereval important respect, quite close to some notions developped by Habermas, and in a sense it is even quite close to some elements of his performative contradiction argument criticized above. We indeed full-heartedly acknowledge the influence of Habermas. As far as we know, Habermas however never explicitely spelled out the precise ideas captured by our notion of a quest for justification. And the purely exegetic question whether he would endorse our formulation falls beyond our scope in the present article. 

\commentOCf{Conclusion à modifier en fonction de nos convergences de position sur le reste de l’article.}
The quest for justifications is, we claim, what we should advocate as decision scientists. Not \emph{because} most people would certainly endorse it. But because of the four reasons above. We claim that we should not be afraid or shy, as decision scientists, to advocate it. There is no reason to flight this normative stance, no reason to (hopelessly) attempt to flight this substance, no reason to strive to reduce it to putative positive foundations. Accordingly,strategy in which decision sciences endorse this normative stance has no reason to flight this substance and, quite the contrary, has the means to entrench its own normative stance: it is an embodiment of the quest for justification. 

At this stage, the reader might think that this provisional conclusion is self-evident. We indeed hope that it is, in the sense that our aim was to capture a normative stance that all decision scientists will be liable to endorse -- not as a matter of fact, but for normative reasons. What is more questionable is whether decision scientists all see this normative stance as advocacy, in our definition of this term. And a question that remains open at this stage is: what are the implications of endorsing this normative stance? The next section explores this question.

\commentOCf{Envisager des exemples de la littérature pour montrer à quel point les chercheurs ne justifient pas leurs normes, et considèrent cet aspect comme hors de leur discipline, pour réduire le risque que le lecteur trouve ça évident (je suis moi-même maintenant persuadé que ça ne l’est pas, si notre thèse est bien comprise). Mes anciens exemples peuvent fournir de la matière, mais je voudrais d’abord m’assurer de notre thèse.}\commentYM{a minima on peut citer les débats sur la non-transparence de l'économie du bien-être, avec les refs données par Antoinette dans le bouquin philo éco}

\section{The recommended model}
The proposition that we defend in this article is hence that, as decision scientists, we should advocate the quest for justification, which provides the right account of our normative stance and the appropriate model for our possible interventions. This is unquestionably, openly a normative stance, one that we should take upon ourselves to advocate, by arguing in favor of it, and by enacting it -- which, by definition, is the same thing.
\commentOC{Si tout le monde est d’accord, pourquoi argumenter ? Et comment ?}

This approach faces a difficult problem, however. This problem reflects a basic and very important ambiguity, which actually reproduces the structure of the procedural/substantive debate presented above. The quest for justification consists in displaying good arguments for the stances we take, judgments we make, etc. But what is a ``good'' argument: is it one that happens to be accepted, or one that should be accepted?

The Rawlsian and Habermassian literature bends towards the second reading, because it is concerned with a counterfactual concept of acceptability: Rawls is not interested in justifications that real people usually accept or will accept, but in justifications that citizens \emph{would} accept, if they were reasonable -- that is, if they had the features that moral persons have in his theory. \cite{habermas_faktizitat_1992} forcefully emphazises this counterfactual aspect.

We have to find a way out of this conundrum without falling back on a variant of moral realism. In other words, we have to find means to distinguish the kind of justifications that we can consider commendable, from those that we cannot, without falling back on a moral realist understanding of what is a ``good'' or ``acceptable'' justification. \cite{meinard_what_2017} ventured a partial solution to the problem in a study of the concept of legitimacy, based on two tenets.

The first tenet is ``incrementalism''. ``Incrementalism'' holds that the idea that one can be able to capture a definitive list of criteria defining what is a good argument, what is the reasonable, what is acceptable, and so on, is illusory. According to ``incrementalism'', one had better work incrementally, to improve step by step justifications, argumentations, etc. This tenet hence consist in accepting that any attempt to specify criteria to capture what makes a justification acceptable is bound to be provisional, and that one can only laboriously improve the blunt concept of acceptability step by step.

The second tenet is ``primacy of practice'': instead of searching for acceptability criteria through theoretical reflection, one should take the stance that consists in putting justifications to the test in real-life situations. According to ``primacy of practice'', justifiability is not a property of a recommendation formulated by a decision analyst, it is a property of the attitude of the analyst producing it.
\commentOC{Ne devrait-on pas écrire : it is a property of the recommendation, joined to the arguments used to support it? Ça me semble moins ambigu.}\commentYM{je ne pense pas que ce soit une propriété des R, peut-être de la manière dont ces R sont développées et implementées, c'est peut-être juste une question de définition de ce qu'est une R ?}

These two tenets allow to overcome a very problematic feature of philosophical approaches such as Rawls' and Habermas', which is that such approaches seem to admit that philosophical arguments can decide if an argument is a good one, or a justification is an acceptable one, even without a single real stakeholder or decision-maker having the opportunity to make up his/her mind about it, which is bound to look unacceptably disconnected from reality.

\cite{meinard_what_2017} used these two tenets to delineate an understanding of the concept of legitimacy. But this approach can easily be translated into an answer to our question in the present article. This translation would state that, as decision scientists, our quest for justification should consist in:
\begin{itemize}
\item[i.]	Systematically justifying our recommendations;
\item[ii.]	Being ready to defend them against criticisms, even when none are formulated;
\item[iii.]	Actually enacting this defense, when we actually face criticisms.
\end{itemize}

Clause i captures the idea that, as application of ``primacy of practice'', our endorsement of $\adv$ should first and foremost take the form of our actually articulating justifications.

Clause ii is here to prevent our using justifications that happen to be accepted, as a matter of fact, at the moment when we articulate them, but whose weaknesses we sweep under the carpet. It also enbodies ``incrementalism'', by admitting that, once we have found arguments in favour of something, we should try to look for ways through which they could be discarded. Real-life examples of decision-aiding practices that flout this clause are given in \cite{meinard_what_2017}.

Clause iii allows clause ii to embody ``primacy of practice'', by putting justications to the test in real-life, instead of confining them to theoretical criteria (without clause iii, clause ii would be a counterfacual, purely dispositional criterion).

Before proceeding, let us ponder on this formulation. Clearly, applying the formula [i-iii] is unlikely to ensure that we will be able to identify \emph{the} ultimate justification for our recommendation. For that, we would need to have access to all the possible arguments, all the possibly relevant information, and we would need a perfect definitive definition of what is a \emph{good} justification. This \emph{first best} is unreachable. A commandable \emph{second best} would be to be able to empirically capture which justifications are acceptable by \acp{DM} and concerned stakeholders once they have taken all the relevant arguments and counterarguments into account. However, to our best knowledge, there is no operational framework to date to capture such ``deliberated judgments'' (see however Cailloux and Meinard), and even once one will be available, it is likely that operationality constraints such as time constraints will make it impossible to deploy such a technology in all applications of decision sciences. The account that we are looking for, and which is for the moment provisionally captured by the formula [i-iii], should hence be seen as an operational third-best--one that captures the essence of $\adv$ in real-life conditions where knowledge is drastically bounded.

That said, as it stands, this approach has two worrying weakness.
The first weakness can readily be identified by referring to the literature on epistemic injustice \cite{fricker_epistemic_2007}. Some people and group have access to knowledge, others have not. The former are in a position to articulate criticisms, the latter are not. By imposing that decision scientists should be ready to defend their recommendations and enact this readiness, the above approach exposes decision science only to part of the spectrum from which criticisms can come. What if there are no criticisms addressed at us, whereas many could have been, but were not, because of epistemic injustices? 
\commentOCf{N’as-tu pas déjà répondu en disant qu’on doit se défendre contre les critiques mêmes non formulées ? Par ailleurs, le problème me semble bien plus général : comment être sûr qu’on a trouvé collectivement toutes les critiques pertinentes ? Même si tout le monde avait les mêmes \og{}accès épistémiques\fg{}, on ne pourrait pas s’en assurer !}
\commentYM{je ne pense pas qu'on puisse être jamais sûr de ça. C'est pourquoi j'insiste sur le fait que c'est l'attitude du decision scientist qui compte}
Would we still feel confident in claiming that our approach materializes the idea of a quest for justifications in such a case? We do not think so. There is therefore something amiss in the above account.

Can we fix the problem by identifying a specific group of people that should be the source of criticisms, or even a procedure that should be used to encourage the formulation of such criticisms? Such an approach would not work, because it is hopeless to believe that we can once and for all identify a relevant group of people (or set of groups of people) and/or a magical procedure. We need a more astute approach.

Though incomplete, the above account contains the key to the conundrum. Indeed, this account successfully addresses a problem which is, in a sense, structurally similar to the one we are now addressing. We cannot expect to be able to articulate once and for all what a good justification is. That is why the above account does not spell out a purportedly definitive list of criteria, and rather delineates an attitude on the part of decision analysts. The same trick can do the job here again. We cannot identify once and for all a perfectly relevant group of people and/or a perfect procedure. What we can do is identify an attitude that will be conducive to the quest for justification, and this attitude is a requirement to actively elicit criticisms. This requirement is the missing element in our account to fix its first weakness.

But this account, although completed to fix the first weakness, still has another worrying weakness, which can be captured by raising the question: when can one admit that one has produced \emph{enough} justifications? Imagine that you have embarked in a discussion with a stakeholder who always has new criticisms to raise. In such a situation, in our logic, should one admit that you cannot stop the discussion at some point or another? This would give to your contradictor a serious advantage, which certainly is undue: if he is ill-intentioned, he can condemn you to indefinitely argue in vain.

The key to solve this new problem can be found in Sen's contribution to political philosophy. \cite{sen_idea_2009} interestingly distinguished two visions of justice: the transcendental vision and the comparative vision. In broad outline, when applied to the notion of justice, a transcendental vision in his jargon is one that claims to answer questions such as ``what is just?'', ``what does justice consist in?'', ``which criterion can one use to decide if a given situation is just or unjust?'' and so on. By contrast, a comparative vision is one that claims that such questions are unanswerable, and that the point of theories of justice is rather to address comparative questions such as ``is situation \emph{x} more just or less just than situation \emph{y}?'' We do not claim in this article to adjudicate the credentials of an application of this account to theories of justice. We simply want to pinpoint that this structure of argument elegantly solves the remaining problem with the above account. If one admits that one cannot give a definitive answer to the question ``what is a good application of decision science in a democracy?'', but that this is not too bad, because the only truly relevant question is a comparative one such as ``is application \emph{A1} better than application \emph{A2}?'', then one can appreciate the relevance of the above account.

The comparatist approach is no panacea. It is, in a sense, a deflationary approach, and we should be prepared to live with the corresponding modesty. 
\commentOC{Comprends pas. En quoi cela résoud-il le problème ? Si le décideur argumente tout le temps en demandant de justifier tout, concernant l’affirmation A1 > A2 (et l’affirmation inverse), que fait-on ?}\commentYM{je ne vois pas le pb. Si le mec est pénibme, quand on en a marre on l'envoie paître... mais si un autre decision analyst va + loin que nous dans la justification, alors c'est lui qui a gagné}
Besides, it can arouse expectations that it cannot deliver. It is therefore important to clarify what a comparatist reading of our approach cannot do. It does not provide a metric, no generally applicable mechanical means to compare any two applications without discussions. There is bound to be myriads of hard cases where one application of decision science will appear better than another on some respect, but worse on another respect. We do not claim to solve this problem, and doubt that it can be solved at a general level.

To sum up, the model that we recommend is that, as decision scientists, we should advocate a quest for justification that consists in:
\begin{itemize}
\item[i.]	Systematically justifying our recommendations;
\item[ii.]	Being ready to defend them against criticisms, even when none are formulated;
\item[iii.]	Actively eliciting criticisms;
\item[iv.]	Actually enacting this defense, when we actually face criticisms;
\item[v.]	Understanding our own justifiability in a comparative sense.
\end{itemize}
\commentOCf{Mon impression est que tout ça va un peu trop vite, dans cette section. Par exemple, je ne trouve pas que iii. (partie \og{}especially\fg{}) soit justifié. Pour des raisons d’efficacité, on pourrait vouloir chercher des arguments plutôt chez ceux qui peuvent en donner facilement.}\commentYM{certes! il peut y avoir des trade-offs entre efficacité et bienfondé normatif, et entre plein d'autres choses aussi. Je ne vois pas ce que ça change}

\section{Practical pitfalls?}
At this stage, skeptical readers will certainly think that our reasoning is a purely theoretical construct which will in all likelihood face insuperable difficulties if one tries to implement it in practice. We indeed think that practical implementability is a crucial issue. Although we cannot have the ambition in this article to definitely settle the implementability issue in all its dimensions, the present section will delve into more concrete considerations in order to sketch how our approach can overcome what we take to be the most difficult implementation challenge it faces. This challenge is the unavoidable rejoinder that the quest for justification will collapse on the pitfalls of disagreements and clashing orders of justification.

In order to address this important criticism, let us take an example. The choice of this example among myriads of possible examples unavoidably involves some arbitrariness, but we will make a point to ensure that this arbitrariness will not undermine our conclusions. The example chosen is environmental economic valuation. One might think at first sight that this example is very specific. However, this example is less reductive than one might think, and it has interesting features for the purpose of exploring the above criticism. Indeed, in practice, a very vast series of issues are actually gathered under this umbrella, and accordingly focusing on this topic means encompassing applications of economic valuation methods to a very large range of issues \citep{kontoleon_biodiversity_2007}. Besides, in recent years, the environment as a valuation object provided the opportunity for researchers to introduce many methodological innovations, which can and possibly will be applied in the years to come to many other kinds of valuation objects \citep{bartkowski_economic_2017}. Therefore, environmental economic valuations actually encompass a very large array of methods, perhaps more than any other valuation object. This makes it an especially interesting example to discuss how different economic methods can be applied to similar objects and justified, which can provide relevant tests to assess the credentials of our approach.

The methods most prominently used in this domain are based on measurements of people's willingness to pay (WTP) (we will leave aside here the more anecdotal case of methods based on willingness to accept), as it can be elicited by surveys addressed at individual or revealed by these individuals' behavior on markets \citep{meinard_ethical_2016}. The first case encompasses contingent valuation and choice experiment, while the second one mainly encompasses travel cost and hedonic pricing methods. A prominent alternative which has seen numerous empirical applications to biodiversity in recent years is deliberative valuation \citep{bartkowski_economic_2017}. This refers to methods based on choice-experiments or WTP questionnaires embedded in protocols of exchanges of information and discussions. These methods were originally motivated by critical discussions of the ontological assumptions underlying WTP-based valuation methods, and more recently \citep{bartkowski_beyond_2018} explored their positive philosophical underpinnings, referring mainly to \citep{sen_idea_2009}. A third, much less developed method was introduced by \citep{meinard_measuring_2017}. Based on theories of impartialization \citep{kolm_macrojustice:_2004} protocols and on a reading of \citep{rawls_theory_1999}' philosophy, this method attempts to capture the impartial preference of citizens for the funding of biodiversity conservation policies.

As opposed to WTP-based valuation methods, an interesting feature of the literature on deliberative valuation and impartial preference measurement is that this literature explicitly explores the respective normative justifications of the methods. A similar identification of normative justifications for WTP-based methods was attempted by \citep{meinard_ethical_2016}, who proposed a typology of WTP-based valuation methods depending on their respective possible normative justifications. These various contributions thereby allow to draw a typology associating each method with a normative justificatory framework: impartial preference measurement is associated with Rawls' framework, deliberative valuation with Sen's philosophy, stated preference WTP-based methods with ``welfarism'' and revealed preference WTP-based methods with ``endowment conservatism''. This precise definition of these normative frameworks should not concern us here, and we will accept the validity of these associations for the purpose of the argument. The important point from our point of view here is that, thanks to these elements, one can identify a series of argumentative justifications that can be deployed to defend all those kinds of methods in their application to our example.

One might be tempted to conclude that, seen through the lenses of the quest for justification, all three methods would allow decision scientists to implement and justify them, following our recommended model. The latter would accordingly seem to be entirely irrelevant in practice. More precisely, one might surmise that various people will, in all likelyhood, disagree on which justification is convincing: some people will accept the justification underlying WTP-based methods, other will harshly criticize them and champion deliberative methods because they will find the justification underlying them more convincing, and so on. Such a scenario, where various groups of people adhere to different and largely irreconcialiable frameworks of justifications (\citep{boltanski_justification_2006}'s ``orders of justification''), is indeed considered to be a major phenomenon by some authors in environmental and social economics \citep{chateauraynaud_contrainte_2007}.

Though we take this criticism very seriously, 
\commentOC{Which criticism? I’m lost. The fact that people disagree? Is this a criticism of what we propose? Why?} \commentYM{``this criticism'' réfère à la posiiton décrite dans le § précédent. Je vais essayer d'exprimer + clairement}
we argue that it misses an important aspect of our approach. The fact that argumented justifications can be carved out for the different methods simply means that the methods can be put to the test of their acceptability by various people or groups. This raises the question: how can one know if a justification is acceptable? A natural but mistaken answer would consist in claiming that a justification qualifies as acceptable if and only if it turns out to be accepted in all the situations in which it can be applied. However, here again, how can one perform that kind of test? At best one can say whether applications of a given methods \emph{have so far been accepted}, but this leaves aside all possible but non actual applications, and replaces acceptability by acceptance (falling in the trap that Habermas had earmarked in his criticism of Rawls).

This insuperable problem suggests that, if our approach were applied to methods, it would indeed collapse due to the empirical fact highlighted by the literature on ``orders of justification'': this fact is that various groups typically refer to different and largely irreconciliable orders of justifications, which can (according to some authors at least) be formalized as sets of normative axioms accepted by some groups but rejected by others. Accordingly, the question of whether the justification underlying a given method is acceptable is too general and abstract to be answered. But such a general acceptability of the justification of a method is not what our recommended model is about. Far from being a concrete challenge to our supposedly too abstract account, the ``orders of justification'' criticism only appears challenging because it raises an all too abstract problem. 

Our recommended model, as articulated above, does not refer to methods or methodological frameworks, it talks about what happens in concrete decision aiding processes (understood in the sense spelled out in \citep{tsoukias_concept_2007}). Decision aiding processes are concrete sets of continued interactions between decision analysts, decision-makers and concerned stakeholders. Because our recommended model is about concrete decision aiding processes, the important element in our approach is not the putative general justification underlying the methods used, but rather the justification that can be articulated for the specific usages of the methods put to use at this or that stage during the decision aiding process. Coming back to our example of environmental economic valuations, the various methods mentioned about can be used for very different purposes at various stages in concrete decision aiding processes. Stated and revealed preferences studies are often used to feed cost-benefit analyses \citep{layard_cost-benefit_1994}. However, as emphasized by \citep{meinard_ethical_2016}, the very same monetary valuations can just as well be used as arguments to strengthen public awareness of the importance of the object they value \citep{salles_valuing_2011}, or to put a provisional figure on the impact that various kinds of actions can have on various groups of stakeholders, or in many other ways. In these various cases, the justification that can be developped can take advantage of the general justification underlying the method used, but it can also integrate many other elements pertaining the context, and the specific usage of the method within the particular decision aiding process.

Claiming that our approach is impractical because it is hopeless to find decision aiding methods that will prove acceptable with respect to all the ``orders of justification'' is therefore irrelevant. This irrelevant criticism however suggest another, more powerful rejoinder that deserves to be addressed as well. This possible criticism would claim that, even within a concrete decision process, when a decision scientist sets himself to articulate a justification and fulfill the requirements i-v of our framework, it is highly likely that in most cases he will face at one stage or another someone who will stick to a given ``order of justification'', and whatever the decision scientist's effort to discuss the justification of his recommendation, and the reasons why some of the axioms underlying his interlocutor's ``order of justification'' should be abandonned, still his interlocutor will reject his justification.

We perfectly agree that such difficult situations can happen, and are prepared to admit that they might often happen. However, taking such situations to be fatal pitfalls for our approach would be confusing two largely disconnected issues: on the one hand, the issue that we address in this article, which is the one of the normative status of decision science interventions, and on the other hand, the issue of the possibility to generate consensual decisions. If a decision scientist in a concrete decision process articulates a justification for his recommendation but, whatever his efforts, he always faces the stubborn resistance of some groups, one cannot take this failure of consensus to prove that the decision scientist failed. If he neglected to articulate a justification, then he failed; if he developed a justification but a more justified recommendation was developped by someone else, then he failed; by contrast, if he pursued as far as he could the quest for justification but faced the resistance of someone sticking to moral realism with respect to a given ``order of justification'', then it cannot be said to have failed.

We do not deny that issues such as how consensual group decisions can be generated, or in which conditions will there be this or that pattern of moral realism among sets of groups of decision-makers and stakeholders, are important. Quite the contrary, we think that decision science approaches which would be able to tackle such issues would be considerably more justifiable. What we claim is that such issues go well beyond the requirements encapsulated in our recommended model of the proper place of decision science.

\section{Conclusion}
In this article, we have introduced a normative account of the role of decision sciences in a democracy. For that purpose, we have emphasized the failure of various strategies designed to allow decision scientists to eschew value-judgements, flight ``substance'' and remain normatively neutral. Such attempts take the crude forms of a failed model of economics as pure science and a failed model of decision science as a transparent norm translater, and the subtler form of a model that endorses pure proceduralism to arrange a place for a purportedly value-neutral decision science upstream democracy.

We have strived to demonstrate that all these models fail, which has led us to develop an inquiry into a basic structure of moral reasoning, unveiling an important contrast between ``moral realism'' and the ``quest for justifications''. When then argued that, if one endorses the tenets of ``incrementalism'' and ``primacy of practice'', and if one conceives of the justifiability of applications of decision sciences in a comparative approach, then decision sciences can find their place upstream democracy without having to endorse a moral realist understanding of the concepts of a ``good'' or ``acceptable'' justification. We accordingly ended up recommending the corresponding model for the proper role of decision sciences in a democracy, which can be articulated as follows:

As decision scientists, we should advocate a quest for justification that consists in:
\begin{itemize}
\item[i.]	Systematically justifying our recommendations;
\item[ii.]	Being ready to defend them against criticisms, even when none are formulated;
\item[iii.]	Actively eliciting criticisms, especially from people or groups that we have good reasons to think are not spontaneously liable or willing or able to articulate such criticisms;
\item[iv.]	Actually enacting this defense, when we actually face criticisms;
\item[v.]	Understanding our own justifiability in a comparative sense.
\end{itemize}

Though this models aims to provide a general account of the normative stance that we should take as decision scientists when we work in a democratic setting, and therefore is undoubtedly very ambitious in its scope, we emphasize that it is also very modest in many respects. It does not provide a metric, no generally applicable mechanical means to compare any two applications of decision sciences without discussions. There is bound to be myriads of hard cases where one application of decision science will appear better than another on some respect, but worse on another respect.

It is clearly part of our hope that this account of the normative stance of decision sciences will not come as a surprise for most decision scientists, and is rather liable to gather large support among them. However, though we accordingly expect that most decision scientist will agree with us at this stage, we surmise that most of them do not have clearly articulated ideas about the concrete implications of this basic normative stance. This is why we devoted considerable space to develop some concrete implications of our account, whose actual implementation might prove more disrupting for decision sciences practices than the cheer endorsement of our abstract normative account taken in its most abstract form.
 


%So far, the argument has been entirely abstract. This section will use a series of example to try to flesh it out to some extent.
%Let us start by a burning issue in environmental sciences, the current biodiversity crisis. The current rate of species extinctions is unprecedented. Many conservationist succumb to authoritarian temptations. Democracy looks poorly adapted to take bold decisions to save the planet. It seems like there is a choice to be made. Either we believe in democracy, participation, etc., in which case planet Earth will collapse; or an authoritarian regime will save the Planet and Humanity against its own willing. I think that such a debate is ill-conceived, and that the approach developed here solves the problem, at least to some extent. The idea that urgency trumps the need for participation and democracy in an argument like any other. If a justification based on this argument wins, then the decision to trump participation is not undemocratic, in any non-trivial sense.

%\commentOC{ Bof. Ta dissolution du débat joue
%sur l’ambiguïté du concept de démocratie. Si on prétend que
%la démocratie, c’est prendre les décisions en fonction des
%jugements délibérés du peuple, alors faire la procédure que
%tu proposes, c’est ne pas céder à l’urgence.}

%The same logic shows that this approach can be used to unlock some of the perennial apparent dilemma plaguing the functioning of democratic policies, due to the fact that the above mentioned output/input debate appears undecidable. A typical example in this respect is the case of the 1991 cancellation of the legislative elections in Algeria between the two rounds of the election, following the government’s understanding of the fact that it would certainly be overwhelmingly won by the “Front Islamic du Salut”, championing a largely undemocratic policy agenda. An output theory would call this decision legitimate, an input one would deem it undemocratic. Our approach claims that none of the alternatives (canceling the election or letting it unfold) is intrinsically legitimate, and none is intrinsically more legitimate than the other. The more legitimate one would be the one buttressed more thoroughly by a quest for justification on the part of its champions. This, clearly, is a local answer. Small scale debates here and there probably have unfolded and generated different winners in this respect. There is no general answer to this question. There are illuminating, well-structures local answers.


\section*{Acknowledgements}
We thank Jerome Lang, Philippe Grill and Juliette Rouchier for powerful comments and suggestions on this manuscript.

\section*{References}

\bibliography{decision,philo-eco,beliefs,deliber}

\appendix

\commentYM{elements of state-of-the-art to integrate: there are numerous papers within economics itself
which discuss the role and status of value judgments (Arrow 1951, Little 1952,
Buchanan 1959, Klappholz 1964, Blaug 1980[1995], Heath 1994, Some have
even considered economics not only to be a moral science, but a mere expression
of ideology (Myrdal 1954). And some have asserted that value judgments
are merely facts like any other (Ng 1972), or that, because the value judgments
typically made in welfare economics are uncontroversial, they may thus be considered
on a par with any positive assertion (Archibald 1959, Hennipman 1976).
Some have conducted contextual analyses of the importance of value judgment
in the contributions of various authors (e.g. Balisciano and Medema 1999, Davis
2005, Cherrier 2009, Andrews 2010), or textual analyses of the influence of value
judgments in allegedly neutral textbooks (e.g. Hill and Myatt 2010). Conversely,
and regrettably, fewer papers conduct minute methodological analyses aimed at
tracing the role of value judgments in economics (Mongin 2006); and, to my
knowledge, no such paper has been concerned specifically with the case of expertise.
We also have to find a way to talk about ``thin'' and ``thick'' predicates sensu Williams 1985, don't know where
Sen (1967: 50) defines a value judgment
as “basic” to a person if no conceivable revision of factual assumptions can
make him revise the judgment. If such revisions can take place, the judgment is
“non-basic” in his value system.
there are six phases to the decision
process, which we may present in chronological order as follows: First comes
(1) the scientific phase. The middle stage, expertise, contains four phases: (2) request,
(3) selection of the model, (4) application to the data, and (5) formulation of
a report, with descriptions, evaluations and prescriptions. The decision phase (6),
which comes last, may include a political process in the case of a public decision.
Mongin 2006: strong neutrality thesis = our section 2, weak neutrality = our third section}


\section{Democracy}
We will tackle the above question here only in the context of what we will call ``western, modern democracies''. This phrase will also be understood in a very broad sense. We do not want to confine this inquiry to a specific theory or vision of democracy. An immense literature in a large series of disciplines has produced a vest array of competing definitions of democracy. As our rationale will unfold in this article, we will be led from time to time to refer explicitly to several of these understandings of the term ``democracy''. But the question that we tackle is not anchored in any of these theories. This question owes part of its significance and importance to the fact that it uses the idea of “western, modern democracies” in the largely undetermined sense in which one uses it in ordinary conversation, broadly to refer to the kinds of institutions and political processes that historically emerged and matured in Europe and North-America, and some of the values and principles that are more or less naturally associated with them.
\commentOC{Conceptuellement, je ne vois pas à ce stade de ma réflexion pourquoi le propos est restreint aux démocraties. Stratégiquement aussi, c’est à mon avis une mauvaise idée. Je trouve le risque de confusion important, avec un terme aussi chargé. (Cf. notre discussion orale.)}
\commentYM{dans une large mesure, je suis d accord avec toi. Dans ma vision des choses, limiter à la démocratie permettait de restreindre le champ, un peu arbitrairement, mais d'une manière qui fait sens pour beaucoup de gens. Le risque si on ne restreint pas le champ est de devoir traiter plein de problématiques pour l'heure non évoquées, par exemple les questions de gouvernance dans les entreprises, qui ne sont pas dans le sujet. Ou alors les mettre de côté... sur la base d'argument ad hoc ?}
This debate indeed drives a wedge thanks to which decision science expertise can enter the scene by the backdoor and find its place upstream any political process. Indeed, whatever the conclusion of the debate, it interestingly frames the issues surrounding democracy in such a way that a role for decision science becomes visible upstream democracy. In the purely procedural approach, decision sciences can play a role in identifying and characterizing the supposedly pure procedure -- without thereby making any value-judgment. In the substantive approach, decision science can play a role in identifying the substance of democracy -- be it a matter of input or output. If a middle way has to be found between these two extremes, in this middle way the proper role of decision science is a bit of the two.
\commentOCf{Je ne comprends pas le rôle de DS ni dans l’une ni dans l’autre perspective.}

This third model hence appears to be liable to materialize in two, very different guises, depending on whether it is anchored in a substantive or a purely procedural stance. In this article, we will eventually recommand a specific version of this model. But identifying which one is to be recommended will first require some more theorizing, to which the two sections to come are devoted. 

\section{Decision science as a norm translater}
\commentOCf{Ce qui reste maintenant dans cette section me semble hors de propos, et je suggère de l’enlever.}

First, the exercise that consists in equating a value-judgment with an axiom is more difficult than most social choice theorists seem to admit, and is perhaps doomed to be elusive. To explain why, let us take \cite{arrow_social_2012}'s “Non Dictatorship” axiom. This axiom states that there is no individual \emph{i} in society such that for all profiles and all pairs of alternatives \emph{x} and \emph{y}, if \emph{i} prefers \emph{x} to \emph{y}, then \emph{x} is preferred to \emph{y} at social level. This axiom is expected to be largely accepted as a minimal requirement for any collective decision rule.

But there are two, very different reasons why one can be expected to endorse this axiom. The first reason is that this axiom and its name echo our shared endorsement of democracy and our shared refusal of dictatorship, where ``dictatorship'' refers to a complex picture of political arrangements, associated in intricate ways with notions such as arbitrariness, illegitimacy, and rule by force, and so on. Our shared refusal of dictatorship, understood in this sense, is also nourished by our culture and historical experiences. But plainly enough, none of these is captured by Arrow's axiom. This leaves us with a second reason to reject the axiom, one that tighly sticks to its formal articulation, which is that we would reject an arrangement, whatever it might be, that happens to satisfy perfectly one and only one individual, always the same. The second reason is the only one that the axiom strictly speaking talks about. But it is upon the first one that hinges the political relevance of the axiom, because the second does not even talk about political settings or ideas. 
\commentOC{Je ne comprends pas. Il faut définir plus précisément alors ce que tu entends par des idées ou settings politiques, et pourquoi l’interprétation restreinte de ND ne rentre pas dans cette catégorie, et pourquoi c’est important qu’il y rentre.}
\commentYM{il y avait une coquille. Je ne sais pas si c'est clair maintenant. Ce que je veux dire c'est que l'axiome a le cul entre deux chaises. Si on le lit à la lettre, il dit quelque chose qui n'a aucun interêt, ou du moins aucun sens politique. Or selon moi tout l'interet du theoreme est qu'il nous dit quelque chose de la politique, et il ne le fait que si on fait une lecture politique, et donc interpretative, des axiomes} 

In our view, this contrast shows that, although Arrow's axiom does capture shared normative intuitions, it certainly does not capture our shared refusal of dictatorship. Axioms undoubtedly are useful to formalize some aspects of value-judgments. But one cannot expect it to be possible to correctly and completely capture value-judgments or values in axioms. 
\commentOC{C’est tout autre chose, ça, non ? Et ça requiert une précision (qu’entends-tu par correctement et complètement ici ?) et une preuve. Et pourquoi serait-ce important ? On ne peut jamais rien faire parfaitement, probablement, mais ce n’est une objection à rien.} \commentYM{on se comprend pas sur cet argument. Ce que je dis est trivial, j'en suis bien conscient. Mais de fait dans la litterature economique il y a cette pretention débile. Mon objet est juste de dire qu'elle est débile.}
Articulating values is an endless task, using axioms for that purpose is an endless part of this endless task.

One might retort here that most social choice theorist would agree, and most of them eschew making the hopelessly ambitious claim to identify axioms capturing values or value-judgements perfectly. Even if this were true, a second argument should lead us to reject the vision of decision science as a norm translater. 

The above scheme through which social choice theorists pick-up axioms that correspond to the decision-maker's value-judgments therefore fails in both its stronger and its weaker version. In its weaker version, it fails because it is anchored in the unwarranted premise that decision-makers are able to make clear and definitive value-judgments once and for all, in the abstract. In its stronger version, on the top of that it also fails because it is anchored in yet another unwarranted premise: that it is undebatable that axioms can aptly capture value-judgments.
\commentOCf{Je ne comprends pas la différence entre les deux versions.}

\end{document}
